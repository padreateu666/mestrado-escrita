%====================================================================
% Resumo em inglês: Escreva no ambiente abaixo o resumo colocando 
% no final, via comando \palavraschaveingles, as palavras-chave
% em inglês. Não coloque ponto final no final das palavras-chave
% pois o comando já insere esse ponto.
%  
%  Exemplo:
%  
%  \begin{resumo-ingles}
%     Many engineering problems involve the analysis of the variation 
%     rate, i.e., the analysis of one or more physical derived 
%     properties over time and/or space.
%
%     \palavraschaveingles{Finite Volume Method, High Order Method}
%  \end{resumo-ingles}
%====================================================================

\begin{resumo-ingles}

    One of the main challenges in the context of applying Computational Fluid Dynamics ($CFD$) methods is understanding the methods and models used in simulating different types of flow. In this context, this study presents the methodology used to build a computational code aimed at solving Newtonian flows with and without thermal energy transfer, as well as non-Newtonian flows. Initially, a detailed description of the physical model of the flows used to validate the implemented routines is provided, after detailing the equations used to model the problems addressed in this work, along with a brief historical context and the specific aspects addressed in the literature regarding them. Subsequently, the numerical methods used to discretize the equations are detailed, as well as the mesh methodology and methods for solving linear systems. Finally, the results obtained from simulating the selected problems are presented, as well as their comparison with literature and analytical models. These results were consistent with the literature, particularly in non-Newtonian flows, where the influence of invariant choice on the quality of results for different flow regimes was observed. The code contributed to understanding the challenges of implementing models and the numerical limitations of the methods used. The work was developed at the Fluid Mechanics Laboratory of the Faculty of Mechanical Engineering at the Federal University of Uberlândia ($MFLab$-$FEMEC$-$UFU$).

\palavraschaveingles{Computational Fluid Dynamics ($CFD$), Non-Newtonian Fluids, Thermal Flows, Finite Volumes}
	
\end{resumo-ingles}
