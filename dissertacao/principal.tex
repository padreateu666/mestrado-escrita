%====================================================================
% Universidade Federal de Uberlândia
% Faculdade de Matemática
% Curso: Bacharelado em Estatística
% Modelo para elaboração de monografia referentes aos trabalhos de
% conclusão de curso do Bacharelado em Estatística
% Autor: Prof. Dr. Alessandro Alves Santana
%====================================================================
\documentclass[12pt,openright]{book}
%====================================================================
% Pacote de estilo que tem por finalidade carregar pacotes do LaTeX
% bem como definir configurações do texto da monografia. 
%====================================================================
\usepackage{estilo}
%====================================================================
% Inserção do arquivo com a lista de abreviaturas e símbolos.
%====================================================================
\makenomenclature
%====================================================================
% Documento Principal
%====================================================================
\begin{document}
\tolerance=5000
%====================================================================
% Para que as linhas do texto sejam numeradas com a finalidade 
% de facilitar a correção pelo orientador, basta descomentar o
% o comando abaixo. Esse comando faz com que o arquivo pdf gerado 
% após a compilação sem erros tenham as linhas da monografia sejam
% enumeradas. Com as linhas numeradas o orientador poderá localizar 
% e descrever melhor para os orientandos os locais do texto que 
% exigem correção.
%====================================================================
%\linenumbers
%====================================================================
% O comando \capa tem 6 argumentos de entrada os quais são 
% descritos abaixo. Tem por função gerar a capa da monografia. 
%
% \capa
% {título da monografia}
% {nome do orientando}
% {ano de defesa da monografia}
% {nome do orientador}
% {nome do segundo membro da banca}
% {nome do terceiro membro da banca}
%
% Observações: 
%
%  1- Escrever o título do trabalho em letras maíusculas.
%       Exemplo: ANÁLISE DE REGRESSÃO EM PROBLEMAS DE FÍSICA
%  2- Nome do aluno com iniciais em letras maiúsculas.
%       Exemplo: Alberto Barbosa da Cunha
%  3- Ano da defesa: 
%       Exemplo: 2018
%  4- Nome do Orientador e demais membros da banca de defesa: 
%     Escrever o nome apenas com as iniciais em letras maiúsculas 
%     como no nome do aluno.
%       Exemplo: João Carlos Silva de Almeida    
%====================================================================
\capa
{INTRODUÇÃO À MODELAGEM MATEMÁTICA E COMPUTACIONAL DE ESCOAMENTOS LAMINARES DE FLUIDOS NEWTONIANOS E NÃO-NEWTONIANOS COM EFEITOS TÉRMICOS}
{Luís Eduardo Silva Borges}
{2024}
{Prof. Dr. Aristeu da Silvera Neto}
{Prof. Dr. Rafael Alves Figueiredo}
{Prof. Dr. Elie Luis Martínez Padilla}
{Prof. Dr. João Rodrigo Andrade}
{Eng. Thiago Assis da Silva}
%====================================================================
% Agradecimentos: Os agradecimentos deverão ser escritos dentro do 
% arquivo agradecimentos.tex. 
%====================================================================     
% Observação: O arquivo agradecimentos.tex já existe dentro da 
% pasta monografia. O que o aluno precisa fazer é abrí-lo 
% e preenchê-lo com o texto dos agradecimentos.  
%====================================================================
\nomenclature[S]{$V_0$}{Velocidade de referência}
\nomenclature[S]{$L$}{Largura do domínio físico e computacional}
\nomenclature[S]{$H$}{Altura do domínio físico e computacional}
\nomenclature[S]{$\rho$}{Massa específica do fluido}
\nomenclature[S]{$\vec{v}$}{Vetor velocidade}
\nomenclature[S]{$t$}{Tempo}
\nomenclature[s]{$\vec{g}$}{Vetor campo gravitacional}
\nomenclature[s]{$\overline{\overline{\pi}}$}{Tensor das tensões}
\nomenclature[s]{$\overline{\overline{\delta}}$}{Tensor unitário}
\nomenclature[s]{$p$}{Pressão}
\nomenclature[s]{$\overline{\overline{\tau}}$}{Tensor das Tensões viscosas}
\nomenclature[s]{$T$}{Temperatura}
\nomenclature[s]{$\alpha$}{Difusividade térmica}
\nomenclature[s]{$\phi$}{Tensor taxa de transformação de energia cinética em térmica}
\nomenclature[s]{$C_p$}{Calor específico}
\nomenclature[s]{$k$}{Condutividade térmica}
\nomenclature[s]{$\mu$}{Viscosidade dinâmica}
\nomenclature[s]{$\nu$}{Viscosidade cinemática}
\nomenclature[s]{$\dot{\gamma}$}{Tensor taxa de cisalhamento}
\nomenclature[s]{$m$}{Fator de condicionamento da Lei de potência}
\nomenclature[s]{$n$}{Expoente da Lei de Potência}
\nomenclature[s]{$\beta$}{Coeficiente de expansão volumétrica}
\nomenclature[s]{$D$}{Comprimento característico do escoamento}
\nomenclature[s]{$U$}{Velocidade característico do escoamento}
\nomenclature[A]{$Re$}{Número de Reynolds}
\nomenclature[A]{$Pr$}{Número de Prandtl}
\nomenclature[A]{$Gr$}{Número de Grashof}
\nomenclature[A]{$Ra$}{Número de Rayleigh}
\nomenclature[s]{$\eta$}{Viscosidade não-newtoniana}
\nomenclature[s]{$p_0$ }{Campo de pressão inicial}
\nomenclature[s]{$\vec{V^\ast}$}{Campo de Velocidade Aproximado}
\nomenclature[s]{$\Delta y$}{Altura da célula computacional}
\nomenclature[s]{$\Delta t$}{Passo de tempo}
\nomenclature[s]{$\Delta x$}{Largura da célula computacional}
\nomenclature[A]{$CFL$}{Número de Courant-Friedrichs-Lewy}
\nomenclature[A]{$DFC$}{Dinâmica dos fluidos computacional}
\nomenclature[A]{$CFD$}{Computational Fluid Dynamics}
\nomenclature[A]{$MFLab$}{Laboratório de Mecânica dos Fluidos}
\nomenclature[A]{$FEMEC$}{Faculdade de Engenharia Mecânica}
\nomenclature[A]{$UFU$}{Universidade Federal de Uberlândia}
\nomenclature[s]{$P^\prime$}{Correção do campo de pressão}
%====================================================================
% Agradecimentos: Escreva no ambiente abaixo os agradecimentos.
%  
%  Exemplo:
%  
%  \begin{agradecimentos}
%     Agradeço a toda minha família e a todos os meus amigos 
%     pelo apoio nessa importante fase do meu processo de formação
%     profissional.
%  \end{agradecimentos}
%====================================================================

\begin{agradecimentos}

Aos meus pais Rejane Aparecida e Wilson Borges, por não medirem esforços para tornar possível minha graduação e todos os sonhos que já tive e ainda vou ter.\\

À minha irmã, Maria Clara, que sempre esteve ao meu lado e fez de tudo para me apoiar nos momentos em que precisei.\\

Ao meu orientador, professor Aristeu, por sua dedicação ao ensino e a pesquisa, por sempre estar disposto a sanar minhas dúvidas, por despertar meu interesse pela área de mecanica dos fluidos e pelo exemplo que é a todos os seus alunos.\\

Ao meu coorientador, professor Rafael, por sua dedicação, pelo bom humor, companheirismo e por sempre estar disposto e disponível para ajudar de todas as formas possíveis.\\

Aos meus professores do curso de Engenharia Mecânica, em especial aos professores Hércio Cândido e Jocelino Sato, que além de manterem vivo meu interesse pelas ciências exatas contribuiram para minha formação como engenheiro e como ser humano.\\

Aos meus amigos do ensino médio, Adrio, Breno, Enoch, Keldson, Lucas e Marcus. Por manterem essa amizade mesmo depois do ensino médio e pelo tempo de descontração. \\

Aos meus amigos de graduação, Saulo, João e Emelly. Pelas risadas, ajuda e conversas ao longo do curso.\\

Aos amigos que fiz durante esta jornada, incluindo Vitor, Thiago, Ingrid, Letícia, Patrícia, Verônica e todos da equipe do LAMAU, pela parceria e apoio.\\

Ao meu amigo Vinicios Zils, pelo companheirismo e amizade. \\

À Maria Clara Agliardi Oliveira, pelo amor, carinho e companheirismo durante todos esses anos de curso. \\

À todos que me ajudaram nesta jornada.\\





















\end{agradecimentos}






\thispagestyle{empty}
\vspace*{\fill}
\begin{flushright}
\textit{"O sonho de uma pessoa nunca acaba."} \\
\textit{— Marshall D. Teach}
\end{flushright}
%====================================================================
% Resumo em português: O resumo em português deverá ser escrito 
% dentro do arquivo resumo.tex.
%====================================================================
% Observação: O arquivo resumo.tex já existe dentro da 
% pasta monografia. O que o aluno precisa fazer é abrí-lo 
% e preenchê-lo com o texto do resumo. Não esqueça de 
% de colocar as palavras-chave em português. No arquivo 
% resumo.tex isso poderá ser observado. 
%====================================================================
%====================================================================
% Resumo em português: Escreva no ambiente abaixo o resumo colocando 
% no final, via comando \palavraschaveportugues, as palavras-chave
% em português. Não coloque ponto final no final das palavras-chave
% pois o comando já insere esse ponto.
%  
%  Exemplo:
%  
%  \begin{resumo-portugues}
%     O presente trabalho foi desenvolvido com a finalidade de 
%     realizar um estudo sobre diversas metodologias de análise 
%     de dados.
%
%     \palavraschaveportugues{Análise de dados, inferência}
%  \end{resumo-portugues}
%====================================================================

\begin{resumo-portugues}

Um dos principais desafios no contexto de aplicação dos métodos de Dinâmica dos Fluidos Computacional ($DFC$), é a compreensão dos métodos e modelos utilizados nos diferentes tipos de escoamentos simulados. Diante deste contexto, no presente trabalho apresenta-se a metodologia utilizada para a construção de um código computacional voltado à solução de escoamentos newtonianos com e sem a transferência de energia térmica, além de escoamentos não-newtonianos. Inicialmente, apresentou-se um detalhamento a respeito do modelo físico dos escoamentos utilizados para validar as rotinas implementadas, após ter detalhado as equações para modelar os problemas abordados no presente trabalho, trazendo um breve contexto histórico dos mesmos, bem como as especificidades tratadas na literatura a respeito destes. Após esta etapa foram detalhados os métodos numéricos utilizados para discretizar as equações, bem como a metodologia de malha e os métodos para solução de sistemas lineares. Por fim, são apresentados os resultados obtidos a partir da simulação dos problemas elencados, bem como a comparação destes com a literatura e com modelos analíticos. Os mesmos se mostraram coerentes com a literatura, em específico, nos escoamentos não-newtonianos, onde observou-se a influência na escolha do invariante na qualidade dos resultados para os diferentes regimes de escoamento. O código corroborou para o entendimento dos desafios da implementação dos modelos e das limitações numéricas dos métodos utilizados. O trabalho foi desenvolvido no Laboratório de Mecânica dos Fluidos da Faculdade de Engenharia Mecânica da Universidade Federal de Uberlândia ($MFLab$-$FEMEC$-$UFU$).

\palavraschaveportugues{ Dinâmica dos Fluidos Computacional, Fluidos Não-Newtonianos, Escoamentos Térmicos, Volumes finitos}
	
\end{resumo-portugues}


%====================================================================
% Resumo em inglês: O resumo em inglês deverá ser escrito dentro 
% do arquivo abstract.tex.
%====================================================================
% Observação: O arquivo abstract.tex já existe dentro da 
% pasta monografia. O que o aluno precisa fazer é abrí-lo 
% e preenchê-lo com o texto do abstract. Não esqueça de 
% de colocar as palavras-chave em inglês. No arquivo 
% abstract.tex isso poderá ser observado. 
%====================================================================
%====================================================================
% Resumo em inglês: Escreva no ambiente abaixo o resumo colocando 
% no final, via comando \palavraschaveingles, as palavras-chave
% em inglês. Não coloque ponto final no final das palavras-chave
% pois o comando já insere esse ponto.
%  
%  Exemplo:
%  
%  \begin{resumo-ingles}
%     Many engineering problems involve the analysis of the variation 
%     rate, i.e., the analysis of one or more physical derived 
%     properties over time and/or space.
%
%     \palavraschaveingles{Finite Volume Method, High Order Method}
%  \end{resumo-ingles}
%====================================================================

\begin{resumo-ingles}

    One of the main challenges in the context of applying Computational Fluid Dynamics ($CFD$) methods is understanding the methods and models used in simulating different types of flow. In this context, this study presents the methodology used to build a computational code aimed at solving Newtonian flows with and without thermal energy transfer, as well as non-Newtonian flows. Initially, a detailed description of the physical model of the flows used to validate the implemented routines is provided, after detailing the equations used to model the problems addressed in this work, along with a brief historical context and the specific aspects addressed in the literature regarding them. Subsequently, the numerical methods used to discretize the equations are detailed, as well as the mesh methodology and methods for solving linear systems. Finally, the results obtained from simulating the selected problems are presented, as well as their comparison with literature and analytical models. These results were consistent with the literature, particularly in non-Newtonian flows, where the influence of invariant choice on the quality of results for different flow regimes was observed. The code contributed to understanding the challenges of implementing models and the numerical limitations of the methods used. The work was developed at the Fluid Mechanics Laboratory of the Faculty of Mechanical Engineering at the Federal University of Uberlândia ($MFLab$-$FEMEC$-$UFU$).

\palavraschaveingles{Computational Fluid Dynamics ($CFD$), Non-Newtonian Fluids, Thermal Flows, Finite Volumes}
	
\end{resumo-ingles}

%====================================================================
% Comando para inclusão do sumário
%====================================================================
\tableofcontents
\thispagestyle{empty}
\cleardoublepage
%====================================================================
% Comando para inclusão da lista de figuras.
%==================================================================== 
% Observação: Caso o trabalho não tenha lista de figuras, comente 
% as 3 linhas abaixo.
%====================================================================
\pagenumbering{Roman}
\listoffigures
\cleardoublepage
%====================================================================
% Comando para inclusão da lista de tabelas.
%====================================================================
% Observação: Caso o trabalho não tenha lista de tabelas, comente 
% as 2 linhas abaixo.
%====================================================================
\listoftables
\cleardoublepage
%====================================================================
% Comando para inclusão da lista de abreviações e símbolos.
%====================================================================
% Observação: Caso o trabalho não tenha lista de abreviações, 
% ou lista de símbolos, comente as 2 linhas abaixo.
%====================================================================
\printnomenclature
\cleardoublepage
%====================================================================
% Capítulos da monografia.
%====================================================================
\onehalfspacing
\clearpage
\pagenumbering{arabic}
\pagestyle{cabecalhorodape}
%====================================================================
% Introdução.
%====================================================================
%====================================================================
% Introdução: Escreva logo após o \chapter{Introdução} a texto 
% de seu trabalho referente a introdução.
%====================================================================
\chapter{Introdução}



%====================================================================
% Desenvolvimento.
%====================================================================
%====================================================================
% Desenvolvimento: Escreva logo após o \chapter{Desenvolvimento} a 
% texto de seu trabalho referente ao desenvolvimento. 
%====================================================================
\chapter{Modelos Físicos}

O código desenvolvido tem como objetivo resolver problemas cuja natureza física possa ser descrita por um domínio bidimensional cartesiano e que apresente comportamento de escoamento incompressível. Apesar dessa generalidade, cada tipo de escoamento apresenta certas peculiaridades, que serão apresentadas adiante.

\section{Escoamento de Couette Plano}

Na Figura \ref{Couettefisico} apresenta-se o modelo físico do problema conhecido na literatura como escoamento de Couette plano \citep{Gibson2008,Case1960,Orszag1980,Bottin1998,Barkley2007} e suas condições de contorno. O problema se trata de um canal plano, cujas placas superior e inferior se movem com diferentes velocidades, assim o que determina a dinâmica do escoamento é a velocidade relativa resultante entre as fronteiras. 

No presente trabalho foi considerada a fronteira superior se movendo com uma velocidade constante $V_0$ e a fronteira inferior com velocidade nula, caracterizando condições de contorno de Dirichlet, considerando também a condição de não deslizamento entre as paredes e o fluido. Nas paredes laterais do escoamento bidimensional, foram consideradas condições de contorno de Neumann, com derivada nula em ambas as componentes de velocidade para representar a condição de simetria. A Figura abaixo mostra o modelo físico do escoamento de Couette.

\begin{figure}[ht]
    \centering
    \includegraphics[width=0.7\textwidth]{imagens/Couette-fisico.png}
    \caption{Modelo Físico do escoamento de Couette.}
    \label{Couettefisico}
    {\footnotesize Fonte: Do próprio autor} % Aqui vai a fonte da figura
\end{figure}

As fronteiras inferior e superior são separadas por uma altura $H$ e possuem uma largura $L$ de forma que $L\ =\ H$. O eixo de coordenadas foi adotado no canto inferior esquerdo do domínio computacional.

Para se manter a coerência numérica que será demonstrada posteriormente, as condições de contorno foram impostas sobre os campos de pressão, sendo mantida a condição de Dirichlet nas fronteiras à esquerda e à direita e condições de segunda espécie das fronteiras superior e inferior.

Foi considerado um escoamento em regime transiente, com condição inicial de velocidade nula em todo o domínio, com exceção das fronteiras superior e inferior. O campo de pressão inicial em todo o escoamento foi considerado uniforme.

O escoamento de Couette foi um dos utilizados para validar o modelo de fluídos não-newtonianos, nos quais as condições de contorno e inicial adotadas são as mesmas do escoamento newtoniano.

Apesar deste escoamento ser popularmente reconhecido como escoamento de Couette, se considera escoamentos de Couette a classe de escoamentos que possuem fronteiras móveis, como tampas ou cilindros rotativos.

\section{Escoamento de Poiseuille Plano}

Outro escoamento comumente encontrado na literatura para a validação de rotinas computacionais, é o promovido pela diferença de pressão no interior de um canal, anular ou plano. Este problema é normalmente chamado de escoamento de Poiseuille \citep{Thomas1953,Bharti2007}, em homenagem ao médico e pesquisador de mesmo nome, que promoveu estudos em escoamentos sanguíneos, que podem ser considerados escoamentos de Poiseuille.

Na Figura \ref{poiseuillefisico} é mostrado o domínio físico bem como as condições de contorno do escoamento de Poiseuille consideradas no presente trabalho:

\begin{figure}[ht]
    \centering
    \includegraphics[width=0.6\textwidth]{imagens/Poiseulle-fisico.png}
    \caption{Modelo Físico do escoamento de Poiseuille.}
    \label{poiseuillefisico}
    {\footnotesize Fonte: Do próprio autor} % Aqui vai a fonte da figura
\end{figure}


Foi considerado um escoamento de Poiseuille em desenvolvimento, com um perfil uniforme de velocidade na entrada, a esquerda, e um perfil totalmente desenvolvido com condição de derivada nula, a direita. Nas fronteiras inferior e superior foram consideradas condições de parede, respeitando o critério de não-deslizamento. As fronteiras inferior e superior são separadas por uma distância $H$ e o canal tem uma largura $L$, de forma que $H \neq\ L$. O centro do sistema de coordenadas foi colocado no canto inferior esquerdo do domínio físico.

Assim foi considerada condição de primeira espécie nas fronteiras superior, inferior e esquerda para todas as componentes de velocidade e condição de Neumann para a parede direita. As condições de contorno para a pressão foram de primeira espécie para a parede direita e segunda espécie para as demais.

A condição inicial do escoamento foi um perfil de velocidade uniforme igual à velocidade de entrada e um gradiente de pressão uniforme em todo o escoamento. Este escoamento foi utilizado para a validação dos modelos não-newtonianos, utilizando a solução contínua. As condições de contorno consideradas são as mesmas para os fluídos newtonianos

Como no caso do escoamento de Couette, a classe dos escoamentos de Poisseulle são todos aqueles escoamentos promovidos por uma diferença de pressão entre a entrada e a saída do escoamento.

\section{Escoamento em Cavidade com Tampa Deslizante}

O escoamento de cavidade com tampa deslizante é um tipo de escoamento de Couette, ou com fronteira móvel, que possui condições de parede, de velocidade nula, em todas as direções do escoamento, com exceção da fronteira superior móvel. 

É mais um tipo de escoamento utilizado para a validação de códigos próprios desenvolvidos para escoamentos, sendo mais complexo que os anteriores por apresentar recirculações e a presença de maiores gradientes de pressão e velocidade. O domínio físico e condições de contorno são mostrados na Figura \ref{lid-drivenfisico}.

\begin{figure}[h]
    \centering
    \includegraphics[width=0.65\textwidth]{imagens/LId-driven-Fisico.png}
    \caption{Modelo Físico da cavidade com tampa deslizante.}
    \label{lid-drivenfisico}
    {\footnotesize Fonte: Do próprio autor} % Aqui vai a fonte da figura
\end{figure}

A cavidade possui dimensões iguais na largura e na altura, com condições de contorno de primeira espécie para as componentes de velocidade em todas as direções. A pressão possui condição de segunda espécie em todas as fronteiras. 

A condição inicial foi de velocidade nula em todo o escoamento com exceção da fronteira superior, respeitando a condição de não deslizamento. A condição inicial de pressão é de um gradiente de pressão uniforme em todo o escoamento.

Modelos e estudos sobre este tipo de escoamento podem ser encontrados em diversos autores como \citet{Santos2022} \citet{Vasconcellos2024} e \citet{Ghia1982}.


\section{Escoamento em Cavidade Diferencialmente Aquecida}

Um dos escoamentos térmicos mais simples que se encontra na literatura é o da cavidade com convecção natural, ou cavidade diferencialmente aquecida. Este escoamento tem o movimento do fluido promovido pela variação da massa específica, devido a uma diferença de temperatura ao longo domínio, que leva a uma diferença de pressão que bombeia uma recirculação dentro da cavidade. 

A Figura \ref{thermal-fisico} mostra o modelo físico do escoamento, no qual, diferente dos demais, possui condições para a temperatura, que é um campo escalar ao longo do escoamento.

\begin{figure}[ht]
    \centering
    \includegraphics[width=0.65\textwidth]{imagens/thermal-fisico.png}
    \caption{Modelo Físico da cavidade diferencialmente aquecida.}
    \label{thermal-fisico}
    {\footnotesize Fonte: Do próprio autor} % Aqui vai a fonte da figura
\end{figure}

As componentes da velocidade recebem condições de primeira espécie com velocidade nula em todo o escoamento, de forma que a pressão, para manter a coerência dos modelos, recebe condições de Neuman em todo o escoamento. 

A temperatura recebe condições de primeira espécie para as fronteiras laterais, o que garante a condição de paredes diferencialmente aquecidas, e condições de segunda espécie nas fronteiras superior e inferior. A condição de segunda espécie para a temperatura, representa uma condição de parede isolada, ou adiabática, na qual não ocorre troca de energia térmica com o meio.

Novamente as fronteiras são quadradas, com uma dimensão de L e H idênticas. As condições iniciais para o problema são as componentes de velocidade nulas, um gradiente de pressão uniforme e uma temperatura uniforme ao longo do escoamento igual a média da temperatura entre as paredes laterais.

%====================================================================
% Fundamentação Teórica. 
%====================================================================

\chapter{Modelo Matemático Diferencial}

Para representar adequadamente qualquer fenômeno físico por meio de modelos matemáticos, é preciso que se adote hipóteses simplificadoras. Essas hipóteses são necessárias pois, na maioria dos casos, a realidade é apenas parcialmente conhecida e mensurável pelos cientistas \citep{Aristeu2020}. Essas simplificações permitem que modelos matemáticos ofereçam soluções aproximadas, porém valiosas para problemas complexos, refletindo aspectos fundamentais do comportamento dos sistemas físicos.

O estudo do escoamento de fluidos desperta interesse na humanidade desde civilizações antigas, como a dos gregos e romanos. A primeira formulação matemática sobre o movimento dos fluidos foi desenvolvida por Leonard Euler.  No entanto, conceitos matemáticos mais avançados, utilizados para representar esse comportamento com maior precisão, especialmente em problemas complexos como os industriais que enfrentamos até os dias de hoje, foram desenvolvidos e aprimorados séculos depois por grandes cientistas como Osborne Reynolds, George Stokes e Augustin-Louis Cauchy \citep{Fortuna2000}. Esses modelos têm como objetivo descrever o balanço de todas as propriedades vetoriais e escalares relevantes no domínio físico do escoamento.

As equações fundamentais que representam os escoamentos são amplamente discutidas e desenvolvidas por diversos autores para diferentes propósitos, como \citep{Aristeu2020,Fortuna2000,Bird1987,White2011,Malalasekera2007}. Essencialmente, são utilizadas ao menos três equações para o problema de fluidos, a primeira é descrita pela Equação \ref{eq:Equação da continuidade}:

\begin{equation}
    \frac{\partial\rho}{\partial t}+\nabla\cdot\left(\rho\vec{v}\right)=0	
    \label{eq:Equação da continuidade}
\end{equation}

Tal equação é chamada de equação da continuidade, na qual $\frac{\partial\rho}{\partial t}$ representa a variação da massa especifica ao longo do tempo, $\vec{v}$ representa o vetor velocidade e $\rho$ representa a própria massa específica do fluido. Tal modelo representa o balanço de massa em todo o domínio físico, respeitando a lei de Lavoisier na qual há a conservação da massa total, quando se considera tanto o ambiente do escoamento quanto o ambiente externo. O operador $\nabla\cdot$ representa a operação vetorial chamada de divergente.

A segunda formulação utilizada é descrita pela Equação \ref{eq:continuidade-completa}.

\begin{equation}
    \rho\frac{\partial\vec{v}}{\partial t}+\rho\left(\vec{v}\cdot\nabla\right)\vec{v}=-\nabla\cdot\overline{\overline{\pi}}+\rho\vec{g}
    \label{eq:continuidade-completa}
\end{equation}

Esta equação é chamada de equação de Cauchy \citep{Aristeu2020} e descreve o balanço da quantidade de movimento linear. Esta equação resulta da aplicação da segunda lei de Newton no teorema do transporte de Reynolds em sua forma diferencial \cite{Aristeu2020,Fortuna2000,Bird1987}, que estabelece que a taxa de variação do momento linear de um fluido é igual à soma das forças que atuam sobre ele \citep{Fortuna2000}. Sua dedução, assim como a da equação do balanço de massa é apresentada em detalhes por \citet{White2011} e \cite{Aristeu2020}.

Na equação de Cauchy o termo $\vec{g}$ representa o campo gravitacional, o termo $\frac{\partial\vec{v}}{\partial t}$ representa o acúmulo de momentum linear no domínio do escoamento e o termo $\rho\left(\vec{v}\cdot\nabla\right)\vec{v}$ representa o transporte advectivo de quantidade de movimento linear. O tensor $\overline{\overline{\pi}}$ representa os efeitos de pressão e viscosidade presentes no escoamento \citep{Aristeu2020} e sua modelagem depende da abordagem utilizada para representar a relação entre as forças e a deformação no fluido. No caso de escoamentos newtonianos, o tensor de tensões é diretamente proporcional ao gradiente de velocidade o que resulta nas equações de Navier-Stokes. Já para fluidos não-newtonianos, essa relação é complexa, e diferentes modelos constitutivos são necessários para descrever adequadamente os efeitos das tensões viscosas. 

De forma geral o tensor $\overline{\overline{\pi}}$, pode ser escrito como a soma de duas parcelas, Eq. \ref{eq:tensor-pi}:

\begin{equation}
\overline{\overline{\pi}}=\ p\overline{\overline{\delta}}+\ \overline{\overline{\tau}}
    \label{eq:tensor-pi}
\end{equation}

Na Eq. \ref{eq:tensor-pi} p representa a pressão atuante no escoamento, normal a cada uma das direções e $\overline{\overline{\delta}}$ representa é o tensor identidade, de forma que a parcela $p\overline{\overline{\delta}}$ pode ser escrita como mostra a, Eq. \ref{eq:tensor-p}:

\begin{equation}
    p\overline{\overline{\delta}} =\ \left(\begin{matrix}p&0&0\\0&p&0\\0&0&p\\\end{matrix}\right)
    \label{eq:tensor-p}
\end{equation}

Na equação \ref{eq:tensor-pi}, $\overline{\overline{\tau}}$ é chamado de o tensor das tensões viscosas, que deve ser modelado de acordo com o modelo de fechamento adotado e representa transferência de quantidade de movimento linear por meio das forças viscosas presentes no fluido.

A última equação que será descrita, se limita a escoamentos com efeitos térmicos, Eq. \ref{balanço-termico}.

\begin{equation}
    \frac{\partial T}{\partial t}+\vec{v}\mathrm{\nabla}\cdot T=\alpha\mathrm{\nabla}^2T+\frac{\phi}{\rho C_p}
    \label{balanço-termico}
\end{equation}

A Eq.\ref{balanço-termico} é o chamado balanço de energia térmica \citep{Aristeu2020,wylen2016}, na qual $\frac{\partial T}{\partial t}$ representa a variação de energia térmica em uma partícula de fluido ao longo do tempo, ou o acumulo de energia térmica, $\nabla\cdot\left(\vec{v}T\right)$ representa o fluxo liquido advectivo de energia térmica por meio das componentes do vetor velocidade, também chamado de transporte advectivo de energia térmica, $\alpha\nabla^2T$ representa o fluxo líquido difusivo de energia térmica, por este motivo é chamado de termo difusivo da equação da energia.

O termo $\phi$ representa a transformação de energia cinética em energia térmica, por meio dos efeitos viscosos, tal termo não foi considerado na análise utilizada pois sua modelagem apenas se torna relevante em escoamento onde os efeitos viscosos são muito fortes e provocam aquecimentos relevantes. Exemplos deste tipo de problemas são o escoamento de lubrificante no interior de um mancal, onde as dimensões são muito pequenas, ou na reentrada de um corpo na atmosfera, no qual as velocidades são muito altas. O escalar $\alpha$ é chamado de difusividade térmica, e representa a junção de três termos: $\rho$,$C_p$ que representa o calor específico do material e $k$ que representa a condutividade térmica do mesmo.

\section{Escoamentos Newtonianos}

Para fluidos newtonianos incompressíveis, que obedecem a lei da viscosidade de Newton e não possuem uma variação de velocidade grande o suficiente para invalidar a hipótese de incompressibilidade, o tensor $\overline{\overline{\tau}}$ pode ser modelado utilizando o modelo de Stokes. Tal modelo relaciona o tensor das tensões com o gradiente do vetor velocidade, seguindo o mesmo princípio da lei da viscosidade de Newton. A dedução das equações que modelam este tipo e fluido podem ser encontradas em detalhes em \citet{Bird1987} e \cite{White2011}, são elas, Eq.\ref{eq:continuidade-reduzida} a \ref{eq:definição rho}:

\begin{align}
    &\vec{\nabla}\cdot\vec{v}\ =\ 0 \label{eq:continuidade-reduzida}\\
	&\frac{\partial\vec{v}}{\partial t}+\left(\vec{v}\cdot\vec{\nabla}\right)\vec{v}=-\frac{1}{\rho}\vec{\nabla}p+\ \frac{1}{\rho}\vec{\nabla}\cdot[\ \nu\ (\vec{\nabla}\vec{v}+\vec{\nabla}{\vec{v}}^T]+g\\
	&\overline{\overline{\tau}}=-\mu\left(\vec{\nabla}\vec{v}+\vec{\nabla}{\vec{v}}^T\right) \nonumber\\
	&\overline{\overline{\tau}}=-\mu\dot{\gamma}\\
	&\dot{\gamma}\ =\ \left(\vec{\nabla}\vec{v}+\vec{\nabla}{\vec{v}}^T\right)\\
	&\nu=\frac{\mu}{\rho} \label{eq:definição rho}
\end{align}

Nas equações consideradas $\mu$ é a viscosidade dinâmica e $\nu$ a viscosidade cinemática do fluido.  O tensor $\dot{\gamma}$ é chamado de tensor taxa de cisalhamento \citep{Bird1987}, ele é proporcional ao gradiente e ao gradiente transposto de velocidade, em alusão à lei da viscosidade de newton este tensor\ pode ser escrito da seguinte forma, Eq. \ref{eq:gama-definição}:

\begin{equation}
     \dot{\gamma}\ =\ \vec{\nabla}\vec{v}+\vec{\nabla}{\vec{v}}^T\ \ =\left(\begin{matrix}2\frac{\partial u}{\partial x}&\frac{\partial u}{\partial y}\ +\ \frac{\partial v}{\partial x}&\frac{\partial w}{\partial x}\ +\ \frac{\partial u}{\partial z}\\\frac{\partial u}{\partial y}\ +\ \frac{\partial v}{\partial x}&2\frac{\partial v}{\partial y}&\frac{\partial w}{\partial y}\ +\ \frac{\partial v}{\partial z}\\\frac{\partial u}{\partial z}\ +\ \frac{\partial w}{\partial x}&\frac{\partial w}{\partial y}\ +\ \frac{\partial v}{\partial z}&2\frac{\partial w}{\partial z}\\\end{matrix}\right)
     \label{eq:gama-definição}
\end{equation}

Ou na forma bidimensional, Eq. \ref{eq:gama-bidimensional}: 

\begin{equation}
    \dot{\gamma}\ =\vec{\nabla}\vec{v}+\vec{\nabla}{\vec{v}}^T=\left(\begin{matrix}2\frac{\partial u}{\partial x}&\frac{\partial u}{\partial y}\ +\ \frac{\partial v}{\partial x}\\\frac{\partial u}{\partial y}\ +\ \frac{\partial v}{\partial x}&2\frac{\partial v}{\partial y}\\\end{matrix}\right)
    \label{eq:gama-bidimensional}
\end{equation}

A abordagem utilizada para o fechamento da equação de Cauchy e a consequente modelagem do tensor de tensões de Cauchy foi inicialmente proposta por Stokes \citep{White2011}. Esta formulação, é válida para fluidos newtonianos e pode ser simplificada de forma a eliminar as derivadas cruzadas, esta simplificação não será apresentada no trabalho, pois a forma completa do tensor taxa de cisalhamento é mais útil para as formulações utilizadas no trabalho

\section{Fluidos Não-Newtonianos}

A lei da viscosidade de Newton, descreve o tensor $\overline{\overline{\tau}}$ como sendo uma relação linear entre um fator de correção ($\mu$ no caso de fluidos newtonianos) e a taxa de cisalhamento ($\dot{\gamma}$). A principal diferença entre esse comportamento ao dos fluidos não-newtonianos é a não linearidade desta relação, o que faz com que surja a necessidade de outras equações de ajuste desta relação em função de $\dot{\gamma}$. Na Figura \ref{nn-white} são mostrados os comportamentos típicos de vários exemplos destes fluidos.


\begin{figure}[ht]
    \centering
    \includegraphics[width=0.4\textwidth]{imagens/white-viscosidade-pag43.png}
    \caption{Comportamento geral dos diferentes tipos de fluido de acordo com a taxa de deformação.}
    \label{nn-white}
    {\footnotesize Fonte: \citet{White2011} pag.43}
\end{figure}


Na Figura apresentada por \citet{White2011} é possível observar as principais classes de fluidos não-newtonianos, os quais divergem entre si pelo tipo de comportamento não linear da tensão de cisalhamento ($\tau$).

O comportamento característico dos fluidos pseudoplásticos é que eles diminuem sua resistência ao cisalhamento com o aumento do mesmo, exemplos desses fluido são o plasma sanguíneo e tinta. Os fluidos plásticos são os que possuem o mesmo comportamento dos pseudoplásticos de forma mais atenuada. O caso limite desses fluido são os plásticos de Bingham, que necessitam de uma tensão inicial grande o suficiente para que comecem a escoar \citep{White2011}. 

Os fluidos dilatantes possuem comportamento oposto aos plásticos, e aumentam sua resistência com o aumento da taxa de cisalhamento, um exemplo dessa classe de fluido é a areia movediça. O presente trabalho não teve como foco nenhum desses modelos específicos, fazendo uma abordagem geral no limite de representatividade dos modelos adotados.

Para se modelar a relação não linear entre viscosidade e taxa de cisalhamento, diversos tipos de abordagens e modelos podem ser encontrados na literatura, de forma que não existe um consenso de um modelo geral representativo de todos os tipos de fluidos. Dentre os modelos mais utilizados está a família de modelos chamada de modelos de fluidos generalizados, que tentam relacionar a viscosidade com o tensor taxa de cisalhamento, como demonstrado na Eq. \ref{tensor-tau}:

\begin{equation}
    \overline{\overline{\tau}}=-\eta\left(\dot{\gamma}\right)\dot{\gamma}
    \label{tensor-tau}
\end{equation}

No qual $\eta\left(\dot{\gamma}\right)$ é chamado de viscosidade não-newtoniana e depende de $\dot{\gamma}$.  A função que define o tensor taxa de cisalhamento depende de cada modelo adotado. Esta formulação busca adotar correlações empíricas para cada um dos fluidos estudados de forma que se consiga ajustar parâmetros para representar seu comportamento físico por meio desta viscosidade. A principal diferença entre os modelos de fluido generalizado é a quantidade de parâmetros a serem ajustados, sendo que de forma geral, quanto mais parâmetros a serem ajustados mais representativo é o modelo, porém mais caro é a experimentação necessária em termos de recursos e tempo.

Dentre as principais variações do modelo de fluido generalizado, as mais utilizadas são a Lei de potência (em inglês \textit{Power Law}) e a chamada Carreau-Yasuda, desenvolvida pelos autores que levam seu nome \citep{Bird1987}.

A Figura \ref{nn-bird} apresentada por \citet{Bird1987} mostra dados experimentais da variação de $\eta$ com a taxa de cisalhamento para alguns fluidos reais. É possível notar na Figura que a variação da viscosidade não-newtoniana pode ser dividida em duas regiões principais, uma horizontal e uma retilínea decrescente sendo essa segunda chamada de região da Lei de Potência.

\begin{figure}[ht]
    \centering
    \includegraphics[width=0.5\linewidth]{imagens/bird-viscosidade-cap4.png}
    \caption{Viscosidade não-newtoniana de três polímeros fundidos a partir da equação de viscosidade de Carreau (a = 2). Dados experimentais: Poliestireno (453 K), Polietileno de alta densidade (443 K) e Phenoxy-A (485 K), com parâmetros $\eta_0$, $\eta_\infty$, $\lambda$ e $n$ específicos para cada polímero.
}
    \label{nn-bird}
    {\footnotesize Fonte: \citet{Bird1987}}
\end{figure}

Para a maioria dos fluidos e escoamentos, utilizando os tipos mais simples de viscosímetros, é impossível detectar a região horizontal do comportamento de $\eta$, de forma que a região da Lei de Potência, para esses casos, é a mais importante e representativa do escoamento \citep{Bird1987}. 

O modelo da Lei de Potência é um modelo empírico que busca modelar a região de mesmo nome do comportamento da viscosidade não-newtoniana por meio de dois parâmetros experimentais, $m$ que é chamado de fator de condicionamento e $n$ que tem grande influência no tipo de comportamento não-newtoniano que será apresentado pelo fluido. Tais parâmetros buscam ajustar uma curva para $\eta$ variando de fluido para fluido, por meio de experimentação material.

Por ter poucos parâmetros de ajuste, a Lei de Potência se torna mais rápida e menos onerosa de se ajustar quando comparada com outras variações de modelos de fluidos generalizados. Por esse motivo, por ser capaz de representar a maioria dos problemas industriais estudados e por sua fácil implementação numérica computacional, esse modelo era um dos mais utilizados dentro de sua família de modelos para representar o comportamento não-newtoniano a nível acadêmico e industrial até poucos anos \citep{Bird1987}. Apesar disto, com a evolução da complexidade dos problemas estudados, outros métodos melhores surgiram para representar os escoamentos \citep{Vasconcellos2024}. O modelo em questão, com suas limitações, ainda é útil para validar os softwares para não-newtonianos antes de implementar modelos mais complexos.

O modelo matemático da Lei de potência é apresentado por \citet{Bird1987}, Eq. \ref{lei de potencia}:

\begin{equation}
    \eta\left(\dot{\gamma}\right)=m{\dot{\gamma}}^{n-1}
    \label{lei de potencia}
\end{equation}

Neste modelo o comportamento do fluido de acordo com n varia da seguinte forma:

\begin{itemize}
    \item 	$n\ <\ 1$ o fluido se comporta como pseudoplástico
    \item 	$n$\ =\ $1$\ o fluido se comporta como newtoniano
    \item	$n$\ >\ $1$\ o fluido se comporta como dilatante.
\end{itemize}

O fator $m$z como já mencionado, é o fator o chamado fator de influência do modelo.

Apesar do modelo da lei de potência ser o mais utilizado industrialmente, a literatura contém diversos outros modelos com suas vantagens e propriedades. Uma lista mais detalhada desses modelos é apresentada por \citet{Vasconcellos2024}.

\section{\texorpdfstring{Modelagem do Tensor $\dot{\gamma}$ em $\eta$}{Modelagem do Tensor de gamma em eta}}

Para manter a consistência da equação do balanço da quantidade de movimento linear, $\eta\left(\dot{\gamma}\right)$ que é utilizado nos modelos de fluidos generalizados, deve ser um escalar. Dessa forma $\dot{\gamma}$ não pode ser considerado em sua forma literal, um tensor com seis componentes, devendo se comportar também como escalar. Por este motivo, é preciso que se use um valor que seja representativo do tensor taxa de cisalhamento ao longo de todo o escoamento. A solução mais comum é o uso de invariantes.

Os invariantes são combinações dos termos de tensores, geralmente seguindo operações matemáticas como o traço ou o determinante do tensor, no qual o valor não varia com observador que as descreve, ou seja, é independente do sistema de coordenadas. Uma abordagem mais rigorosa desse critério pode ser encontrada em \citet{Aristeu2020}, \citet{Bird1987} e \cite{Vasconcellos2024}.

 Várias formas de invariantes podem ser encontrados na literatura, de forma que uma das formas mais comuns é a apresentada e adotada por \citet{Vasconcellos2024}, para uma determinada matriz $A$, tal que, Eq. \ref{matriz A genérica}:

\begin{equation}
    A\ =\ \left(\begin{matrix}S_{xx}&S_{xy}&S_{xz}\\S_{xy}&S_{yy}&S_{zy}\\S_{xz}&S_{zy}&S_{zz}\\\end{matrix}\right) 
    \label{matriz A genérica}
\end{equation}

Os invariantes podem ser escritos da seguinte forma, Eq. \ref{invariariante 1 aristeu} a \ref{invariariante 3 aristeu} \citep{Aristeu2020}:

\begin{align}
    &I_1\ =\ tr(A)=S_{xx}\ +\ S_{yy}+\ S_{zz}\label{invariariante 1 aristeu}\\
    &I_2\ =\ \frac{1}{2}(tr(A)^2-tr(A^2))=S_{xx}S_{yy} + S_{yy}S_{zz} + S_{zz}S_{xx} - S_{xy}^2 - S_{xz}^2 - S_{zy}^2 \label{segundo invariante Aristeu}\\
    &I_3\ =\ det(A)\label{invariariante 3 aristeu}
\end{align}

Nas equações acima, $det()$ simboliza a operação de determinante e $tr()$ o traço da matriz.

Outra forma de invariante é apresentado por \citet{Bird1987}, Eq. \ref{invariante 1 bird} a \ref{invariante 3 bird}:

\begin{align}
    &I_1\ =\ tr(A)=S_{xx}\ +\ S_{yy}+\ S_{zz}\label{invariante 1 bird}\\
    &I_2\ =\ tr(A^2)={S_{xx}}^2\ +\ {S_{yy}}^2+\ {S_{zz}}^2\ +\ {2S_{xy}}^2\ +{\ 2S_{xz}}^2\ +\ {{2S}_{zy}}^2\label{segundo invariante bird} \\
    &I_3\ =\ tr(A^3)\label{invariante 3 bird}
\end{align}

Dentre as opções de invariante a serem utilizados, tem-se que em alguns escoamentos o primeiro e o terceiro invariante, independente da abordagem, podem resultar em zero, impossibilitando sua escolha. Sendo assim, o segundo invariante é normalmente utilizado para representar o tensor taxa de deformação.

Devido à natureza quadrática do segundo invariante, diversos autores propõem que operações sejam feitas para que se possa utilizá-lo com consistência na modelagem das equações de balanço da quantidade de movimento linear. Uma das mais utilizada é apresentada por \citet{Vasconcellos2024}, Eq. \ref{gama-nu}:

\begin{equation}
    \dot{\gamma}\ =\ \sqrt{\frac{1}{2}I_2}
    \label{gama-nu}
\end{equation}

Devido ao uso da raiz na modelagem do tensor $\dot{\gamma}$, é razoável que se evite a utilização de valores negativos para o segundo invariante, por essa razão o modelo de invariante apresentado por \citet{Bird1987} foi utilizado como referência neste trabalho, com excessão do escoamento de Poiseuille no qual foram feitas comparações entre os modelos

Uma maior variedade de abordagens do tensor $\dot{\gamma}$ pode ser encontrado no trabalho de \citet{Vasconcellos2024}.

Considerando a matriz bidimensional mostrada na Eq. \ref{eq:gama-bidimensional} o tensor em sua representação escalar, bidimensional, pode ser escrito da seguinte forma, Eq. \ref{gama-bidi-aberto}:

\begin{equation}
    \dot{\gamma}\ =\ \sqrt{2{\frac{\partial u}{\partial x}}^2+2{\frac{\partial v}{\partial y}}^2\ +\left(\frac{\partial u}{\partial y}\ +\ \frac{\partial v}{\partial x}\right)^2\ }
    \label{gama-bidi-aberto}
\end{equation}

Dessa forma é possível manter $\eta\left(\dot{\gamma}\right)$ como um escalar, preservar a compatibilidade na equação de balanço e simplificando a abordagem numérica que será descrita posteriormente.

\section{Aproximação de Oberbeck-Boussinesq para Escoamentos Não-Isotérmicos}

Em escoamentos onde há a variação de temperatura ao longo do domínio físico, todas as propriedades dependentes do fluido, como viscosidade, massa específica e difusividade térmica são diferentes para cada valor de temperatura, de forma que para a total representação da realidade seria necessário considerar essa variação na modelagem matemática dos escoamentos. 

Estudos experimentais e numéricos, porém, demonstraram que para pequenas variações de temperatura e para escoamentos incompressíveis, essa variação nas propriedades escalares descritas não é de grande relevância para a qualidade dos resultados, sendo seu custo computacional e prático não compatível com sua contribuição \citet{borgnakke2018} e \citet{Bird1987}.

Durante estudos sobre convecção natural em escoamentos, \citet{boussinesq1903,oberbeck1888} \textit{apud} \citet{Santos2022})concluíram que o campo de temperatura exerce influência no balanço da quantidade de movimento linear, através do desequilíbrio forças peso e empuxo ocasionadas pela variação da massa específica do fluido. Dessa forma se utiliza a consideração de que a variação da massa específica é desprezível em todos os termos com exceção do termo gravitacional. Está é a chamada aproximação de Boussinesq-Oberbeck \citep{Santos2022}.

Utilizando esta aproximação a equação de balanço do momentum linear fica da seguinte forma, Eq. \ref{balanço com energia termica} e \ref{delta T}:

\begin{align}
    &\frac{\partial\vec{v}}{\partial t}+\left(\vec{v}\cdot\vec{\nabla}\right)\vec{v}=-\frac{1}{\rho}\vec{\nabla}p+\frac{1}{\rho}\vec{\nabla}\cdot\overline{\overline{\tau}}+\beta\Delta T\vec{g} \label{balanço com energia termica}\\
   &\Delta T = (T_{ref} - T) \label{delta T}
\end{align}

Na qual o termo $\beta$ é o coeficiente de expansão volumétrica, uma propriedade física inerente do fluido de interesse e tem como função correlacionar a variação de massa específica com a variação de temperatura do fluido. Um melhor detalhamento matemático dessas propriedades pode ser encontrado em \cite{borgnakke2018}.

Em escoamentos com fluido não-newtonianos, a variação dos parâmetros que descrevem $\eta\left(\dot{\gamma}\right)$, como $m$ e $n$ na lei de potência, não é desprezível, mesmo utilizando a considerações de Boussinesq-Oberbeck e precisam de modelagem adicional para garantir a fidelidade dos resultados simulados. Não foram feitas simulações térmicas de fluidos não-newtonianos no presente trabalho. \citet{Bird1987} apresenta exemplos de escoamentos não-isotérmicos utilizando fluidos não-newtonianos com a Lei de Potência Carreau-Yasuda, principalmente para escoamentos em canais anulares.

\section{Números Adimensionais}

\subsection{Reynolds}

É geralmente aceito como parâmetro mais importante da grande maioria dos escoamentos, sua formulação foi dada primeiramente por Osborne Reynolds em 1883 \citep{White2011} Este parâmetro pode ser interpretado como uma razão entre os efeitos advectivos e os efeitos difusivos de um escoamento. Sua modelagem é escrita como sendo, Eq. \ref{numero de Reynolds}:

\begin{equation}
    Re\ =\ \frac{\rho UD}{\mu}\ =\ \frac{UD}{\nu}
    \label{numero de Reynolds}
\end{equation}

Na qual $U$ e $D$ são, respectivamente, a velocidade característica e o tamanho característico do escoamento. Para as abordagens analíticas esses valores são definidos de acordo com o escoamento e há um relativo consenso na literatura para os escoamentos clássicos, de forma que geralmente é definido pelo autor quais critérios o mesmo utiliza para a escolha dos parâmetros. Na abordagem numérica, se pode definir o número de Reynolds local, no qual $U$ é a velocidade da célula computacional e $D$ o $dx$ ou $dy$ que são as dimensões da célula computacional \citep{maliska2023}.

O valor deste parâmetro é o principal, na maioria dos escoamentos, para se determinar o quão instável é o problema e qual a sua probabilidade de transicionar para a turbulência devido as instabilidades numéricas presentes no escoamento. 

Para fluidos não-newtonianos, outras formas de se escrever o número de Reynolds são apresentadas pela literatura, exemplos podem ser vistos em \citet{Madlener2009}. Para a lei de potência, uma abordagem comumente utilizada é a apresentada abaixo, Eq. \ref{reynolds não newtoniano}.

\begin{equation}
    Re\ =\ \frac{\rho U^{2-n}L^n}{m}
    \label{reynolds não newtoniano}
\end{equation}

Nos quais $m$ e $n$ são os parâmetros a serem ajustados utilizando a Lei de Potência.

Tal modelo adaptado, permite prever o comportamento dos fluidos não-newtonianos e a comparar resultados entre diferentes trabalhos, de forma que estes se correlacionam pelos números adimensionais.


\subsection{Prandtl}

O número de Prandtl é um importante número adimensional para escoamentos não isotérmicos. A definição física deste parâmetro é descrita como a razão da difusividade de momentum linear pela difusividade térmica. Outra forma de se interpretar, segundo \citet{Santos2022} é a efetividade do transporte por difusão de quantidade de movimento linear pela térmica. O número de Prandtl é definido como, Eq. \ref{prandt}:

\begin{equation}
    Pr\ =\ \frac{\nu}{\alpha}
    \label{prandt}
\end{equation}

\subsection{Grashof}

O número de Grashof é utilizado em escoamentos não isotérmicos e pode ser descrito como a razão entre os efeitos de empuxo pelos efeitos viscosos \citep{White2011}. O número de Grashof é definido como sendo, Eq. \ref{grashof}:

\begin{equation}
    Gr\ =\frac{g\beta\Delta{TL}^3}{\nu^2}\ 
    \label{grashof}
\end{equation}

\subsection{Rayleigh}

Em escoamentos de convecção natural, o número de Rayleigh se torna tão importante quanto o número de Reynolds para definir qual será o comportamento do escoamento. A definição inicial deste número foi feita inicialmente por Lord Rayleigh (1842- 1919) \citep{White2011} em seus estudos em convecção natural. A definição física deste número pode ser dada também como uma relação entre os efeitos advectivos e os efeitos difusivos do escoamento, outra interpretação são os efeitos de empuxo sobre os efeitos viscosos \citep{White2011}. A formulação matemática deste número é dada por, Eq. \ref{rayleigh}:

\begin{equation}
    Ra=\frac{g\beta\Delta{TL}^3}{\nu\alpha}
    \label{rayleigh}
\end{equation}

Na qual $\Delta T$ é uma diferença de temperatura caraterística do escoamento, que deve ser definida por cada autor de forma a manter a repetibilidade dos experimentos materiais e computacionais, é a mesma definida pela Eq. \ref{delta T}. $L$ é novamente uma dimensão característica arbitrária do escoamento.

Outra forma de definir o número de Rayleigh comumente utilizada na literatura é com a junção de dois outros números adimensionais, o número de Prandtl e o número de Grashof \citep{Santos2022} da seguinte forma, Eq. \ref{ray alternativo}:

\begin{equation}
    Ra=Pr\ Gr
    \label{ray alternativo}
\end{equation}

Para garantir a repetibilidade dos resultados, é necessário informar o valor dos três parâmetros adimensionais utilizados em cada uma das simulações ou experimentos materiais.

\section{Formulação Geral}

Pode-se resumir a forma das equações utilizadas no desenvolvimento dos modelos numérico e computacionais como, Eq. \ref{continuidade-completa} a \ref{momentum-completa}:

\begin{align}
    &\vec{\mathrm{\nabla}}\cdot\left(\vec{v}\right)=0 \label{continuidade-completa}\\
    &\frac{\partial T}{\partial t}+\vec{v}\vec{\mathrm{\nabla}}\cdot T=\alpha\mathrm{\nabla}^2T \label{energia-completa}\\
    &\frac{\partial\vec{v}}{\partial t}+\left(\vec{v}\cdot\vec{\nabla}\right)\vec{v}=-\frac{1}{\rho}\vec{\nabla}p+\frac{1}{\rho}\vec{\nabla}\cdot\eta\left(\dot{\gamma}\right)\dot{\gamma}+\beta\Delta T\vec{g}\label{momentum-completa}
\end{align}

De forma que para fluidos newtonianos tem-se que $\eta\left(\dot{\gamma}\right)\ =\ \mu$. 


%====================================================================
% Metodologia.
%====================================================================
%====================================================================
% Metodologia: Escreva logo após o 
% \chapter{Metodologia} o texto de seu trabalho referente 
% aos métodos utilizados no que desenvolvimento de seu estudo. 
%====================================================================
\chapter{Modelo Matemático-Numérico}

Para a solução das equações diferenciais por modelos computacionais, se faz necessária a discretização utilizando modelos discretos. Este capítulo tem como objetivo detalhar o tratamento numérico utilizado no desenvolvimento do presente trabalho, bem como exemplificar as dificuldades encontradas. Também serão abordado os modelos utilizados para a resolução de sistemas lineares e suas limitações. 

Uma revisão da teoria de malhas e de métodos de discretização será apresentada para explicar a origem e a formulação dos métodos utilizados na formulação do trabalho.

\section{Malha Deslocada}

Quando se lida com malhas computacionais para fluidos normalmente se tem duas opções de malha a serem utilizadas, malhas deslocadas (Figura \ref{malha deslocada}) e malhas colocadas (Figura \ref{malha colocada}). Tais métodos possuem diferentes aplicações e teorias computacionais desenvolvidas especificamente para cada um, de forma que existem softwares abertos e comerciais com ambos os modelos \cite{Malalasekera2007}.

\begin{figure}[ht]
    \centering
    \includegraphics[width=0.65\textwidth]{imagens/modelo_de_malha.png}
    \caption{Modelo de malha deslocada}
    \label{malha deslocada}
    {\footnotesize Fonte: Do próprio autor}
\end{figure}

Neste método de malha as informações escalares, como a temperatura e a pressão, são guardados no centro das células enquanto as componentes vetoriais são armazenadas nas fronteiras laterais de cada célula. A componente $u$ do vetor $\vec{v}$, de forma que $\vec{v}\ =\ (u,v)$, é armazenado nas fronteiras verticais das células (setas horizontais), enquanto a componente $v$ é armazenada nas fronteiras horizontais (setas verticais), como mostrado na Figura \ref{malha deslocada} \citep{Fortuna2000}. Na prática, tem-se múltiplas malhas computacionais, uma para cada uma das componentes vetoriais e uma para as componentes escalares, sendo que cada malha possui suas próprias fronteiras e domínio.

É descrito por \citet{Malalasekera2007}, que esta abordagem corrige algumas inconsistências da malha colocada quanto a descrição do campo de pressão, que é crucial para a convergência numérica do escoamento. Por este motivo a malha deslocada é a mais utilizada em softwares que utilizam malhas estruturadas, na qual as fronteiras e o padrão de malha é bem definido e normalmente cartesiano, apara a resolução dos escoamentos.

\begin{figure}[ht]
    \centering
    \includegraphics[width=0.65\textwidth]{imagens/modelo_de_malha_colocada_ingrid.png}
    \caption{Modelo de malha colocada. }
    \label{malha colocada}
    {\footnotesize Fonte: Do próprio autor}
\end{figure}

Embora a malha colocada apresente algumas inconsistências no cálculo do gradiente de pressão, ela é amplamente utilizada em softwares que adotam malhas não estruturadas. Nesses casos, não existe um padrão rígido para a geração da malha, o que dificulta a aplicação de múltiplas malhas para diferentes variáveis, como ocorre na abordagem deslocada. Malhas não estruturadas são frequentemente empregadas em softwares comerciais e de código aberto, como o Ansys e o OpenFOAM, devido à sua flexibilidade para lidar com geometrias complexas e industriais. Existe uma vasta gama de teorias específicas que buscam mitigar as inconsistências dessa abordagem \citep{Malalasekera2007}.

Apesar das limitações das malhas estruturadas em representar geometrias complexas, frente as malhas não estruturadas, metodologias como a fronteira imersa oferecem uma solução prática para representar geometrias espaciais complexas, tornando-as viáveis para uso em contextos industriais. Essa abordagem tem ganhado relevância, especialmente em áreas como a engenharia aeronáutica e automotiva no Brasil, onde a demanda por soluções eficientes é crescente.

\section{Volumes Finitos}

Dentre as possíveis metodologias para a discretização de equações diferenciais parciais, o mais utilizado para as equações que represemtam os escoamentos, é o método dos volumes finitos, por garantir o correto balanço das informações físicas, como a conservação da massa global dos sistemas \citet{Malalasekera2007} e \citet{maliska2023}.

O método dos volumes finitos divide o domínio computacional em pequenos volumes de controle, chamados células computacionais, e faz a média das informações físicas ao longo desses volumes, com a informação física sendo guardada no centro de cada um deles. Uma abordagem matemática mais detalhada desta metodologia pode ser encontrada em \citet{Malalasekera2007} e \citet{maliska2023}, a partir da discretização das equações utilizadas será possível obter um melhor entendimento da aplicação do método.

Seguindo esta abordagem, utilizando volumes finitos, uma estratégia muito utilizada para a implementação das condições de contorno é a das células fantasmas. Na Figura \ref{modelo de malha} é mostrado um domínio computacional com as células fantasmas em questão. A estratégia é utilizar células que não pertencem ao domínio fluido para que as condições de contorno sejam impostas de forma indireta.

\begin{figure}[ht]
    \centering
    \includegraphics[width=0.7\linewidth]{imagens/malha-modelo-fantasma.png}
    \caption{Domínio físico com células fantasmas para condições de contorno.}
    \label{modelo de malha}
    {\footnotesize Fonte: Do próprio autor}
\end{figure}

As condições são impostas da seguinte forma para Dirichlet, Eq. \ref{direchlet}:

\begin{equation}
    P_E=2\ast P-P_P
    \label{direchlet}
\end{equation}

E para Neumann, Eq. \ref{neumann}:

\begin{equation}
    P_E=P_P
    \label{neumann}
\end{equation}

Isso é feito para as outras células fantasmas, não foram considerados condições de terceira espécie no presente trabalho.

As direções definidas como positiva são as indicadas pela Figura \ref{malha deslocada}. As direções são definidas com a letra $i$ para a direção $x$ e a letra $j$ para a direção $y$, de forma que, Eq. \ref{definição de ij inicio} a \ref{definição de ij fim}:

\begin{align}
    &P_P=P_{ij}\label{definição de ij inicio}\\
    &P_E=P_{i-1j}\\
    &P_D=P_{i+1j}\\
    &P_S=P_{ij+1}\\
    &P_I=P_{ij+1}\label{definição de ij fim}
\end{align}

Na qual $P_S$ é a célula acima da de interesse (superior) e $P_I$ a célula abaixo. Será considerado $n$  sobrescrito$ (p_{ij}^n)$ para o passo de tempo anterior e $n+1$ para o passo de tempo atual. 

Para o presente trabalho, a menos que se espeficique o contrario, as integrais triplas consideradas são feitas no domínio computacional da célula, sendo assim a seguinte notação se torna equivalente para este trabalho: 

\begin{equation}
    \int_{t}^{t + \Delta t} \int_{y_{j - \frac{1}{2} \Delta y}}^{y_{j + \frac{1}{2} \Delta y}} \int_{x_{i - \frac{1}{2} \Delta x}}^{x_{i + \frac{1}{2} \Delta x}} \, dx \, dy \, dt = \int \int \int \, dx \, dy \, dt \nonumber
\end{equation}

\section{Discretização da Equação da Energia}

Retomando a equação \ref{energia-completa} como ponto de partida temos: 

\begin{equation}
    \frac{\partial T}{\partial t}+\vec{v}\mathrm{\nabla}\cdot T=\alpha\mathrm{\nabla}^2T \nonumber
\end{equation}

Aplicando os operadores e expandindo a equação, considerando o modelo bidimensional, Eq. \ref{energia-aberta}:

\begin{equation}
    \frac{\partial T}{\partial t}+\left(u\frac{\partial T}{\partial x}\ +\ v\frac{\partial T}{\partial y}\right)=\alpha\left(\frac{\partial^2T}{\partial x^2}\ +\ \frac{\partial^2T}{\partial x^2}\right) 
    \label{energia-aberta}
\end{equation}

Para realizar a discretização, se aplica a integral em cada um dos termos com relação às variáveis $x$, $y$ e $t$. começando pelo termo temporal temos, Eq. \ref{energia-temporal}:

\begin{equation}
    \int_{t}^{t+\Delta t}{\iint\frac{\partial T}{\partial t}dxdydt}=\left(T_P^{n+1}-T_P^n\right)\mathrm{\Delta x\Delta y}
    \label{energia-temporal}
\end{equation}

O passo de tempo no qual será considerado os demais termos é arbitrário e diferentes abordagens podem ser encontradas na literatura \citep{Fortuna2000}. Para a equação de balanço de energia térmica, foi utilizado o método totalmente implícito, tomando todos os termos no tempo atual. O método utilizado para o termo temporal é também chamado de método de Euler.

Para a discretização do termo advectivos temos, Eq. \ref{advec-energia-integral}:

\begin{equation}
    \iiint{\left(u\frac{\partial T}{\partial x}\ +\ v\frac{\partial T}{\partial y}\right)dydxdt\ =\ }u\left(T_d-T_e\right)\mathrm{\Delta y}\ \mathrm{\Delta t}\ +\ v\left(T_i-T_s\right)\mathrm{\Delta x}\ \mathrm{\Delta t}           
    \label{advec-energia-integral}
\end{equation}

Foi utilizado o método das diferenças centradas para calcular o valor das propriedades escalares nas paredes, de forma que Eq. \ref{energia-centradas-inicio} a \ref{energia-centradas-fim}:

\begin{align}
    &T_e=\frac{T_P+T_E}{2}\label{energia-centradas-inicio}\\
    &T_d=\frac{T_D+T_P}{2}\\
    &T_i=\frac{T_I+T_P}{2}\\
    &T_S=\frac{T_P+T_S}{2}\label{energia-centradas-fim}
\end{align}

Substituindo as equações \ref{energia-centradas-inicio} a \ref{energia-centradas-fim} na equação \ref{advec-energia-integral} e aplicando as devidas simplificações temos, Eq. \ref{advec-desenvolvida}:

\begin{equation}
    \left(u\frac{\partial T}{\partial x}\ +\ v\frac{\partial T}{\partial y}\right)\ =\ u\left(\frac{T_D-T_E}{2}\right)\mathrm{\Delta y}\ \mathrm{\Delta t}\ +\ v\left(\frac{T_I-T_S}{2}\right)\mathrm{\Delta x}\ \mathrm{\Delta t}
    \label{advec-desenvolvida}
\end{equation}

Apesar do método das diferenças centradas não levar em conta a orientação do escoamento \citep{Malalasekera2007}, seus resultados foram satisfatórios para os problemas apresentados.

Para o termo difusivo temos, Eq. \ref{termo-difusivo-aberto}: 

\begin{equation}
    \alpha\left(\frac{\partial^2T}{\partial x^2}\ +\ \frac{\partial^2T}{\partial x^2}\right)\ =\ \alpha\left(\left(\ {\frac{\partial T}{\partial x}}_d-\ {\frac{\partial T}{\partial x}}_e\right)\mathrm{\Delta y}\ +\ \left(\ {\frac{\partial T}{\partial y}}_i-\ {\frac{\partial T}{\partial y}}_s\right)\mathrm{\Delta x}\right)\mathrm{\Delta t}              \label{termo-difusivo-aberto}
\end{equation}

Considerando novamente o esquema de diferenças centradas temos, Eq. \ref{cds-derivadas-inicio} a \ref{cds-derivadas-fim}:

\begin{align}
    &{\frac{\partial T}{\partial x}}_e\ =\ \frac{T_P-T_E}{\mathrm{\Delta x}}\label{cds-derivadas-inicio}\\
    &{\frac{\partial T}{\partial x}}_d\ =\ \frac{T_D-T_P}{\mathrm{\Delta x}}\\
    &{\frac{\partial T}{\partial y}}_i\ =\ \frac{T_I-T_P}{\mathrm{\Delta y}}\\
    &{\frac{\partial T}{\partial y}}_s\ =\ \frac{T_P-T_S}{\mathrm{\Delta y}}\label{cds-derivadas-fim}\\
\end{align}

Substituindo as equações \ref{cds-derivadas-inicio} a \ref{cds-derivadas-fim} na equação \ref{termo-difusivo-aberto} temos, Eq. \ref{termo-difusivo-completo}:

\begin{equation}
    \alpha\left(\frac{\partial^2T}{\partial x^2}\ +\ \frac{\partial^2T}{\partial x^2}\right)\ =\alpha\left(\left(\ \frac{T_E\ -\ 2T_P\ +T_D\ }{\mathrm{\Delta x}}\right)\mathrm{\Delta y}\ +\ \left(\frac{T_I\ -\ 2T_P\ +T_S\ }{\mathrm{\Delta y}}\right)\mathrm{\Delta x}\right)\mathrm{\Delta t}\ 
    \label{termo-difusivo-completo}
\end{equation}

Agrupando as equações \ref{energia-temporal}, \ref{advec-desenvolvida} e \ref{termo-difusivo-completo} temos, Eq. \ref{energia-completa-discretizada}:

\begin{align}
    &\left(T_p^{n+1}-T_p^n\right)\mathrm{\Delta x\Delta y} + u\left(\frac{T_D - T_E}{2}\right)\mathrm{\Delta y}\ \mathrm{\Delta t} \nonumber \\
    &\quad + v\left(\frac{T_I - T_S}{2}\right)\mathrm{\Delta x}\ \mathrm{\Delta t} \nonumber \\
    &= \alpha\left(\frac{T_E - 2T_P + T_D}{\mathrm{\Delta x}}\mathrm{\Delta y} + \frac{T_I - 2T_P + T_S}{\mathrm{\Delta y}}\mathrm{\Delta x}\right)\mathrm{\Delta t}  
    \label{energia-completa-discretizada}
\end{align}

Substituindo pela nomenclatura adotada no tópico anterior e agrupando termos similares temos, Eq. \ref{energia-discretizada-ij}:

\begin{align}
    &T_{ij}^{n+1}\left(1 + \frac{2\alpha\Delta t}{\mathrm{\Delta}x^2} + \frac{2\alpha\Delta t}{\mathrm{\Delta}y^2} \right) \nonumber \\
    &\quad - T_{i+1j}^{n+1}\frac{\alpha\Delta t}{\mathrm{\Delta}x^2} - T_{i-1j}^{n+1}\frac{\alpha\Delta t}{\mathrm{\Delta}x^2} \nonumber \\
    &\quad - T_{ij+1}^{n+1}\frac{\alpha\Delta t}{\mathrm{\Delta}y^2} - T_{ij-1}^{n+1}\frac{\alpha\Delta t}{\mathrm{\Delta}y^2} \nonumber \\
    &\quad + u_{ij}^n\left(\frac{T_{i+1j}^{n+1} - T_{i-1j}^{n+1}}{2\mathrm{\Delta x}}\right) \mathrm{\Delta t} \nonumber \\
    &\quad + v_{ij}^n\left(\frac{T_{ij+1}^{n+1} - T_{ij-1}^{n+1}}{2\mathrm{\Delta y}}\right)\mathrm{\Delta t} = T_{ij}^n
    \label{energia-discretizada-ij}
\end{align}

Com essa formulação é possível montar um sistema de equações do tipo:

\begin{equation}
    Ax\ =\ b\nonumber
\end{equation}

Que é compatível com os métodos de solução de sistemas lineares.

\section{Acoplamento Pressão-Velocidade}

A solução da equação de balanço da quantidade de movimento, acaba sendo dificultada pela quantidade de incógnitas a serem modeladas em relação a quantidade de equações disponíveis \citep{Aristeu2020}.

De fato, retomando as equações \ref{continuidade-completa} e \ref{momentum-completa} temos:

\begin{align}
    &\vec{\mathrm{\nabla}}\cdot\left(\vec{v}\right)=0\nonumber\\
    &\frac{\partial\vec{v}}{\partial t}+\left(\vec{v}\cdot\vec{\nabla}\right)\vec{v}=-\vec{\nabla}p+\vec{\nabla}\cdot\eta\left(\dot{\gamma}\right)\dot{\gamma}+\beta\Delta T\vec{g}\nonumber
\end{align}

Nas equações de balanço tem-se três incógnitas a serem modeladas: u, v e p de forma que tem-se menos equações variáveis. Se faz necessário recorrer a equação da continuidade, retomando a equação \ref{continuidade-completa} e abrindo temos, Eq. \ref{continuidade - aberta}:

\begin{equation}
    \frac{\partial u}{\partial x}\ +\ \frac{\partial v}{\partial y}=0
    \label{continuidade - aberta}
\end{equation}

É possível perceber que a equação da continuidade não contém a componente de pressão, de forma que seria necessário manipulações matemáticas para utilizar a equação em conjunto com as demais. 

Metodologias como \textit{Semi-Implicit Method for Pressure-Linked Equations (SIMPLE)}, \textit{Pressure-Implicit with Splitting of Operators (PISO)} e o método da projeção de Chorin são comumente utilizadas na literatura para lidar com este problema de fechamento \citep{Malalasekera2007}. No desenvolvimento do software foi adotado uma variação do método de Chorin, também chamado de método do passo fracionado (\textit{Fractional Step em inglês}) \citep{Santos2022}.

A partir do campo inicial de pressão, do passo de tempo anterior, chamado de $p_0$ é determinado um campo de velocidade aproximado $\vec{V^\ast}$. De forma que, Eq. \ref{continuidade-correção} e \ref{continuidade-correção-temporal}:

 \begin{align}
     &\frac{\partial{\vec{v}}^\ast}{\partial t}+\left(\vec{v}\cdot\vec{\nabla}\right)\vec{v}=-\frac{1}{\rho}\vec{\nabla}p_0+\frac{1}{\rho}\vec{\nabla}\cdot\eta\left(\dot{\gamma}\right)\dot{\gamma}+\beta\Delta T\vec{g}\label{continuidade-correção}\\
     &\frac{v^\ast-v^n}{\Delta t}+\left(\vec{v}\cdot\vec{\nabla}\right)\vec{v}=-\frac{1}{\rho}\vec{\nabla}p_0+\frac{1}{\rho}\vec{\nabla}\cdot\eta\left(\dot{\gamma}\right)\dot{\gamma}+\beta\Delta T\vec{g}\label{continuidade-correção-temporal}
 \end{align}

 Foi considerado a integral em $t$ utilizando o método de Euler, para simplificar a descrição do modelo.

 Enquanto a real velocidade seria, Eq. \ref{velocidade-real} e \ref{velocidade-real-temporal}:

 \begin{align}
     &\frac{\partial\vec{v}}{\partial t}+\frac{1}{\rho}\left(\vec{v}\cdot\vec{\nabla}\right)\vec{v}=-\frac{1}{\rho}\vec{\mathrm{\nabla}}p+\frac{1}{\rho}\vec{\nabla}\cdot\eta\left(\dot{\gamma}\right)\dot{\gamma}+\beta\Delta T\vec{g}\label{velocidade-real}\\
    &\frac{v^{n+1}-v^n}{\Delta t}+\left(\vec{v}\cdot\vec{\nabla}\right)\vec{v}=-\frac{1}{\rho}\vec{\mathrm{\nabla}}p+\frac{1}{\rho}\vec{\nabla}\cdot\eta\left(\dot{\gamma}\right)\dot{\gamma}+\beta\Delta T\vec{g}\label{velocidade-real-temporal}
 \end{align}

 Subtraindo a equação \ref{continuidade-correção-temporal} da equação \ref{velocidade-real-temporal} temos, Eq. \ref{correção de  velocidade -nd}:

 \begin{equation}
     \frac{\vec{v}\ -\ {\vec{v}}^\ast}{\Delta t}=-\frac{1}{\rho}\vec{\mathrm{\nabla}}\left(p-p_0\right)
     \label{correção de  velocidade -nd}
 \end{equation}

 É possível definir um fator de correção para levar o campo de pressão do tempo anterior para o tempo atual da seguinte forma, Eq. \ref{p=p-po}:

 \begin{equation}
     p=p_0+\ p^\prime
     \label{p=p-po}
 \end{equation}

 De forma que podemos reorganizar os termos em, Eq. \ref{pl=p-p0}:

 \begin{equation}
     p^\prime\ =\ p-p_0	
     \label{pl=p-p0}
 \end{equation}

 Assim, substituindo a equação \ref{pl=p-p0} na equação \ref{correção de  velocidade -nd}, temos, Eq. \ref{correção-correta-nd}:

 \begin{equation}
     \frac{\vec{v}\ -\ {\vec{v}}^\ast}{\Delta t}=-\frac{1}{\rho}\vec{\mathrm{\nabla}}p^\prime\ 
     \label{correção-correta-nd}
 \end{equation}

 Aplicando o divergente em ambos os lados da equação, Eq. \ref{momentum-desenvolvimento}:

\begin{align}
     &\vec{\mathrm{\nabla}}\cdot\ \frac{\vec{v}\ -\ {\vec{v}}^\ast}{\Delta t}=-\frac{1}{\rho}\vec{\mathrm{\nabla}}\cdot\vec{\mathrm{\nabla}}p^\prime\ \nonumber\\
     &\frac{\vec{\mathrm{\nabla}}\cdot\vec{v}\ -\ \vec{\mathrm{\nabla}}\cdot{\vec{v}}^\ast}{\Delta t}=-\frac{1}{\rho}\vec{\mathrm{\nabla}}^2p'\label{momentum-desenvolvimento}
\end{align}

 Utilizando a equação da continuidade temos, Eq. \ref{correção-pressão-completo}:

\begin{align}
    &\vec{\mathrm{\nabla}} \cdot \vec{v} = 0 \nonumber \\
    &\vec{\mathrm{\nabla}} \cdot \vec{v}^* = \frac{\Delta t}{\rho} \vec{\mathrm{\nabla}}^2 p' \label{correção-pressão-completo}
\end{align}

 Manipulando a equação \ref{momentum-desenvolvimento} temos, Eq. \ref{v=vl-p}: 

 \begin{equation}
     \vec{v}=\ {\vec{v}}^\ast-\frac{\Delta t}{\rho}\vec{\mathrm{\nabla}}p^\prime
     \label{v=vl-p}
 \end{equation}

 Que para o presente trabalho pode ser escrito como, Eq. \ref{correção-ul} e \ref{correção-vl}:

\begin{align}
    &u=\ u^\ast-\frac{\mathrm{\Delta t}}{\rho}\frac{\partial p^\prime}{\partial x}\label{correção-ul}\\
   &v=\ v^\ast-\frac{\mathrm{\Delta t}}{\rho}\frac{\partial p^\prime}{\partial x}\label{correção-vl}
\end{align}

O que fecha a modelagem do acoplamento pressão-velocidade utilizando o método da projeção de Chorin.

Sintetizando as etapas para a resolução dos campos de pressão e velocidade temos:

\begin{itemize}
    \item Cálculo do campo de velocidade aproximado com a pressão do passo de tempo anterior usando a equação \ref{continuidade-correção-temporal}:
    \begin{equation*}
        \frac{v^\ast-v^n}{\Delta t}+\left(\vec{v}\cdot\vec{\nabla}\right)\vec{v}=-\frac{1}{\rho}\vec{\nabla}p_0+\frac{1}{\rho}\vec{\nabla}\cdot\eta\left(\dot{\gamma}\right)\dot{\gamma}+\beta\Delta T\vec{g}
    \end{equation*}
    \item Cálculo da correção do campo de pressão, Eq. \ref{correção-pressão-completo}:
    \begin{equation*}
    \vec{\mathrm{\nabla}} \cdot \vec{v}^* = \frac{\Delta t}{\rho} \vec{\mathrm{\nabla}}^2 p' \end{equation*}
    \item Correção do campo de velocidade para o passo de tempo atual, Eq. \ref{v=vl-p}:
    \begin{equation*}
        \vec{v}=\ {\vec{v}}^\ast-\frac{\Delta t}{\rho}\vec{\mathrm{\nabla}}p^\prime
    \end{equation*}
    
    
    \item Correção do campo de pressão para o passo de tempo posterior, Eq. \ref{p=p-po}:
    \begin{equation*}
        p=p_0+\ p^\prime
    \end{equation*} 
\end{itemize}

\section{Discretização das Equações do Acoplamento Pressão-Velocidade}

Expandindo a equação \ref{momentum-completa} em suas componentes temos, Eq. \ref{momentum-u} e \ref{momentum-v}

\begin{align}
    &\frac{\partial u}{\partial t}+\left(u\frac{\partial u}{\partial x}\ +\ v\frac{\partial v}{\partial y}\right)=-\frac{1}{\rho}\frac{\partial p}{\partial x}+\frac{1}{\rho}\vec{\mathrm{\nabla}}\cdot\eta\left(\dot{\gamma}\right)\dot{\gamma} \label{momentum-u}\\
    &\frac{\partial v}{\partial t}+\left(u\frac{\partial v}{\partial x}\ +\ v\frac{\partial v}{\partial y}\right)=-\frac{1}{\rho}\frac{\partial p}{\partial y}+\frac{1}{\rho}\vec{\mathrm{\nabla}}\cdot\eta\left(\dot{\gamma}\right)\dot{\gamma}+\beta\Delta Tg\label{momentum-v}
\end{align}

Devido à complexidade do termo difusivo é pertinente realizar o detalhamento deste termo separadamente. Assim temos, Eq. \ref{termo difusivo - p1}, \ref{termo difusivo - p2}:

\begin{align}
    &\vec{\mathrm{\nabla}}\cdot\eta\left(\dot{\gamma}\right)\dot{\gamma}\ =\vec{\mathrm{\nabla}}\ \cdot\left(\begin{matrix}\eta\left(\dot{\gamma}\right)2\frac{\partial u}{\partial x}&\eta\left(\dot{\gamma}\right)\left(\frac{\partial u}{\partial y}\ +\ \frac{\partial v}{\partial x}\right)\\\eta\left(\dot{\gamma}\right)\left(\frac{\partial u}{\partial y}\ +\ \frac{\partial v}{\partial x}\right)&\eta\left(\dot{\gamma}\right)2\frac{\partial v}{\partial y}\\\end{matrix}\right)\label{termo difusivo - p1}\\
    &\vec{\mathrm{\nabla}}\ \cdot\left(\begin{matrix}\eta\left(\dot{\gamma}\right)2\frac{\partial u}{\partial x}&\eta\left(\dot{\gamma}\right)\left(\frac{\partial u}{\partial y}\ +\ \frac{\partial v}{\partial x}\right)\\\eta\left(\dot{\gamma}\right)\left(\frac{\partial u}{\partial y}\ +\ \frac{\partial v}{\partial x}\right)&\eta\left(\dot{\gamma}\right)2\frac{\partial v}{\partial y}\\\end{matrix}\right)\ \nonumber\\
    &=\ \left(\begin{matrix}\frac{\partial}{\partial x}\left(\eta\left(\dot{\gamma}\right)2\frac{\partial u}{\partial x}\right)\ +\frac{\partial}{\partial y}\left(\eta\left(\dot{\gamma}\right)\left(\frac{\partial u}{\partial y}\ +\ \frac{\partial v}{\partial x}\right)\right)\ \\\frac{\partial}{\partial x}\left(\eta\left(\dot{\gamma}\right)\left(\frac{\partial u}{\partial y}\ +\ \frac{\partial v}{\partial x}\right)\right)\ +\ \frac{\partial}{\partial y}\left(\eta\left(\dot{\gamma}\right)2\frac{\partial v}{\partial y}\right)\\\end{matrix}\right)\ \label{termo difusivo - p2}
\end{align}

Resolvendo o divergente e substituindo nas equações \ref{momentum-u} e \ref{momentum-v} resulta em, Eq. \ref{momentum-cdifusivo-u} e \ref{momentum-cdifusivo-v}:

\begin{align}
    &\frac{\partial u}{\partial t}+\left(u\frac{\partial u}{\partial x}\ +\ v\frac{\partial v}{\partial y}\right)=-\frac{1}{\rho}\frac{\partial p}{\partial x}+\frac{1}{\rho}\frac{\partial}{\partial x}\left(\eta\left(\dot{\gamma}\right)2\frac{\partial u}{\partial x}\right)\ +\frac{1}{\rho}\frac{\partial}{\partial y}\left(\eta\left(\dot{\gamma}\right)\left(\frac{\partial u}{\partial y}\ +\ \frac{\partial v}{\partial x}\right)\right) \label{momentum-cdifusivo-u} \\
    &\frac{\partial v}{\partial t}+\left(u\frac{\partial v}{\partial x}\ +\ v\frac{\partial v}{\partial y}\right)=-\frac{1}{\rho}\frac{\partial p}{\partial y}+\frac{1}{\rho}\frac{\partial}{\partial x}\left(\eta\left(\dot{\gamma}\right)\left(\frac{\partial u}{\partial y}\ +\ \frac{\partial v}{\partial x}\right)\right)\ +\frac{1}{\rho}\ \frac{\partial}{\partial y}\left(\eta\left(\dot{\gamma}\right)2\frac{\partial v}{\partial y}\right)+\beta\Delta Tg\label{momentum-cdifusivo-v}
\end{align}

Tomando como referência o passo a passo adotado na discretização dos termos da equação da energia, temos:

A discretização do termo temporal, Eq. \ref{temporal-u} e \ref{temporal-v}:

\begin{align}
    &\iiint\frac{\partial u}{\partial t}dxdydt=\left(u_P^\ast-u_P^n\right)\ \Delta x\Delta y\label{temporal-u}\\
    &\iiint\frac{\partial v}{\partial t}dxdydt=\left(v_P^\ast-v_P^n\right)\ \Delta x\Delta y\label{temporal-v}
\end{align}

A discretização do termo advectivo utilizando diferenças centradas, Eq. \ref{discretização - advectivo -  u} e \ref{discretização - advectivo -  v}:

\begin{align}
    &u\iiint\frac{\partial u}{\partial x}dxdydt+v\iiint\frac{\partial u}{\partial y}dxdydt=u_p\left(\frac{u_D-u_E}{2}\right)\mathrm{\Delta y}\ \Delta t+\ v_p\left(\frac{u_I-u_S}{2}\right)\Delta x\ \Delta t\label{discretização - advectivo -  u} \\
    &u\iiint\frac{\partial v}{\partial x}dxdydt+v\iiint\frac{\partial v}{\partial y}dxdydt=u_p\left(\frac{v_D-v_E}{2}\right)\mathrm{\Delta y}\ \Delta t+\ v_p\left(\frac{v_I-v_S}{2}\right)\Delta x\ \Delta t  \label{discretização - advectivo -  v}
\end{align}

Para a discretização do termo de pressão temos, Eq. \ref{discretização - px} e \ref{discretização - py}:

\begin{align}
    &\frac{\partial p}{\partial x}=\ \Delta t\Delta y\ \left(p_D-p_E\right)\label{discretização - px}\\
    &\frac{\partial p}{\partial y}=\ \Delta t\Delta x\left(p_I-p_S\right)\label{discretização - py}
\end{align}

Sendo essa notação baseada na definição de domínio adotada no tópico anterior.

A discretização do termo difusivo pode ser feitas por partes, a primeira resulta em, Eq. \ref{difusivo - p1x} e \ref{difusivo - p1y}:

\begin{align}
    &\iiint{\frac{\partial}{\partial x}\left(\eta\left(\dot{\gamma}\right)2\frac{\partial u}{\partial x}\right)}dxdydt=\ 2\left(\left(\eta\left(\dot{\gamma}\right)\frac{\partial u}{\partial x}\right)_d-\ \left(\eta\left(\dot{\gamma}\right)\frac{\partial u}{\partial x}\right)_e\right)\Delta y\Delta t \nonumber \\
    \nonumber \\
    \nonumber \\
    &\iiint{\frac{\partial}{\partial y}\left(\eta\left(\dot{\gamma}\right)2\frac{\partial v}{\partial y}\right)}dxdydt=2\left(\left(\eta\left(\dot{\gamma}\right)\frac{\partial v}{\partial y}\right)_i-\ \left(\eta\left(\dot{\gamma}\right)\frac{\partial v}{\partial y}\right)_s\right)\Delta x\Delta t \nonumber \\
    \nonumber \\
    \nonumber \\
    &2\left(\left(\eta\left(\dot{\gamma}\right)\frac{\partial u}{\partial x}\right)_d-\ \left(\eta\left(\dot{\gamma}\right)\frac{\partial u}{\partial x}\right)_e\right)\Delta y\Delta t=2\left({\eta\left(\dot{\gamma}\right)}_d\frac{u_D-u_P\ }{\Delta x}-\ {\eta\left(\dot{\gamma}\right)}_e\frac{u_P-u_E\ }{\Delta x}\right)\Delta y\Delta t\ \label{difusivo - p1x} \\
    \nonumber \\
    \nonumber \\
    &2\left(\left(\eta\left(\dot{\gamma}\right)\frac{\partial v}{\partial y}\right)_i-\ \left(\eta\left(\dot{\gamma}\right)\frac{\partial v}{\partial y}\right)_s\right)\mathrm{\Delta x\Delta t}=2\left({\eta\left(\dot{\gamma}\right)}_i\frac{v_I-v_P\ }{\mathrm{\Delta y}}-\ {\eta\left(\dot{\gamma}\right)}_s\frac{v_P-v_S\ }{\mathrm{\Delta y}}\right)\mathrm{\Delta x\Delta t}\label{difusivo - p1y}
\end{align}

Para posterior uso nos métodos numéricos, é eficiente dividir o termo que resta da difusão em duas partes, Eq. \ref{difusivo - p2x} e \ref{difusivo - p2y}:

\begin{align}
    &\frac{\partial}{\partial y}\left(\eta\left(\dot{\gamma}\right)\left(\frac{\partial u}{\partial y}\ +\ \frac{\partial v}{\partial x}\right)\right)=\frac{\partial}{\partial y}\left(\eta\left(\dot{\gamma}\right)\frac{\partial u}{\partial y}\right)\ +\frac{\partial}{\partial y}\left(\eta\left(\dot{\gamma}\right)\ \frac{\partial v}{\partial x}\right)\nonumber \\
    &\frac{\partial}{\partial x}\left(\eta\left(\dot{\gamma}\right)\left(\frac{\partial u}{\partial y}\ +\ \frac{\partial v}{\partial x}\right)\right)=\frac{\partial}{\partial x}\left(\eta\left(\dot{\gamma}\right)\frac{\partial u}{\partial y}\right)\ +\frac{\partial}{\partial x}\left(\eta\left(\dot{\gamma}\right)\ \frac{\partial v}{\partial x}\right)\ \  \nonumber \\
    &\iiint{\frac{\partial}{\partial y}\left(\eta\left(\dot{\gamma}\right)\frac{\partial u}{\partial y}\right) \, dxdydt} 
    + \iiint{\frac{\partial}{\partial y}\left(\eta\left(\dot{\gamma}\right)\frac{\partial v}{\partial x}\right) \, dxdydt} \nonumber \\
    = & \left(\eta\left(\dot{\gamma}\right)_i \frac{u_I - u_P}{\Delta y} - \eta\left(\dot{\gamma}\right)_s \frac{u_P - u_S}{\Delta y}\right) \Delta x \Delta t \nonumber \\
    & + \left(\eta\left(\dot{\gamma}\right)_i \frac{\partial v}{\partial x}_i - \eta\left(\dot{\gamma}\right)_s \frac{\partial v}{\partial x}_s\right) \Delta x \Delta t
    \label{difusivo - p2x} \\
    & \iiint{\frac{\partial}{\partial x}\left(\eta\left(\dot{\gamma}\right)\frac{\partial u}{\partial y}\right) dxdydt} 
    + \iiint{\frac{\partial}{\partial x}\left(\eta\left(\dot{\gamma}\right)\frac{\partial v}{\partial x}\right) dxdydt} \nonumber \\
    = & \left(\eta\left(\dot{\gamma}\right)_d \frac{\partial u}{\partial y}_d - \eta\left(\dot{\gamma}\right)_e \frac{\partial u}{\partial y}_e\right) \Delta y \Delta t \nonumber \\
    & + \left(\eta\left(\dot{\gamma}\right)_d \frac{v_D - v_P}{\Delta x} - \eta\left(\dot{\gamma}\right)_e \frac{v_P - v_E}{\Delta x}\right) \Delta y \Delta t
    \label{difusivo - p2y}
\end{align}

Unindo as equações \ref{temporal-u}, \ref{discretização - advectivo -  u}, \ref{discretização - px}, \ref{difusivo - p1x} e \ref{difusivo - p2x} para a coordenada u e \ref{temporal-v}, \ref{discretização - advectivo -  v}, \ref{discretização - py}, \ref{difusivo - p1y} e \ref{difusivo - p2y} temos, Eq. \ref{continuidade-u-discretizada-completa} e \ref{continuidade-v-discretizada-completa}:

\begin{align}
    & \left(u_P^{\ast n+1} - u_P^n\right) \Delta x \Delta y 
    + u_P \left(\frac{u_D - u_E}{2}\right) \Delta y \Delta t 
    + v_P \left(\frac{u_I - u_S}{2}\right) \Delta x \Delta t \nonumber \\
    = & -\frac{\Delta t \Delta y}{\rho} \left(p_D - p_E\right) 
    + 2 \frac{1}{\rho} \left(\eta\left(\dot{\gamma}\right)_d \frac{u_D - u_P}{\Delta x} 
    - \eta\left(\dot{\gamma}\right)_e \frac{u_P - u_E}{\Delta x}\right) \Delta y \Delta t \nonumber \\
    & + \frac{1}{\rho} \left(\eta\left(\dot{\gamma}\right)_i \frac{u_I - u_P}{\Delta y} 
    - \eta\left(\dot{\gamma}\right)_s \frac{u_P - u_S}{\Delta y}\right) \Delta x \Delta t \nonumber \\
    & + \frac{1}{\rho} \left(\eta\left(\dot{\gamma}\right)_i \frac{\partial v}{\partial x}_i 
    - \eta\left(\dot{\gamma}\right)_s \frac{\partial v}{\partial x}_s\right) \Delta x \Delta t
    \label{continuidade-u-discretizada-completa} \\
    & \left(v_P^\ast - v_P^n\right) \Delta x \Delta y 
    + u_P \left(\frac{v_D - v_E}{2}\right) \Delta y \Delta t 
    + v_P \left(\frac{v_I - v_S}{2}\right) \Delta x \Delta t \nonumber \\
    = & -\frac{\Delta t \Delta x}{\rho} \left(p_I - p_S\right) 
    + 2 \frac{1}{\rho} \left(\eta\left(\dot{\gamma}\right)_i \frac{v_I - v_P}{\Delta y} 
    - \eta\left(\dot{\gamma}\right)_s \frac{v_P - v_S}{\Delta y}\right) \Delta x \Delta t \nonumber \\
    & + \frac{1}{\rho} \left(\eta\left(\dot{\gamma}\right)_d \frac{\partial u}{\partial y}_d 
    - \eta\left(\dot{\gamma}\right)_e \frac{\partial u}{\partial y}_e\right) \Delta y \Delta t \nonumber \\
    & + \frac{1}{\rho} \left(\eta\left(\dot{\gamma}\right)_d \frac{v_D - v_P}{\Delta x} 
    - \eta\left(\dot{\gamma}\right)_e \frac{v_P - v_E}{\Delta x}\right) \Delta y \Delta t 
    + \beta \Delta Tg
    \label{continuidade-v-discretizada-completa}
\end{align}

Considerando o termo advectivo no passo de tempo anterior, para evitar não linearidades no sistema de equações, e o termo difusivo, com exceção das derivadas cruzadas no tempo atual para aumentar a estabilidade do sistema. Para facilitar a escrita das equações será considerado que $v_P^\ast=\ v_{ij}^{\ast n+1}$, agrupando os termos das equações temos, Eq. \ref{discretizada-u-nomenclatura} e \ref{discretizada-v-nomenclatura}:

\begin{align}
    &u_{i,j}^{n+1} - a \frac{2 \Delta t}{\rho \Delta x^2} \left[ \eta(y_j) u_{i+1,j}^{n+1} + \eta(y_j) u_{i-1,j}^{n+1} \right] - \frac{\Delta t}{\rho \Delta y^2} \left[ \eta(y_{j+1}) u_{i,j+1}^{n+1} + \eta(y_{j-1}) u_{i,j-1}^{n+1} \right] = b_u\label{discretizada-u-nomenclatura} \\
    &v_{ij}^{\ast n+1}a_v-\frac{2\mathrm{\Delta t}}{\rho\Delta y^2}\left({\eta\left(\dot{\gamma}\right)}_iv_{ij+1}^{\ast n+1}+\ {\eta\left(\dot{\gamma}\right)}_sv_{ij-1}^{\ast n+1}\right)-\frac{\mathrm{\Delta t}}{\rho\Delta x^2}\left({\eta\left(\dot{\gamma}\right)}_dv_{i+1j}^{\ast n+1}+{\eta\left(\dot{\gamma}\right)}_ev_{i-1j}^{\ast n+1}\right)=b_v\label{discretizada-v-nomenclatura} \\
    & b_v = v_{ij}^n - \frac{\Delta t}{\rho} \frac{p_{ij}^n - p_{ij-1}^n}{\Delta y} 
    + \frac{\Delta t}{\rho \Delta x} \left(\eta\left(\dot{\gamma}\right)_d \frac{\partial u}{\partial y}_d - \eta\left(\dot{\gamma}\right)_e \frac{\partial u}{\partial y}_e\right) \nonumber \\
    & - u_{ij}^n \Delta t \left(\frac{v_{i+1j}^n - v_{i-1j}^n}{2 \Delta x}\right) 
    - v_{ij}^n \Delta t \left(\frac{v_{ij+1}^n - v_{ij-1}^n}{2 \Delta y}\right) 
    + \beta (T_{ij} - T_{ref}) g \nonumber \\
    &b_u = u_{ij}^n - \frac{\Delta t}{\rho} \frac{p_{ij}^n - p_{i-1j}^n}{\Delta x} 
    - u_{ij}^n \Delta t \left(\frac{u_{i+1j}^n - u_{i-1j}^n}{2 \Delta x}\right) 
    - v_{ij}^n \Delta t \left(\frac{u_{ij+1}^n - u_{ij-1}^n}{2 \Delta y}\right) \nonumber \\
    & + \frac{\Delta t}{\rho \Delta y} \left(\eta\left(\dot{\gamma}\right)_i \frac{\partial v}{\partial x}_i - \eta\left(\dot{\gamma}\right)_s \frac{\partial v}{\partial x}_s\right) \nonumber \\
    &a_u=\left(1+2\frac{\left({\eta\left(\dot{\gamma}\right)}_d+\ {\eta\left(\dot{\gamma}\right)}_e\right)\mathrm{\Delta t}\ }{\rho\Delta x^2}+\ \frac{\left({\eta\left(\dot{\gamma}\right)}_i+\ {\eta\left(\dot{\gamma}\right)}_s\right)\mathrm{\Delta t}\ }{\rho\Delta y^2}\right) \nonumber \\
\end{align}

Para a discretização do cálculo da correção da pressão, a equação resultante é uma equação de Poisson, de forma que sua estabilidade numérica é mais delicada que as demais equações, não atendendo por exemplo os requisitos de convergência dos métodos numéricos mais comuns, como será demostrado mais adiante. Retomando a equação \ref{correção-pressão-completo} temos, Eq. \ref{p' - aberto}:

\begin{align}
    &{\vec{\mathrm{\nabla}}}^2p^\prime=\frac{\rho}{\mathrm{\Delta t}}\ \vec{\mathrm{\nabla}}\cdot{\vec{v}}^\ast\nonumber \\
    &\frac{\partial^2 p'}{\partial x^2} + \frac{\partial^2 p'}{\partial y^2} = \frac{\rho}{\Delta t} \left( \frac{\partial u^*}{\partial x} + \frac{\partial v^*}{\partial y} \right) \label{p' - aberto}
\end{align}

Seguindo os mesmos critérios das equações anteriores temos, Eq. \ref{correção da pressão - aberta - discretizada}:

\begin{align}
    &\left(\frac{p_D^\prime - p_P^\prime}{\Delta x} - \frac{p_P^\prime - p_E^\prime}{\Delta x} \right) \Delta y 
    + \left(\frac{p_I^\prime - p_P^\prime}{\Delta y} - \frac{p_P^\prime - p_S^\prime}{\Delta y} \right) \Delta x \nonumber\\
    &= \frac{\rho}{\Delta t} \left( (u_p^\ast - u_E^\ast) \Delta y + (v_p^\ast - v_S^\ast) \Delta x \right) \nonumber\\
    &\quad + \left(\frac{p'_D - 2p'_P + p'_E}{\Delta x^2}\right) + \left(\frac{p'_I - 2p'_P + p'_S}{\Delta y^2}\right) \nonumber\\
    &= \frac{\rho}{\Delta t} \left( \frac{(u^*_P - u^*_E)}{\Delta x} + \frac{(v^*_P - v^*_S)}{\Delta y} \right)\label{correção da pressão - aberta - discretizada}
\end{align}


Agrupando termos semelhantes e substituindo a nomenclatura, Eq. 5.6-23:

\begin{equation}
    p_{ij}^\prime\left(\frac{2}{\Delta x^2}+\ \frac{2}{\Delta y^2}\right)-\frac{p_{i+1j}^\prime}{\Delta x^2}\ -\frac{p_{i-1j}^\prime}{\Delta x^2}-\frac{p_{ij+1}^\prime}{\Delta y^2}-\frac{p_{ij-1}^\prime}{\Delta y^2}=\ \frac{\rho}{\mathrm{\Delta t}}\left(\frac{\left(\ u_{ij}^\ast-\ u_{i-1j}^\ast\right)}{\Delta x}+\frac{\left(\ v_{ij}^\ast-\ v_{ij-1}^\ast\right)}{\Delta y}\right)
    \label{correção-pressão-final-discreto}
\end{equation}

O desenvolvimento numérico das demais etapas do método de Chorin é simples em relação aos demais, Eq. \ref{chorin - vet} a \ref{chorin - v}:

\begin{align}
    &\vec{v}=\ {\vec{v}}^\ast-\frac{\Delta t}{\rho}\vec{\mathrm{\nabla}}p^\prime \label{chorin - vet} \\
    &u=\ u^\ast-\frac{\Delta t}{\rho}\frac{\partial p^\prime}{\partial x} \label{chorin - u} \\
    &v=\ v^\ast-\frac{\Delta t}{\rho}\frac{\partial p^\prime}{\partial x} \label{chorin - v}
\end{align}

Utilizando a nomenclatura adotada, Eq. \ref{correção-velocidade-final-u} e \ref{correção-velocidade-final-v}:

\begin{align}
    &u_{ij}^{n+1}=\ u_{ij}^\ast-\frac{\Delta t}{\rho}\left(\frac{p_{ij}^\prime-p_{i-1j}^\prime}{\Delta x}\right)\label{correção-velocidade-final-u} \\
    &v_{ij}^{n+1}=\ v_{ij}^\ast-\frac{\Delta t}{\rho}\left(\frac{p_{ij}^\prime-p_{ij-1}^\prime}{\Delta y}\right)\label{correção-velocidade-final-v}
\end{align}

Finalmente, para a correção da pressão, Eq. \ref{p = p' +p0 discreto}:

\begin{align}
    &p=p_0+\ p^\prime  \nonumber \\
    &p_{ij}^{n+1}=p_{ij}^n+\ p_{ij}^\prime \label{p = p' +p0 discreto}
\end{align}

Com isso se tem todas as equações necessárias para construir as rotinas computacionais.

\section{Métodos de Solução de Sistemas Lineares}

Para a solução de sistemas lineares do tipo Ax=b comumente é utilizado os chamados métodos iterativos. Dois dos métodos iterativos mais utilizados são o método de Gauss-Seidel, utilizado principalmente no meio acadêmico devido sua fácil implementação, e o método dos Gradientes conjugados, utilizado principalmente pela sua rápida convergência.

\subsection{Método de Gauss-Seidel}

O método de Gauss-Seidel se baseia-se na decomposição da matriz $A$ em uma soma de três componentes: a parte diagonal, a parte estritamente triangular inferior e a parte estritamente triangular superior. A cada iteração, o método atualiza os valores de $x$ utilizando diretamente os resultados obtidos nas iterações anteriores \citep{Fortuna2000}.

Sua fórmula iterativa pode ser expressa como, Eq. \ref{sistema gauss-seidel}:

\begin{equation}
    x_i^{k+1}=\ \frac{1}{a_{ii}}(b_i-\ \sum_{j<i}{a_{ij}k_j^{k+1}-\sum_{j>i}{a_{ij}k_j^{k+1}\ }\ }
    \label{sistema gauss-seidel}
\end{equation}

Na qual $x_i^{k+1}$ é o valor atualizado da incógnita $i$-ésima na $k$-ésima iteração.

A condição de convergência do método de Gauss-Seidel é que a matriz $A$ seja diagonal dominante, ou seja, os termos da diagonal da matriz $A$ devem ser maiores que a soma dos demais termos na linha. Por este critério ser simples de se atender para sistemas transientes, como as equações do balanço do momentum linear. Nessas situações, o método de Gauss-Seidel tende a convergir a uma taxa aceitável.

Por outro lado, o método se torna pouco eficiente quando aplicado em equações como a equação de Poisson, por esta não ter uma diagonal dominante claramente definida. Isto faz com que a convergência se torne extremamente difícil para estes casos particulares. 

É importante ressaltar que apesar de sua robustez em alguns casos, o método de Gauss-Seidel geralmente apresenta uma taxa de convergência mais lenta em comparação ao método dos Gradientes Conjugados, mesmo quando os critérios de convergência são atendidos. Mais informações a respeito deste método podem ser encontradas em \citet{Fortuna2000}.

\subsection{Método dos gradientes conjugados}

A teoria acerca do método dos gradientes conjugados, exige uma formulação complexa, que dá origem a vários métodos de resolução de sistemas lineares além deste, como os gradientes biconjugados. Para o melhor entendimento desta formulação, uma descrição matemática detalhada é apresentada por \citet{Shewchuk1994}. No presente trabalho, será abordado apenas os critérios de convergência e as etapas para a aplicação do método, que serão abordadas no modelo computacional.

A convergência do método dos Gradientes Conjugados requer que a matriz $A$ seja simétrica e definida positiva. A condição de definida positiva é satisfeita por todos os sistemas utilizados na presente formulação. No entanto, a simetria da matriz depende do modelo físico em questão. Por exemplo, no caso de fluidos newtonianos, o sistema linear resultante do balanço da quantidade de movimento linear é simétrico. Por outro lado, essa condição de simetria é violada quando a viscosidade varia ao longo do domínio, o que impede a aplicação do método em todos os casos para este conjunto de sistemas.

Sistemas que envolvem propriedades escalares, como temperatura ou correção de pressão, cuja matriz $A$ não varia ao longo do domínio físico, podem ser resolvidos eficientemente com o método dos Gradientes Conjugados. Nessas situações, o método garante uma rápida convergência e supera as dificuldades enfrentadas pelo método de Gauss-Seidel, especialmente ao lidar com a equação de Poisson.

Unindo ambas as metodologias, o presente trabalho adotou o método de Gauss-Seidel para a solução dos sistemas de $u^\ast$ e $v^\ast$, e o método dos gradientes conjugados para a solução de $p^\prime$ e $T$.

\section{Convergência e Estabilidade}

Para garantir a estabilidade dos sistemas lineares de interesse, foi adotado a condição de Courant-Friedrichs-Lewy ou CFL \citep{Courant1928}. Que é um critério que garante que as informações dentro do escoamento não irão “pular” de células dentro de um único passo de tempo, o que poderia deixar o escoamento instável. Um exemplo do caso que é evitado pelo $CFL$ pode ser visto na Figura \ref{informação pulando de celula}. O valor utilizado em todas as simulações presentes foi tal que $ CFL\le0.8$.

Entre os métodos numéricos utilizados, o método de Gauss-Seidel, apesar de sua simplicidade e convergência mais lenta, demonstrou ser menos sensível tanto aos critérios de convergência quanto ao tamanho do passo de tempo nas simulações. Nos testes realizados, o método alcançou a convergência mesmo com valores do número de $CFL$ superiores a 2. 

Em contraste, o método dos Gradientes Conjugados mostrou-se significativamente mais sensível às condições do sistema linear e ao valor do $CFL$. Observou-se que, com valores de $CFL$ ligeiramente superiores a 1, o método perdeu sua convergência. Embora apresente uma taxa de convergência mais rápida quando dentro dos critérios de estabilidade, o método dos Gradientes Conjugados exigiu um controle mais rigoroso sobre o passo de tempo para garantir resultados estáveis.

\begin{figure}[ht]
    \centering
    \includegraphics[width=0.5\linewidth]{imagens/informacao_pulando_de_celula.png}
    \caption{informação pulando de célula, caso que é evitado com o CFL.}
    \label{informação pulando de celula}
    {\footnotesize Fonte: Do próprio autor}
\end{figure}

Para verificar a convergência das simulações, principalmente da equação de Poisson por ser mais sensível, se utilizou o critério do divergente do campo de pressão. Considerando que a equação da continuidade, que deu origem à equação de Poisson para a correção da pressão, tenha sido bem resolvida pelo sistema, tem-se que, Eq. \ref{gradientes - para - conferir }:

\begin{equation}
    \vec{\mathrm{\nabla}}\cdot\left(\vec{v}\right)=\frac{\partial u}{\partial x}+\ \frac{\partial v}{\partial y}=0 
    \label{gradientes - para - conferir }
\end{equation}

De forma que a soma dessa equação em todo domínio numérico, não pode superar a tolerância dos métodos numéricos adotados. Este cálculo foi realizado a cada passo de tempo durante a validação das rotinas e é feito ao final de cada simulação para garantir coerência física para os resultados.



%====================================================================
% Resultados.
%====================================================================
%====================================================================
% Resultados: Escreva logo após o 
% \chapter{Resultados} o texto de seu trabalho referente 
% aos resultados obtidos no desenvolvimento de seu estudo. 
%====================================================================
\chapter{Modelos Computacionais}

Com base no modelo numérico, foram construídas rotinas computacionais para solucionar os escoamentos. O código foi desenvolvido em linguagem C, utilizando as bibliotecas padrões de entrada e saída de dados. O software desenvolvido está disponibilizado integralmente no apêndice B, nesta secção serão feitos comentários e explicações pontuais a respeito da implementação dos métodos, as condições de contorno e o valor das propriedades físicas utilizadas

As propriedades físicas do escoamento foram variadas a fim de garantir que os valores adimensionais sejam compatíveis com a literatura, para a comparação de resultados. 

\section{Couette Plano}

Para o escoamento de Couette, como já detalhado no modelo computacional e visível da Figura \ref{Couettefisico}. Tem-se as seguintes condições de contorno.

O valor de u na parede superior é igual a velocidade de referência, neste caso adotado como $1\ \left[\frac{m}{s}\right]$, com esta exceção, todos os valores de condições de contorno, respeitando a Tabela \ref{contorno - couette}, são iguais a zero.	
\begin{table}[ht]
    \caption{Condições de contorno para o escoamento de Couette}
    \vspace{0.1cm}
    \centering
    \begin{tabular}{c c c c }
    \hline
    {\bfseries Parede/Propriedade} & $ \bm{u}$ & $ \bm{v}$ & $ \bm{p}$ \\
    \hline
    Esquerda & 2ª espécie & 2ª espécie & 1ª espécie \\

    Direita & 2ª espécie & 2ª espécie & 1ª espécie \\

    Superior & 1ª espécie & 1ª espécie & 2ª espécie \\

    Inferior & 1ª espécie & 1ª espécie & 2ª espécie \\
    \hline
    \end{tabular}
    \label{contorno - couette}\\
    \vspace{0.4cm}
    {\footnotesize Fonte: Do próprio autor}
\end{table}


As propriedades $\eta$, $\nu$ e m foram variadas de forma que se obtivesse diferentes números de Reynolds ao longo do escoamento, tanto no caso newtoniano quanto no caso não-newtoniano. Os valores utilizados bem como os resultados correspondentes a eles são detalhados no capítulo de resultados.

Os valores adimensionais utilizados para os escoamentos de Couette simulados, bem como as propriedades dos métodos numéricos como a tolerância e a quantidade máxima de iterações de Gauss-Seidel são dispostos na tabela \ref{propriedades - couette}, e serão retomados na secção de resultados.

\begin{table}[ht]
    \caption{Condições de contorno para o escoamento de Couette}
    \vspace{0.2cm}
    \centering
    \begin{tabular}{ l c }
    \hline
    {\bfseries Propriedade} & {\bfseries Valor} \\
    \hline
    Velocidade fronteira superior ($v_{ref}$) & 1 $\frac{m}{s}$ \\
    Massa específica ($\rho$) & 1 $\frac{kg}{m^3}$ \\
    Tolerância & $10^{-9}$ \\
    H & 1 [m] \\
    L & 1 [m] \\
    Máximo de iterações & $10^4$ \\
    \hline
    \end{tabular}
    \label{propriedades - couette} \\
    \vspace{0.2cm}
    {\footnotesize Fonte: Do próprio autor}
\end{table}

O domínio utilizado foi de 41x41, sendo que, para garantir um $CFL$ baixo, se utilizou 9600 passos de tempo, o que garantiu a convergência para todos os escoamentos simulados. O tempo computacional final foi de $100 (s)$.

O domínio utilizado é representado na Figura \ref{Couettenumerico}:

\begin{figure}[ht]
    \centering
    \includegraphics[width=0.7\textwidth]{imagens/Couette-computacional.png}
    \caption{Modelo computacional do escoamento de Couette.}
    \label{Couettenumerico}
    {\footnotesize Fonte: Do próprio autor} 
\end{figure}

\section{Poiseuille Plano}

As condições do escoamento de Poiseuille plano são mostrados na tabela \ref{contorno - Poiseuille}, o escoamento de Poiseulle, assim como o de Couette, foi utilizado para testar tantos fluidos newtonianos quanto os não-newtonianos.

\begin{table}[H]
    \caption{Condições de contorno para o escoamento de Poiseuille}
    \vspace{0.2cm}
    \centering
    \begin{tabular}{ c c c c }
    \hline
    {\bfseries Parede/Propriedade} & $ \bm{u}$ & $ \bm{v}$ & $ \bm{p}$ \\
    \hline
    Esquerda & 1ª espécie & 1ª espécie & 2ª espécie \\

    Direita & 2ª espécie & 2ª espécie & 1ª espécie \\

    Superior & 1ª espécie & 1ª espécie & 2ª espécie \\

    Inferior & 1ª espécie & 1ª espécie & 2ª espécie \\
    \hline
    \end{tabular}
    \label{contorno - Poiseuille}\\
    \vspace{0.2cm}
    {\footnotesize Fonte: Do próprio autor}
\end{table}

Pare este escoamento, as condições de contorno são iguais a zero, com exceção da componente $u$ do vetor velocidade à esquerda, que tem um perfil uniforme de velocidade, variável entre os escoamentos.

As mesmas propriedades, que foram variadas no escoamento de Couette, foram variadas para diferentes números de Reynolds neste mesmo escoamento, porém, além destas também se variou a velocidade de entrada a esquerda do escoamento. Como mostrado na secção de resultados.

As dimensões deste escoamento, ilustradas na Figura \ref{Poiseuillenumerico}, são maiores que os demais, pois o perfil de velocidade precisa de espaço e tempo para se desenvolver completamente. Neste caso a escolha do domínio físico pode influenciar no transiente e no resultado, para diferentes números de Reynolds. A escolha de malha utilizada foi de 100x100, para garantir que para altos números de $n$ e de Reynolds, o escoamento possa convergir.

\begin{figure}[ht]
    \centering
    \includegraphics[width=0.5\textwidth]{imagens/Poiseulle-numerico.png}
    \caption{Modelo computacional do escoamento de Poiseulle.}
    \label{Poiseuillenumerico}
    {\footnotesize Fonte: Do próprio autor} 
\end{figure}

As propriedades físicas que não foram variadas para o escoamento de Poiseuille são as mesmas que para o escoamento de Couette, com exceção da dimensão L, de forma que podem ser referenciadas pela tabela \ref{propriedades - Poiseuille}.

\begin{table}[ht]
    \caption{Propriedades físicas para o escoamento de Poiseuille}
    \vspace{0.2cm}
    \centering
    \begin{tabular}{ll}
    \toprule
    {\bfseries Propriedade} & {\bfseries Valor} \\
    \midrule
    Velocidade fronteira superior ($v_{ref}$) & \SI{1}{\meter\per\second} \\
    Massa específica ($\rho$) & \SI{1}{\kilogram\per\cubic\meter} \\
    Tolerância & \SI{1e-9}{} \\
    H & \SI{1}{\meter} \\
    L & \SI{30}{\meter} \\
    Máximo de iterações & $10^4$ \\
    Diferença de pressão ($\Delta p$) & \SI{0.6}{\pascal} \\
    \bottomrule
    \end{tabular}
    \label{propriedades - Poiseuille} \\
    \vspace{0.2cm}
    {\footnotesize Fonte: Do próprio autor}
\end{table}

\section{Cavidade com Tampa Deslizante}

Sendo o escoamento da cavidade com tampa deslizante um escoamento mais complexo que os demais, tem-se que suas condições de contorno, mostradas na Tabela \ref{contorno - lid-driven}, trazem complicações numéricas.

\begin{table}[ht]
    \caption{Condições de contorno para o escoamento de cavidade com tampa deslizante}
    \vspace{0.2cm}
    \centering
    \begin{tabular}{ c c c c }
    \hline
    {\bfseries Parede/Propriedade} & $ \bm{u}$ & $ \bm{v}$ & $ \bm{p}$ \\
    \hline
    Esquerda & 1ª espécie & 1ª espécie & 2ª espécie \\

    Direita & 1ª espécie & 1ª espécie & 2ª espécie \\

    Superior & 1ª espécie & 1ª espécie & 2ª espécie \\

    Inferior & 1ª espécie & 1ª espécie & 2ª espécie \\
    \hline
    \end{tabular}
    \label{contorno - lid-driven}\\
    \vspace{0.2cm}
    {\footnotesize Fonte: Do próprio autor}
\end{table}

Para este escoamento, por ter condições de segunda espécie em todas as direções de $p$, a equação de Poisson resulta em um sistema indeterminado e singular, com determinante igual a zero. Tal condição faz com que a equação da correção da pressão não satisfaça os critérios para a convergência do método de Gauss-Seidel. Esse é um dos exemplos de escoamentos que mostram a instabilidade da equação de correção da pressão e porque, dentre as equações resolvidas para resolver escoamentos, esta é uma das mais difíceis em termos numéricos. 

Para os valores de condição de contorno, é mostrado na Figura \ref{lid-drivennumerico} que todos os valores, são iguais a zero, com exceção da fronteira superior que é igual a velocidade de referência. A malha utilizada variou de acordo com o número de Reynolds, de forma que um maior número de Reynolds exigiu uma malha mais refinada. Pode-se assumir que uma malha de 131x131 resolve todo o intervalo de número de Reynolds proposto.

\begin{figure}[ht]
    \centering
    \includegraphics[width=0.6\textwidth]{imagens/LId-driven-Numerico.png}
    \caption{Modelo computacional da cavidade com tampa deslizante.}
    \label{lid-drivennumerico}
    {\footnotesize Fonte: Do próprio autor} 
\end{figure}

Para este escoamento apenas a viscosidade foi variada a fim de se variar o número de Reynolds, visto que este problema não foi utilizado para validar modelos não-newtonianos. As propriedades constantes são as mesmas que as dos escoamentos de Couette e Poiseuille.

\section{Cavidade Diferencialmente Aquecida}

Para este escoamento, se faz necessário a definição das condições de contorno de mais uma propriedade, a temperatura. As condições de contorno são descritas pela tabela \ref{contorno - thermal}:

\begin{table}[ht]
    \caption{Condições de contorno para cavidade diferencialmente aquecida}
    \vspace{0.2cm}
    \centering
    \begin{tabular}{ c c c c c }
    \hline
    \bfseries {Parede/Propriedade} & $ \bm{u}$ & $ \bm{v}$ & $ \bm{p}$ & $ \bm{T}$ \\
    \hline
    Esquerda & 1ª espécie & 1ª espécie & 2ª espécie & 1ª espécie \\

    Direita & 1ª espécie & 1ª espécie & 2ª espécie & 1ª espécie \\

    Superior & 1ª espécie & 1ª espécie & 2ª espécie & 2ª espécie \\

    Inferior & 1ª espécie & 1ª espécie & 2ª espécie & 2ª espécie \\
    \hline
    \end{tabular}
    \label{contorno - thermal}\\
    \vspace{0.2cm}
    {\footnotesize Fonte: Do próprio autor}
\end{table}

Os valores utilizados para as condições de contorno são zero em todos em todas as direções para todas as propriedades com exceção da temperatura. A temperatura das paredes esquerda e direita são definidas como $80 ^\circ\mathrm{C}$ e $50 ^\circ\mathrm{C}$  respectivamente e permanecidas constantes durante todo o escoamento.

As propriedades físicas utilizadas são descritas pela Tabela \ref{propriedades - thermal}.

\begin{table}[ht]
    \caption{Propriedades físicas utilizadas do escoamento de cavidade diferencialmente aquecida}
    \vspace{0.2cm}
    \centering
    \begin{tabular}{l l}
    \toprule
    {\bfseries Propriedade} & {\bfseries Valor} \\
    \midrule
    Velocidade fronteira superior ($v_{ref}$) & \SI{1}{\meter\per\second} \\
    Massa específica ($\rho$) & \SI{1}{\kilogram\per\cubic\meter} \\
    Tolerância & \SI{1e-9}{} \\
    H & \SI{1}{\meter} \\
    L & \SI{1}{\meter} \\
    Máximo de iterações & $10^4$ \\
    $\alpha$ & \SI{0.01}{\meter\squared\per\second} \\
    $g$ & \SI{9.81}{\meter\per\second\squared} \\
    $\nu$ & \SI{0.0071}{\meter\squared\per\second} \\
    $Pr = \frac{\nu}{\alpha}$ & 0.71 \\
    \bottomrule
    \end{tabular}
    \label{propriedades - thermal} \\
    \vspace{0.2cm}
    {\footnotesize Fonte: Do próprio autor}
\end{table}

O coeficiente de expansão térmica é variável e calculado a partir da seguinte equação, Eq. \ref{equação para betha}:

\begin{equation}
\beta = \frac{Ra \cdot \nu  \cdot \alpha}{g \cdot (T_{quente} - T_{frio}) \cdot H^3}
    \label{equação para betha}
\end{equation}

De forma que se possa impor o valor de coeficiente de Rayligh para diferentes condições de escoamento. Possibilitando comparar estes valores com a literatura. 

A Figura \ref{Thermal-numerico} exemplifica o domínio físico utilizado para as equações, é possível observar a semelhança com o escoamento anterior, mudando apenas a velocidade da fronteira superior e a questão térmica do escoamento:

\begin{figure}[ht]
    \centering
    \includegraphics[width=0.65\textwidth]{imagens/thermal-numerico.png}
    \caption{Modelo computacional da cavidade térmica.}
    \label{Thermal-numerico}
    {\footnotesize Fonte: Do próprio autor} 
\end{figure}

A malha computacional utilizada para este escoamento foi de 100x100, a fim de se garantir que todo o intervalo de números de Rayleigh fosse atingido. 

\section{Algoritmos de Solução de Sistemas Lineares}

A implementação do método de Gauss-seidel é feita da como mostra a Figura \ref{Algoritmo-Seidel}

\begin{figure}[ht]
    \centering
    \includegraphics[height=0.5\textwidth]{imagens/loop-Gauss-Seidel.png}
    \caption{Lógica do algoritmo de Gauss-Seidel utilizado no código.}
    \label{Algoritmo-Seidel}
    {\footnotesize Fonte: Do próprio autor} 
\end{figure}

Inicialmente, o método começa com a atualização das condições de contorno, garantindo que os valores das células fantasmas sejam atualizadas de forma correta. Em seguida, se aplica a equação \ref{sistema gauss-seidel}, onde cada incógnita é atualizada de forma sucessiva com base nos valores mais recentes das demais incógnitas. Após cada iteração, calcula-se o erro, comparando a solução atual com a obtida no passo anterior. O processo continua até que um dos critérios de parada do número máximo de equações ou o erro menor que o especificado seja atingido. Por ser um método iterativo relativamente simples, o Gauss-Seidel requer menos etapas de implementação em comparação com os gradientes conjugados.

O método dos gradientes conjugados, por ser mais complexo, requer mais etapas para sua implementação. seu algorítimo é representado pela Figura \ref{Algoritmo-Conjulgados}.

\begin{figure}[ht]
    \centering
    \includegraphics[height=1.0\textwidth]{imagens/loop-gradientes.png}
    \caption{Diagrama de blocos do método dos gradientes conjugados.}
    \label{Algoritmo-Conjulgados}
    {\footnotesize Fonte: Do próprio autor} 
\end{figure}


Inicialmente se inicializa os vetores de resíduo $r$ e o vetor conjugado $p$ com o estado inicial do sistema. Na rotina desenvolvida se utilizou como chute inicial para essa inicialização o passo de tempo anterior para acelerar a convergência do método. Após isto se calcula um passo $\alpha$ e se desloca os vetores $x$ e $r$ na direção $p$ com um passo $\alpha$. Caso o critério de tolerância não tenha sido atingido, se calcula uma correção $\beta_k$ de forma que se possa iniciar o passo de tempo posterior

Devido a teoria de gradientes conjugados garantir a convergência do método em no máximo $n$ iterações para um sistema de $n$ variáveis, não se utilizou o critério de número máximo de iterações para a parada, uma vez que caso a convergência não tenha sido atingida antes desse valor o método falhou em resolver o sistema. Neste caso o código é interrompido.


%====================================================================
% Conclusões.
%====================================================================
%====================================================================
% Conclusões: Escreva logo após o 
% \chapter{Conclusões} o texto de seu trabalho referente 
% as conclusões de seu estudo. 
%====================================================================
\chapter{Resultados}

Nesta secção serão abordados os resultados obtidos através das simulações e sua comparação com diferentes autores, o objetivo é validar os métodos implementados e verificar o quão aproximado estão com a literatura. Serão utilizados, além de artigos consolidados, resultados analíticos do escoamento como mostrado no apêndice A.

\section{Escoamento de Couette Plano}
Para este escoamento, devido sua simplicidade, se dispõe na literatura de soluções analíticas para a modelagem de seu comportamento. Primeiramente foi feita a análise em regime permanente para fluidos newtonianos, a fim de se verificar a estabilidade do método. Nas figuras \ref{couette - newton - p1} e \ref{couette - newton - p2}, é visível a evolução do escoamento para fluidos newtonianos para diferentes números de Reynolds.

\begin{figure}[ht]
    \centering
    \subfloat[Perfil de velocidade para numéro de Reynolds 100. Utilizando $\nu = 0.01$]{%
        \includegraphics[width=0.45\textwidth]{imagens/Couette_N_Re100.png}
    }
    \hspace{0.05\textwidth} % espaço entre as figuras
    \subfloat[Perfil de velocidade para numéro de Reynolds 400. Utilizando $\nu = 0.0025$]{%
        \includegraphics[width=0.45\textwidth]{imagens/Couette_N_Re400.png}
    }
    \caption{Perfils de velocidade no centro da cavidade para valores de Reynolds de 100 e 400}
    \label{couette - newton - p1}
    {\footnotesize Fonte: Do próprio autor}
\end{figure}


\begin{figure}
    \centering
    \subfloat[Perfil de velocidade para numéro de Reynolds 1000. Utilizando $\nu = 0.001$]{%
        \includegraphics[width=0.45\textwidth]{imagens/Couette_N_Re1000.png}
    }
    \hspace{0.05\textwidth} % espaço entre as figuras
    \subfloat[Perfil de velocidade para numéro de Reynolds 1600. Utilizando $\nu = 0.000625$]{%
        \includegraphics[width=0.45\textwidth]{imagens/Couette_N_Re1600.png}
    }
    \caption{Perfils de velocidade no centro da cavidade para valores de Reynolds de 1000 e 1600}
    \label{couette - newton - p2}
    {\footnotesize Fonte: Do próprio autor}
\end{figure}

É possível observar a influência do número de Reynolds no desenvolvimento do escoamento, de forma que quanto maior esse valor, mais lentamente ele se desenvolve para o regime permanente.
Independentemente do número de Reynolds ou do tipo de fluido, o escoamento sempre converge para uma reta, como é possível demonstrar em sua solução analítica apresentada por \citet{Vasconcellos2024}.

Para validar a simulação com o modelo analítico, foi simulado o escoamento de Couette, com o tempo indo de 0 a 100, e se comparou as soluções numéricas e analíticas, Figura \ref{couette-newton-analitico}. O valor de $\nu=\ 0.000625\ \frac{m^2}{s}$ foi utilizado para garantir o valor de Reynolds desejado $(Re = 1600)$. É possível observar o nível de coerência entre os resultados, de forma que não são perceptíveis desvios entre eles.

\begin{figure}[ht]
    \centering
    \includegraphics[width=0.7\textwidth]{imagens/Perfil de velocidade_Couette_analitico_numerico.png}
    \caption{Comparação entre o perfil analítico e o perfil numérico simulados para diferentes tempos com um número de Reynolds de 1600 e um valor de $\nu$ igual a 0.000625. As linhas contínuas representam o perfil numérico e as marcas * representam os pontos da solução analítica.}
    \label{couette-newton-analitico}
    {\footnotesize Fonte: Do próprio autor} 
\end{figure}

Para os modelos de escoamentos não-newtonianos, foi utilizado a Lei de potência e se variou os valores de m e n para atingir os valores de Reynolds desejado com tempos fixos. Na Figura \ref{couette - nãonewton} é mostrado o comportamento de fluidos com diversos valores de n para $Re = 100$ e $Re =  400$.

\begin{figure}[ht]
    \centering
    \subfloat[perfis de velocidade para diversos valores de n e para um número de Reynolds de 100, utilizando a Lei de Potência. Tempo final de simulação 5 s]{%
        \includegraphics[width=0.45\textwidth]{imagens/NN_couette_Re100.png}
    }
    \hspace{0.05\textwidth} % espaço entre as figuras
    \subfloat[perfis de velocidade para diversos valores de n e para um número de Reynolds de 400, utilizando a Lei de Potência. Tempo final de simulação 5 s]{%
        \includegraphics[width=0.45\textwidth]{imagens/NN_couette_Re400.png}
    }
    \caption{Perfils de velocidade no centro da cavidade para diferentes valores de Reynolds, utilizando fluidos não-newtonianos}
    \label{couette - nãonewton}
    {\footnotesize Fonte: Do próprio autor}
\end{figure}

É possível observar que, apesar de seguirem a mesma tendência que os fluidos newtonianos com a variação do número de Reynolds, para um maior valor deste número mais lenta é a estabilização para regime permanente, os fluidos não-newtonianos são fortemente influenciados pelos valores da Lei de Potência. 

Para fluidos com n\ >\ 1 o escoamento se estabiliza mais rapidamente, um comportamento esperado para fluídos dilatantes que aumentam sua viscosidade com a taxa de cisalhamento o que gera um comportamento similar a um fluido newtoniano de alta viscosidade para o escoamento de Couette. Por outro lado, para valores de n\ <\ 1, tem-se uma demora maior para a estabilização do escoamento quando comparado com fluidos newtonianos (n\ =1), o que é um comportamento esperado de fluidos pseudoplásticos, que tem o comportamento viscoso reduzido com o aumento da taxa de cisalhamento.

Na Figura \ref{couette - nãonewton- compara} se compara os resultados obtidos no presente trabalho, com os resultados obtidos por \citet{Vasconcellos2024}.

\begin{figure}[ht]
    \centering
    \subfloat[Perfil de velocidade para t = 5s. Próprio autor.]{%
        \includegraphics[width=0.45\textwidth]{imagens/NN_couette_Re100.png}
    }
    \hspace{0.05\textwidth} % espaço entre as figuras
    \subfloat[Perfil de velocidade para t = 5s. \citet{Vasconcellos2024}.]{%
        \includegraphics[width=0.45\textwidth]{imagens/5.1 sec Luiz_melhorado.png}
    }
    \caption{Comparação entre os resultados do presente trabalho e aqueles obtidos por \citet{Vasconcellos2024}}
    \label{couette - nãonewton- compara}
\end{figure}

Nessa mesma Figura é possível observar que apesar da semelhança qualitativa entre os resutlados, os valores ainda não são exatos entre os mesmos. Tal fato permanece em aberto no presente trabalho e a principal suspeita é a escolha dos invariantes, que se deu de forma diferente entre os trabalhos.

Fluidos com valor de  n maiores que 2 não são comumente utilizados ou encontrados na literatura, de forma que estes apenas foram considerados para fins de visualização de casos limítrofes.

\section{Escoamento de Poiseuille Plano}

Para o escoamento de Poiseuille se tem os resultados das equações analíticas para comparar os resultados da simulação. Como mencionado no modelo físico, o escoamento de Poiseulle se baseia na diferença de pressão entra a entrada e a saída do modelo físico, de forma que se definiu uma diferença de pressão fixa, evidenciada no modelo computacional. Para que se imponha essa diferença de pressão, a velocidade de entrada é ajustada de acordo com as equações do apêndice A.

Inicialmente, para se validar as equações desenvolvidas para este escoamento, é simulado um escoamento newtoniano com valor de número de $Re = 100$. Para este escoamento foram utilizados uma viscosidade cinemática de  $\nu=\ 0.01 \frac{m^2}{s}$ e um perfil de velocidade de entrada de $V\ =\ \frac{1}{6}\ \left[\frac{m}{s}\right]$ para que se mantivesse a diferença de pressão inicialmente definida. O perfil de velocidade resultante no escoamento é mostrado na Figura \ref{Poiseulle-newton-analitico}.


 \begin{figure}[ht]
    \centering
    \includegraphics[width=0.6\textwidth]{imagens/Poiseulle_newtonian_Re100.png}
    \caption{Perfil de velocidade analítico em comparação com o modelo computacional, utilizando um valor de $Re = 100$.  Foram impostos uma viscosidade de $\nu=\ 0.01$ e um perfil de velocidade na entrada de $V\ =\ \frac{1}{6}\ \left[\frac{m}{s}\right]$.}
    \label{Poiseulle-newton-analitico}
    {\footnotesize Fonte: Do próprio autor} 
\end{figure}

É possível observar, a partir desses resultados, o nível de coerência do modelo analítico com o computacional, de forma que se possa utilizar estes valores de forma confiável para verificar os resultados da simulação. A tabela \ref{poiseulle-newtonian-compara-tabela} mostra a comparação entre os valores analíticos e numéricos e o erro entre eles.

\begin{table}[ht]
\centering
\caption{Comparação entre os valores da componente u analíticos e computacional para $Re =100$}
\begin{tabular}{c c c c}
\hline
{\bfseries Distância em $y$} [m] & {\bfseries} Analítico $ \bm{\frac{m}{s}}$ & {\bfseries Numérico} $ \bm{\frac{m}{s}}$ & {\bfseries Erro relativo (\%)} \\
\hline
0,99 & 0,012 & 0,012181 & 1,12 \\

0,82 & 0,15  & 0,149436 & 0,02 \\

0,67 & 0,22  & 0,220737 & 0,05 \\

0,52 & 0,249 & 0,249257 & 0,06 \\

0,50 & 0,25  & 0,249851 & 0,06 \\

0,48 & 0,249 & 0,249257 & 0,06 \\

0,33 & 0,22  & 0,220737 & 0,05 \\

0,18 & 0,15  & 0,149436 & 0,02 \\

0,012 & 0,012 & 0,012181 & 1,12 \\
\hline
\end{tabular}
\label{poiseulle-newtonian-compara-tabela}\\
{\footnotesize Fonte: Do próprio autor} 
\end{table}

Tem-se que os maiores erros ocorrem nas extremidades das curvas, enquanto o erro no centro do domínio se aproxima de zero. Esse resultado mostra que apesar das considerações feitas nas discretizações do modelo numérico, os resultados obtidos pelo modelo computacional são suficientes para comparação com a literatura.

Define-se a diferença de pressão no escoamento de maneira indireta a partir de outros parâmetros, com base nas seguintes equações (apêndice A):

\begin{align*}
    &u\left(y\right)=\frac{\lambda y^2}{2}\ \ -\ \frac{\lambda hy}{1} \\
    &u_{medio}=\frac{\lambda h^3}{12} \\
    &\lambda\ =\ \frac{1}{\nu}\frac{\mathrm{\Delta p}}{L}
\end{align*}

Considerando uma altura $H = 1 [m]$, largura $L = 30 [m]$, viscosidade $\nu$ constante, é possível impor a queda de pressão neste tipo de escoamento a partir da imposição da velocidade de entrada $u_{medio}$. Para visualização do efeito da viscosidade no escoamento, considerou-se uma queda de pressão fixa de $\Delta p = 0.6 Pa$. Variando o número de Reynolds para diferentes valores, obteve-se os perfis de velocidade mostrados na Figura \ref{Poiseulle-newton-varios RE}.

 \begin{figure}[ht]
    \centering
    \includegraphics[width=0.7\textwidth]{imagens/Poiseulle_nn_Re100_0.6.png}
    \caption{Diferentes perfis de velocidade a $\Delta p = 0.6$ e diferentes números de Reynolds.}
    \label{Poiseulle-newton-varios RE}
    {\footnotesize Fonte: Do próprio autor} 
\end{figure}

Para $Re = 100$, o perfil de velocidade é mais estreito e a velocidade máxima é menor que os demais, de forma que os efeitos viscosos são maiores no escoamento, levando a gradientes de velocidade menores nesta região. Para valores de Reynolds de $400$ e $1000$, o perfil de velocidade é mais acentuado e a velocidade máxima do escoamento é maior, gerando um aumento no gradiente vertical de velocidade a medida que o numero de Reynolds sobe. Tal comportamento é esperado uma vez que os efeitos viscosos são menos predominantes para altos numéros de Reynolds, o escoamento tende a ficar menos amortecido e com gradientes maiores

Para escoamento de Poiseuille utilizando a Lei de Potência, se variou o número de $n$ para valores fixos de velocidade de entrada e se observou como o escoamento se comportou nestas situações. Na Figura \ref{Poiseulle-nao-newton-varios n} mostra-se o comportamento do fluido para valores de $n$ variando de 1 a 3, com Reynolds fixo em 100 e velocidade média de entrada fixa em $V = \frac{1}{6} \left[\frac{m}{s}\right]$.

\begin{figure}[ht]
    \centering
    \includegraphics[width=0.7\textwidth]{imagens/Poiseulle_nn_v13.png}
    \caption{Diferentes perfis de velocidade para o escoamento de Poiseuille, com  valores de $n$ de 1, 1.5 e 3, para velocidade de entrada constante. Comparação com os resultados do modelo analítico}
    \label{Poiseulle-nao-newton-varios n}
    {\footnotesize Fonte: Do próprio autor} 
\end{figure}

Os valores de $n$ escolhidos, representam fluidos com comportamento newtoniano, para o caso de $n = 1$, e dilatante. Os pontos '*' mostrados nas retas são os resultados do modelo analítico desenvolvido no apêndice A.

Para o caso de $n = 1$ é possível observar que o comportamento é o mesmo que quando simulado utilizando a formulação puramente newtoniana $\eta = \mu$, como o esperado. O escoamento é parabolico e o modelo analítico se ajusta perfeitamente à curva projetada.

Para um valor de $n = 1.5$ se observa uma tendência do perfil de velocidade a se tornar mais agudo, com o aumento da viscosidade. Dessa forma se tem uma velocidade máxima maior para uma mesma velocidade de entrada, o que indica uma maior queda de pressão deste escoamento. A maior viscosidade leva a uma maior dificuldade do fluido escoar.

Para um perfil de velocidade com $n = 3$, um caso extremo e normalmente não representativo para fluidos, o perfil de velocidade se torna bastante agudo, indicando um forte gradiente de pressão presente no escoamento. Consequentemente, a diferênça de pressão deste perfil de velocidade é significativamente maior que os demais.

Como no escoamento de Poiseuille, se impôs o perfil de velocidade de entrada, a queda de pressão resultante do escoamento, para se obter os dados de entrada para a equação analítica, é obtido de forma indireta, de acordo com a velocidade máxima obtida na simulação numérica. Dessa forma se tem as quedas de pressão obtidas em cada uma das simulações mostradas na Figura \ref{Poiseulle-nao-newton-varios n}, indicadas na tabela \ref{queda de pressão para n maior 1}:

\begin{table}[ht]
    {\footnotesize Fonte: Do próprio autor}
    \centering
    \caption{Valores de $n$ e sua respectiva queda de pressão para velocidade de entrada constante}
    \begin{tabular}{c c}
        \hline
        {\bfseries Valor de} $ \bm{n}$ & {\bfseries Queda de pressão (Pa)} \\
        \hline
        1.00 & 12 \\
        1.5 & 24.06 \\
        3.0 & 203.26 \\
        \hline
    \end{tabular}
    \label{queda de pressão para n maior 1}
\end{table}


Para comparar os efeitos da escolha do invariante na modelagem do tensor $\eta$, foram simulados escoamentos com valores de $n = 0.5$ e $n = 2$ com os diferentes os modelos descritos pela Equação \ref{segundo invariante bird} e pela Equação\ref{segundo invariante Aristeu}. Os resultados foram comparados com a solução analítica do modelo e são apresentados pelas Figuras \ref{invariantes 2} e \ref{invariantes 0.5}.


\begin{figure}[ht]
    \centering
    \includegraphics[width=0.7\textwidth]{imagens/invariantes - n - 2}
    \caption{Perfis de velocidade pala um valor de \(n = 2\) e um \(\Delta p = 50 \text{Pa}\). Comparação do perfil analítico com os invariantes apresentados por \citet{Bird1987} e \citet{Aristeu2020}.}
    \label{invariantes 2}
    {\footnotesize Fonte: Do próprio autor} 
\end{figure}

\begin{figure}[ht]
    \centering
    \includegraphics[width=0.7\textwidth]{imagens/invariantes - n - 0.5}
    \caption{Perfis de velocidade para um valor de \( n = 0.5 \) e um \( \Delta p = \sqrt{50} \, \text{Pa} \). Comparação do perfil analítico com os invariantes apresentados por \citet{Bird1987} e \citet{Aristeu2020}.}
    \label{invariantes 0.5}
    {\footnotesize Fonte: Do próprio autor} 
\end{figure}

Na Figura \ref{invariantes 2}, é possível observar que o modelo de invariante apresentado por \citet{Aristeu2020} e apresentado da Equação \ref{segundo invariante Aristeu}, se adequa melhor aos fluidos com comportamento dilatante, de forma que o perfil obtido utilizando o invariante de \citet{Bird1987} não condiz com um perfil fisicamente possível para este escoamento. Desta forma, observou-se que este comportamento é constante no geral, de forma que os perfis mostrados na Figura \ref{queda de pressão para n maior 1} foram todos obtidos com o modelo de \citet{Aristeu2020}, mostrando a consistência do mesmo aos fluidos que são representados por um valor de $n > 1$ na lei de potência. Nota-se ainda que apesar do ajuste compativel com o modelo analítico ao longo de todo o escoamento, ainda há uma diferênça de resultados no centro da cavidade, onde os gradientes de velocidade são maiores.

Já na Figura \ref{invariantes 0.5} é possível observar o comportamento oposto do discutido anteriormente, no qual o modelo apresentado por \citet{Bird1987} se ajustou perfeitamente ao modelo analítico apresentado, enquanto o modelo de \citet{Aristeu2020} apresentou desvios no formato parabolico do perfil de velocidades. Este comportamento se manteve constante durante testes realizados na validação dos modelos no qual se variou, para ambos os tipos de fluidos, o perfil de velocidades de entrada e consequentemente a queda de pressão presente no escoamento. 

Apesar do resultado obtido para o perfil de velocidade para o escoamento de Poiseuille, não se pode afirmar que o mesmo comportamento se repetirá para escoamentos complexos, sendo necessário a comparação, de preferência com resultados experimentais com tais escoamentos. Contudo, os perfis obtidos mostram claramente a influência da escolha do invariante no comportamento do fluído e do escoamento, sendo de grande importância o estudo deste fator antes de proceder com as simulações numéricas de diferêntes fluídos. 

\section{Cavidade com Tampa Deslizante}

Para o escoamento da cavidade com tampa deslizante, se construiu, separadamente, os perfis do escoamento para $Re = 100$, de forma a elucidar a análise feita para os diferentes números de Reynolds. Após isto, apresenta-se os gráficos para os demais números de Reynolds. O perfil de velocidade no centro geométrico da cavidade é representado pelas figuras \ref{lid-diven-100-perfil} a \ref{lid-diven-3200-perfil}.

\begin{figure}[ht]
    \centering
    \subfloat[Perfil de velocidade u em x = 0.5 para Re = 100.]{%
        \includegraphics[width=0.45\textwidth]{imagens/lid_driven_Re100_u.png}
    }
    \hspace{0.05\textwidth} % espaço entre as figuras
    \subfloat[Perfil de velocidade v em y = 0.5 para Re = 100.]{%
        \includegraphics[width=0.45\textwidth]{imagens/lid_driven_Re100_v.png}
    }
    \caption{Perfil de velocidade u  e v no centro geométrico da cavidade para Reynolds 100.}
    \label{lid-diven-100-perfil}
    {\footnotesize Fonte: Do próprio autor}
\end{figure}

\begin{figure}[ht]
    \centering
    \subfloat[Perfil de velocidade u em x = 0.5 para Re = 400.]{%
        \includegraphics[width=0.45\textwidth]{imagens/lid_driven_Re400_u.png}
    }
    \hspace{0.05\textwidth} % espaço entre as figuras
    \subfloat[Perfil de velocidade v em y = 0.5 para Re = 400.]{%
        \includegraphics[width=0.45\textwidth]{imagens/lid_driven_Re100_v.png}
    }
    \caption{Perfil de velocidade u  e v no centro geométrico da cavidade para Reynolds 400.}
    \label{lid-diven-400-perfil}
    {\footnotesize Fonte: Do próprio autor}
\end{figure}


\begin{figure}[ht]
    \centering
    \subfloat[Perfil de velocidade u em x = 0.5 para Re = 1000.]{%
        \includegraphics[width=0.45\textwidth]{imagens/lid_driven_Re1000_u.png}
    }
    \hspace{0.05\textwidth} % espaço entre as figuras
    \subfloat[Perfil de velocidade v em y = 0.5 para Re = 1000.]{%
        \includegraphics[width=0.45\textwidth]{imagens/lid_driven_Re1000_v.png}
    }
    \caption{Perfil de velocidade u  e v no centro geométrico da cavidade para Reynolds 1000.}
    \label{lid-diven-1000-perfil}
    {\footnotesize Fonte: Do próprio autor}
\end{figure}


\begin{figure}[ht]
    \centering
    \subfloat[Perfil de velocidade u em x = 0.5 para Re = 3200.]{%
        \includegraphics[width=0.45\textwidth]{imagens/lid_driven_Re3200_u.png}
    }
    \hspace{0.05\textwidth}
    \subfloat[Perfil de velocidade v em y = 0.5 para Re = 3200.]{%
        \includegraphics[width=0.45\textwidth]{imagens/lid_driven_Re3200_v.png}
    }
    \caption{Perfil de velocidade u  e v no centro geométrico da cavidade para Reynolds 3200.}\label{lid-diven-3200-perfil}
    {\footnotesize Fonte: Do próprio autor}
\end{figure}

Para valores de $Re = 100$, observa-se uma compatibilidade grande entre os resultados obtidos e aqueles apresentados por \citet{Ghia1982}, com discrepâncias em regiões com grande gradiente de velocidade. Para um valor de $Re = 400$, tem-se que os pontos seguem compativeis com a referência. A complexidade do escoamento aumenta com o aumento do número de Reynolds, de forma que as diferenças entre os perfis de velocidade, principalmente nas extremidades, começa a ficar mais acentuada.

Para perfis de $Re = 1000$ e $Re = 3200$ verifica-se um aumento das diferenças entre as simulações, principalmente nas extremidades. Tais discrepâncias ocorre pelo aumento da complexidade das recirculações e do movimento do fluido proximo as paredes do escoamento. Além disto, os métodos e a precisão utilizada no presente trabalho, são diferentes dos utilizados por \citet{Ghia1982}, dado a diferença de época dos trabalhos.

Para analisar os resultados obtidos, foram gerados gráficos com as linhas de fluxo para o escoamento com número de Reynolds de 3200, este valor foi escolhido por apresentar maior número de recirculações primárias e secundárias. Na Figura \ref{lid-diven-3200-streamline} se ilustra os resultados obtidos com o código desenvolvido em comparação com os trazidos por \citet{Ghia1982}.

\begin{figure}[ht]
    \centering
    \subfloat[Linhas de fluxo para Re = 3200. Próprio autor]{%
        \includegraphics[width=0.50\textwidth]{imagens/lid_driven_Re3200_streamLine_menor.png}
    }
    \hspace{0.05\textwidth}
    \subfloat[Linhas de fluxo para Re = 3200. \citet{Ghia1982}]{%
        \includegraphics[width=0.40\textwidth]{imagens/ghia-3200.png}
    }
    \caption{Linhas de fluxo para $Re = 3200$.}\label{lid-diven-3200-streamline}
    {\footnotesize Fonte: Do próprio autor} 
\end{figure}

Observa-se, nas duas figuras, uma boa concordância quanto à presença e à localização da recirculação primária, que ocupa a região central da cavidade, com seu núcleo posicionado ligeiramente abaixo da tampa deslizante. Essa recirculação é bem definida, com linhas de corrente em espiral convergindo para um ponto de estagnação central, o que é esperado para este escoamento devido o movimento promovido pela fronteira superior. Os resultados monstram a precisão dos métodos empregados.

Além disso, as figuras mostram recirculações secundárias nos cantos da cavidade, especialmente nos cantos inferiores esquerdo e direito e no canto superior esquerdo, que são características do escoamento com tampa deslizante para Reynolds elevados. Essas recirculações menores indicam a complexidade da estrutura de recirculação que surge devido às interações entre a tampa em movimento e as paredes laterais da cavidade quando se aumenta o numéro de Reynolds e, consequentemente, as instabilidades do escoamento. Pequenas variações no tamanho e na intensidade dessas recirculações entre as figuras podem ser atribuídas a diferenças na discretização espacial e na resolução numérica empregadas, visto que os métodos utilizados por \citet{Ghia1982} são bem diferentes que os atuais.

\section{Cavidade Diferencialmente Aquecida}

Para avaliar o comportamento da cavidade diferencialmente aquecida, foram gerados resultados com valores de $Ra$ variando de $10^3$ a $10^6$. Estes resultados foram comparados com os de \citet{Santos2022}, nas figuras \ref{isotermas - autor - p1}  e \ref{isotermas - autor - p2}, mostra-se as isotermas obtidas com o código do presente trabalho, enquanto a Figura \ref{isotermas - referencias} mostra as isotermas apresentadas por \citet{Santos2022}.

\begin{figure}[ht]
    \centering
    \subfloat[]{
        \includegraphics[width=0.42\textwidth]{imagens/thermal_cavity_Ra1e3_isotherma.png}
    }
    \hspace{0.05\textwidth}
    \subfloat[]{
        \includegraphics[width=0.42\textwidth]{imagens/thermal_cavity_Ra1e4_isotherma.png}
    }
    \caption{Isotermas para números de Rayleigh  de $10^3$ a $10^4$}
    \label{isotermas - autor - p1}
    {\footnotesize Fonte: Do próprio autor} 
\end{figure}


\begin{figure}[ht]
    \centering
    \subfloat[]{
        \includegraphics[width=0.42\textwidth]{imagens/thermal_cavity_Ra1e5_isotherma.png}
    }
    \hspace{0.05\textwidth}
    \subfloat[]{
        \includegraphics[width=0.42\textwidth]{imagens/thermal_cavity_Ra1e6_isotherma.png}
    }
    \caption{Isotermas para números de Rayleigh de $10^5$ e $10^6$}
    \label{isotermas - autor - p2}
    {\footnotesize Fonte: Do próprio autor} 
\end{figure}



\begin{figure}[ht]
    \centering
    \includegraphics[width=0.7\textwidth]{imagens/isotermas-referencias.png}
    \caption{Isotermas apresentadas por \citet{Santos2022} Da esquerda para a direita o Rayleigh varia de $10^3$ a $10^6$ em ordem crescente da esquerda para direita.}
    \label{isotermas - referencias}
    {\footnotesize Fonte: \citet{Santos2022}} 
\end{figure}

Para $Ra = 10^3$, ambos os resultados exibem isotermas com pequenas curvaturas, o que caracteriza um regime predominantemente difusivo, de condução térmica. Os resultados apresentados são similares e indicam a baixa influência dos efeitos advectivos que se obtem com baixos valores de Rayleigh.

Para valores de Rayleigh variando de $10^4$ a $10^5$, as isotermas começam a se curvar com relação ao centro da cavidade térmica, exibindo um grandiente de temperatura mais acentuado, provocado pela maior agitação do fluido. É evidente que os efeitos advectivos são mais acentuados para estes regimes de escoamento. A curvatura das isotermas também tende a indicar a presença de recirculações na região.

Para um valor de Rayleigh igual a $10^6$, o escoamento é predominantemente advectivo, as isotermas se alinham horizontalmente. As isotermas obtidas em ambos os trabalhos apresentam semelhanças qualitativas, de forma que transmitem as mesmas informações sobre este regime de escoamento 



Foram geradas as linhas de fluxo para regimes de escoamento com $Ra = 10^4$ e $Ra = 10^5$. Os resultados mostram o comportamento do escoamento para as diferentes condições. Os resultados foram comparados com o \citet{Gangawane2015}.A Figura \ref{streamline-thermal-autor} mostra as isotermas do autor enquanto a Figura \ref{streamline-thermal-referencia} as de \citet{Gangawane2015}.

\begin{figure}[H]
    \centering
    \subfloat[]{
        \includegraphics[width=0.45\textwidth]{imagens/Thermal_cavity_Ra1e4_streamLine.png}
    }
    \hspace{0.05\textwidth}
    \subfloat[]{
        \includegraphics[width=0.45\textwidth]{imagens/Thermal_cavity_Ra1e5_streamLine.png}
    }
    \caption{Linhas de Fluxo para valores de Rayleigh $10^4$, a esquerda, e $10^5$, a direita.}\label{streamline-thermal-autor}
    {\footnotesize Fonte: Do próprio autor} 
\end{figure}

\begin{figure}[H]
    \centering
    \includegraphics[width=0.7\textwidth]{imagens/linhas_de_fluxo_referencia.png}
    \caption{Linhas de Fluxo para valores de Rayleigh $10^4$, a esquerda, e $10^5$, a direita. \citet{Gangawane2015}.}
    \label{streamline-thermal-referencia}
    {\footnotesize Fonte: \citet{Gangawane2015}} 
\end{figure}

As imagens evidenciam que os escoamentos foram feitos por diferentes referenciais entre o presente trabalho e \citet{Gangawane2015}, o que explica a rotação das imagens. Para $Ra = 10^4$, as duas simulações mostram uma recirculação central, com uma certa simetria na cavidade. Os escoamentos estão qualitativamente coerêntes, de forma que a imagem trazida pelo presente trabalho, trás um número maior de informações.

Para o escoamento com $Ra = 10^5$, os efeitos advectivos se tornam mais intensos, de forma que recirculações mais complexas aparecem no centro da cavidade. Os resultados mostrados não são totalmente simétricos, o que pode indicar uma perturbação numérica no mesmo. Considerando a rotação dos eixos, devido ao eixo de coordenadas diferentes entre os autores, os resultados estão qualitativamente coerentes. Os resultados mostram a capacidade do código desenvolvido em captar padrões de recirculação complexos no interior da cavidade diferencialmente aquecida.





%====================================================================
\chapter{Conclusão}

Apresentou-se neste trabalho uma descrição detalhada dos modelos e métodos utilizados para o desenvolvimento das rotinas computacionais utilizadas. Foram abordados diferentes tipos de escoamentos incompressiveis bidimensionais, incluindo newtonianos isotérmicos e não isotérmicos além de escoamentos Não-Newtonianos. Simulou-se diferentes tipos de escoamentos consagrados na literatura, como o escoamento de Poiseulle, de Couette, da cavidade diferencialmente aquecida e da Cavidade com tampa deslizante, que serviram para a validação das rotinas implementadas. Comparações dos resultados obtidos com os de diversos autores foram utilizadas a fim de validar e verificar a coerência dos métodos implementados. 

Entre os escoamentos, variou-se parâmetros adimensionais, de forma a validar diferentes regimes de escoamento e o comportamento físico e numérico da simulação em cada um deles. Foram utilizadas técnicas como a malha deslocada e o passo fracionado, os quais foram detalhados durante o trabalho, para que se conseguisse a precisão adequada para os escoamentos.

Os resultados obtidos com os escoamentos clássicos mostraram um grau de compatibilidade com a literatura consoante com os métodos utilizados nas rotinas implementadas, de tal forma que foi possível verificar a validade da implementação computacional.

Um importante resultado obtido, foi a descrição do comportamento dos diferentes modelos de invariantes, de acordo com o regime de escoamento simulado e com o tipo de fluido não-newtoniano que cada um dos modelos estava representando. De forma que o modelo apresentado por \citet{Bird1987} lida melhor com fluidos newtonianos pseudoplasticos, e o modelo apresentado por \cite{Aristeu2020} lida melhor com fluidos dilatantes. 

Esta condição, apesar de se mostrar verdadeira no escoamento de Poiseuille, deve ser verificada em trabalhos futuros a fim de validar sua veracidade, uma vez que o escoamento simulado é de baixa complexidade e seus resultados podem não transmitir o comportamento dos modelos a niveis industriais.

Tem-se que o trabalho desenvolvido foi de essencial importância para o aprofundamento no entendimento a respeito da natureza e dos desafios dos métodos numéricos e computacionais utilizados na solução de escoamentos de diferentes naturezas.


% Referências bibliográficas
%====================================================================
\newpage
\def\thispagestyle#1{}
%\bibliographystyle{babplain-fl}
\bibliographystyle{apalikes}
\bibliography{bibliografia}
%====================================================================
% Apêndices da monografia
% Pode ser que o trabalho tenha vários apêndices. Coloque os 
% apêndices em uma pasta e dentro da pasta crie o arquivo tex com o 
% texto associado ao apêndice. 
%====================================================================
\begin{appendices}
%===========================================================equation=========
% Texto referente ao primeiro apêndice  
%====================================================================
\chapter{Solução Analítica para Escoamentos de Poiseuille}

O escoamento de Poiseuille, com suas simplificações, permite que encontremos soluções analíticas para o regime permanente tanto para fluidos newtonianos quanto para fluidos não-newtonianos utilizando a Lei de Potência. De forma que este problema se torna útil para a validação dos modelos de escoamento não-newtoniano e para o tratamento numérico utilizado. Sendo então, pertinente a apresentação matemática de sua formulação analítica. 

\section{Poiseuille Newtoniano}

Considerando um escoamento em regime permanente, com apenas a componente $u$ de velocidade diferente de zero e um gradiente de velocidade e partindo da Equação \ref{momentum-completa}, tem-se:

\begin{align}
    &\ 0\ =-\vec{\nabla}p+\vec{\nabla}\cdot\eta\left(\dot{\gamma}\right)\dot{\gamma} \nonumber \\
    &\vec{\nabla}p=\vec{\nabla}\cdot\eta\left(\dot{\gamma}\right)\dot{\gamma}
\end{align}


Considerando $\eta\left(\dot{\gamma}\right)\ =\ \mu$:

\begin{align}
    &\frac{\partial p}{\partial x}=\mu\frac{\partial^2u}{\partial y^2} \nonumber \\
    &\frac{1}{\mu}\frac{\partial p}{\partial x}\ =-\lambda \nonumber \\
    &\frac{\partial^2u}{\partial y^2}\ =-\lambda
\end{align}

Considerando as condições de contorno, $u(0) = 0$ e $ u(h) = 0$, e realizando as devidas integrações tem-se:

\begin{align}
    &\frac{\partial\ u}{\partial y}=-\lambda y+c_1 \nonumber \\
    &u(y)=-\frac{\lambda y^2}{2}+c_1y+c_2 \nonumber \\
    &u\left(0\right)=-\frac{\lambda0^2}{2}+c_10+c_{2\ }=c_2=0 \nonumber \\
    &u\left(h\right)=-\frac{\lambda h^2}{2}+c_1h=\ 0 \nonumber \\
    &c_1=\ \frac{\lambda h}{2} \nonumber \\
    &u\left(y\right)=-\frac{\lambda y^2}{2}+\frac{\lambda h}{2}y \nonumber \\
    &u(y)=\frac{\lambda y}{2}(h-y) \nonumber \\
\end{align}

Para este escoamento, a velocidade máxima atingida pelo fluído ocorre em $y = frac{h}{2}$, que é o meio da cavidade para o sistema de coordenadas adotado. Outro valor importante é a velocidade média do escoamento, que deve ser igual à velocidade uniforme de velocidade de entrada. tem-se então que:

\begin{align}
    &u\left(\frac{h}{2}\right)=\frac{\lambda h^2}{8}-\frac{\lambda h^2}{4}=v_{max}\ =\ -\frac{\lambda h^2}{8} \nonumber \\
    &u_{medio}\ =\ \int_{0}^{h}{u(y)\ dy\ }\ =\int_{0}^{h}\left(\frac{\lambda y^2}{2}-\frac{\lambda h}{2}y\right)dy\  \nonumber \\
    &u_{medio}\ =\ \left(\frac{\lambda y^3}{6}-\frac{\lambda h}{4}y^2\right)_0^h\ =\ \frac{\lambda h^3}{6}-\frac{\lambda h^3}{4}\ =\ -\frac{\lambda h^3}{12}
\end{align}

Estas equações resolvem bem e completamente o escoamento de Poiseuille newtoniano. 

\section{Poiseuille Não-Newtoniano}

Para fluidos não-newtonianos tem-se que $\eta\left(\dot{\gamma}\right)\ =\ m\dot{\gamma}^{n-1}$, considerando as mesmas simplificações feitas para o escoamento newtoniano tem-se que os dois perfis de invariantes apresentados no presente trabalho levam ao mesmo resultado:

\begin{align}
    &\frac{\partial p}{\partial x}=\frac{\partial}{\partial y}\left(m{\frac{\partial u}{\partial y}}^{n-1}\frac{\partial u}{\partial y}\right)\nonumber \\
    &\frac{\partial p}{\partial x}=\frac{\partial}{\partial y}\left(m{\frac{\partial u}{\partial y}}^n\right) \nonumber \\
    &\frac{\partial p}{\partial x}=m\frac{\partial}{\partial y}\left({\frac{\partial u}{\partial y}}^n\right) \nonumber \\
    &\frac{1}{m}\frac{\partial p}{\partial x}\ =\ -\ \lambda
\end{align}

De forma que para esta formulação, é útil redefinir de forma temporária o eixo de coordenadas do domínio considerado. O domínio temporário é mostrado na figura \ref{poiseuille-luiz-temporario}:

\begin{figure}[H]
    \centering
    \includegraphics[width=0.65\textwidth]{imagens/poiseuille-luiz.png}
    \caption{Domínio físico temporário para a solução do escoamento de Poiseuille não-newtoniano.}
    \label{poiseuille-luiz-temporario}
    {\footnotesize Fonte: Adaptado de \citet{Vasconcellos2024}}
\end{figure}

De forma que neste novo sistema de coordenadas tem-se o domínio variando de $-h $ a $h$ e com velocidade máxima ocorrendo em $y = 0$. Assim, aplicando as devidas integrais e utilizando duas das novas condições, sendo elas $u(h) = 0$ e $\frac{\partial u(0)}{\partial y} = 0$, tem-se: 

\begin{align}
    &-\ \lambda y+c_1={\frac{\partial u}{\partial y}}^n \nonumber \\
    &\frac{\partial u}{\partial y}\ =\ \left(-\ \lambda y+c_1\right)^\frac{1}{n} \nonumber \\
    &\frac{\partial u}{\partial y}\ =\ \left(-\ \lambda y+c_1\right)^\frac{1}{n} \nonumber \\
    &\frac{\partial u(0)}{\partial y}\ =\ \left(-\ \lambda0+c_1\right)^\frac{1}{n}\ =0 \nonumber \\
    &\frac{\partial u\left(0\right)}{\partial y}=c_1\ =0 \nonumber \\
    &\frac{\partial u}{\partial y}=-\lambda^\frac{1}{n}y^\frac{1}{n} \nonumber \\
    &u\left(y\right)=-\frac{\lambda^\frac{1}{n}y^{\frac{1}{n}\ +1}}{\frac{1}{n}\ +1}+c_2=-n\frac{\lambda^\frac{1}{n}y^{\frac{1}{n}\ +1}}{n\ +1}+c_2 \nonumber \\
    &u\left(h\right)=-n\frac{\lambda^\frac{1}{n}h^{\frac{1}{n}\ +1}}{n\ +1}+c_2\ =\ 0 \nonumber \\
    &c_2\ =n\frac{\lambda^\frac{1}{n}h^{\frac{1}{n}\ +1}}{n\ +1}\  \nonumber \\
    &u\left(y\right)=\ -\ n\frac{\lambda^\frac{1}{n}y^{\frac{1}{n}\ +1}}{n\ +1}+n\frac{\lambda^\frac{1}{n}h^{\frac{1}{n}\ +1}}{n\ +1} \nonumber 
\end{align}

É possível calcular o valor da velocidade máxima deste escoamento considerando $u = 0$:

\begin{equation}
    u\left(0\right)=\ n\frac{\lambda^\frac{1}{n}h^{\frac{1}{n}\ +1}}{n\ +1}\ =v_{max}
\end{equation}

Pode-se então escrever a equação que descreve o perfil de velocidade em função desta velocidade máxima:

\begin{equation}
    u\left(y\right)=v_{max}\left(1-\left(\frac{y}{h}\right)^{\frac{1}{n}+1}\right)
\end{equation}


Como se considerou a condição de simetria no meio do escoamento, o perfil de velocidade obtido pela equação pode gerar contornos irreais devido ao sinal negativo de $y$, para valores inferiores à metade da cavidade. Isto se da pelo expoênte fracionário presente na equação. Para solucionar este problema, se aplicou o módulo sobre a variavel dependente da equação obtida com o desenvolvimento apresentado. tem-se então que:

\begin{equation}
    u\left(y\right)=v_{max}\left(1-\left(\frac{|y|}{h}\right)^{\frac{1}{n}+1}\right)
\end{equation}

Para que se possa utilizar a equação no domínio considerado no presente trabalho, se utilizou de um deslocamento de $\frac{h}{2}$ da equação acima em y, de forma que o perfil de velocidade obtido esteja dentro do domínio indo de $y = 0$ a $y = h$. Finalmente a equação obtida: 

\begin{equation}
    u\left(y\right)=v_{max}\left(1-\left(\frac{|y- \frac{h}{2}|}{h}\right)^{\frac{1}{n}+1}\right)
\end{equation}.

%====================================================================
% Texto referente ao segunda apêndice  
%====================================================================






\chapter{Código Desenvolvido}


Apresenta-se, em sua totalidade, o código desenvolvido:


\begin{lstlisting}[caption={Código desenvolvido por Luís Eduardo Silva Borges - 2024}, label={lst:codigo-c}]

//****************************************************************
//      Código desenvolvido Na Universidade Federal de Uberlândia
//      Luís Eduardo Silva Borges
//      MFLAB - 2024
//****************************************************************


    int main(){
        double *u; //velocidade em x
        double *us; // u* ustar, a velocidade estimada u com a pressão no passo de tempo anterior
        double *v; //velocidade em y
        double *vs; // v* vstar, a velocidade estimada v com a pressão no passo de tempo anterior 
        double *p; //campo de pressão
        double *pl; //p' é a correção de p 
        double *T; // temperatura
        int nosx,nosy; //quantidade de nós nas direções x e y
        int nosx_u,nosy_u; // nos utilizados em u
        int nosx_v,nosy_v; // nos utilizados em u 
        int nosx_p,nosy_p; // nos utilizados em u 
        int nost; //nós de tempo
        double dt,dx,dy; // passos de tempo e de espaço
        double lx,ly,lt; //altura, largura e tempo final
        double erro_u = 1 ; //erro em relação à iteração anterior de u 
        double erro_v = 1 ; //erro em relação à iteração anterior de v
        double erro_p = 1 ; //erro em relação à iteração anterior de p
        double erro_T = 1 ; //erro em relação à iteração anterior de T
        double *residual_T; // para os gradientes conjulgados
        double *residual_p;
        double *residual_u;
        double *residual_v;
        double *p_p;
        double *p_u;
        double *p_v;
        double *p_t; // para os para os gradientes conjulgados
        double xalpha; // para os para os gradientes conjulgados
        double xbetha; // para os para os gradientes conjulgados
        double tol = 1E-9; // tolerância do método iterativo
        double v_ref = 1.0*0; // velocidade da tampa da cavidade
        double ni =  0.04; // viscosidade cinemática do fluido
        double rho = 1; // densidade do fluido
        int iteracoes = 0;
        double omega = 0.8 ; // fator de amortecimento para convergencia
        int count = 0; // variavel para contar as iterações;  
        double alpha = 0.0071; //difusividade termica
        double g = 9.81; // gravidade
        double betha = 0.00013 ; // coeficiente de expansão volumétrica
        double Tq = 80,Tf = 50; //temperatura quente e fria
        double Tref = (Tq+Tf)/2; // temperatura de referencia da convecção
        double Pr = ni/alpha; //numero de prandtl
        double Ra = 1e3;
        double *eta; // Viscosidade não-newtoniana
        double *gamax; // Fator de correção de u 
        double *gamay; // fator de correção de v
        double m,n = 1; // coeficientes da lei de potência
        int useTermal = 1.0;
        clock_t start, end;
        double cpu_time_used;
        double v_medio = 0;
        m = 0.01;
        start = clock();
    
        nosx = 33; //125
        nosy = 33;
        nost = 1600;
    
        lx = 1;
        ly = 1;
        lt = 120;
        dx = lx/(nosx);
        dy = ly/(nosy);
        dt = lt/(nost);
        
        betha = (Ra*ni*alpha)/(g*(Tq-Tf)*ly*ly*ly)*useTermal;
    
        printf("Prandtl = %f",Pr);
        printf("\n Raylgh = %f\n ",Ra);
    
        nosx_u = nosx+3;
        nosy_u = nosy+2;
    
        nosx_v = nosx+2;
        nosy_v = nosy+3;
    
        nosx_p = nosx+2;
        nosy_p = nosy+2;
    
        u = malloc((nosx_u)*(nosy_u)*sizeof(double));
        residual_u = malloc((nosx_u)*(nosy_u)*sizeof(double));
        p_u = malloc((nosx_u)*(nosy_u)*sizeof(double));
        us = malloc((nosx_u)*(nosy_u)*sizeof(double));
    
    
        cond_iniciais_u(u,(nosx_u),(nosy_u),v_ref,v_medio);
        cond_iniciais_u(us,(nosx_u),(nosy_u),v_ref,v_medio);
        cond_iniciais_conjul(p_u,nosx_u*nosy_u);
        cond_iniciais_conjul(residual_u,nosx_u*nosy_u);
    
        v = malloc((nosx_v)*(nosy_v)*sizeof(double));
        vs = malloc((nosx_v)*(nosy_v)*sizeof(double));
        residual_v = malloc((nosx_v)*(nosy_v)*sizeof(double));
        p_v = malloc((nosx_v)*(nosy_v)*sizeof(double));
    
        cond_iniciais_v(v,nosx_v,nosy_v);
        cond_iniciais_v(vs,nosx_v,nosy_v);
        cond_iniciais_conjul(p_v,nosx_v*nosy_v);
        cond_iniciais_conjul(residual_v,nosx_v*nosy_v);
    
        p = malloc((nosx_p)*(nosy_p)*sizeof(double));
        pl = malloc((nosx_p)*(nosy_p)*sizeof(double));
        residual_p = malloc((nosx_p)*(nosy_p)*sizeof(double));
        p_p = malloc((nosx_p)*(nosy_p)*sizeof(double));
    
        cond_iniciais_p(p,nosx_p,nosy_p);
        cond_iniciais_p(pl,nosx_p,nosy_p);
        cond_iniciais_conjul(residual_p,nosx_p*nosy_p);
        cond_iniciais_conjul(p_p,nosx_p*nosy_p);
    
        T = malloc((nosx_p)*(nosy_p)*sizeof(double));
        residual_T = malloc((nosx_p)*(nosy_p)*sizeof(double));
        p_t = malloc((nosx_p)*(nosy_p)*sizeof(double));
        
        cond_iniciais_T(T,nosx_p,nosy_p,Tq,Tf,Tref);
        cond_iniciais_conjul(residual_T,nosx_p*nosy_p);
        cond_iniciais_conjul(p_t,nosx_p*nosy_p);
        
    
        eta = malloc((nosx_p)*(nosy_p)*sizeof(double));
        gamax = malloc((nosx_p)*(nosy_p)*sizeof(double));
        gamay = malloc((nosx_p)*(nosy_p)*sizeof(double));
    
        saveXY(nosy_p, nosx_p, dx, dy);
        
        for(int k=1; k<nost;k++)
        {
        /*campo de eta*/
    
        for(int i = 1;i<nosy_p-1;i++)
            {
                for(int j =1; j<nosx_p -1;j++)
                {
                    double ux = (u[j+i*nosx_u+1]-u[j+i*nosx_u])/dx; //influencia de u* em p por diferenças finitas
                    double vy = (v[j+(i+1)*nosx_v]-v[j+i*nosx_v])/dy; //influencia de v* em p por diferenças finitas
                    double uy = ((u[j+(i+1)*nosx_u+1]+u[j+(i+1)*nosx_u])   -  (u[j+(i-1)*nosx_u+1]+us[j+(i-1)*nosx_u]))/(4*dy);
                    double vx = ((v[j+(i+1)*nosx_v +1]+v[j+i*nosx_v +1]) - (v[j+(i+1)*nosx_v-1]+vs[j+i*nosx_v-1]))/(4*dx);
                    
                    double invariante = abs(4*(ux*vy) - (uy+vx)*(uy+vx));
                    //double invariante =2*ux*ux + 2*uy*uy + (uy+vx)*(uy+vx);
                    double gama = sqrt(invariante);
                    if(gama == 0 )
                    {
                        
                        eta[j+i*nosx_p] = m;
                    }
                    else
                    {
                        eta[j+i*nosx_p] = m*pow(gama,n-1);
                    }
                }               
            }
            
        /***************************** */
        
            /*campo de gamax e gamay*/
            for(int i = 1;i<nosx_p-1;i++)
            {
                eta[i] = 2*(eta[i+nosx_p] - eta[i+nosx_p+nosx_p]) + eta[i+nosx_p+nosx_p];//superior
                eta[nosx_p*(nosy_p-1) + i] =  2*(eta[nosx_p*(nosy_p-1) + i - nosx_p] - eta[nosx_p*(nosy_p-1) + i - nosx_p - nosx_p]) + eta[nosx_p*(nosy_p-1) + i - nosx_p - nosx_p]; //inferior 
            }
            /******************************* */
            /*ghost cells esquerda e direita*/
            for(int i = 1;i<nosy_p-1;i++)
            {
                eta[i*nosx_p] = 2*(eta[i*nosx_p+1] - eta[i*nosx_p+2]) + eta[i*nosx_p+2]; //esquerda
                eta[(i+1)*nosx_p-1]= 2*(eta[((i+1)*nosx_p-1)-1] - eta[((i+1)*nosx_p-1)-2])+eta[((i+1)*nosx_p-1)-2]; //direita
            }
            for(int i = 1;i<nosy_p-1;i++)
                {
                    for(int j =1; j<nosx_p -1;j++)
                    {
                        double ux = (u[j+i*nosx_u+1]-u[j+i*nosx_u])/dx; //influencia de u* em p por diferenças finitas
                        double vy = (v[j+(i+1)*nosx_v]-v[j+i*nosx_v])/dy; //influencia de v* em p por diferenças finitas
                        double uy = ((u[j+(i+1)*nosx_u+1]+u[j+(i+1)*nosx_u])   -  (u[j+(i-1)*nosx_u+1]+u[j+(i-1)*nosx_u]))/(4*dx);
                        double vx = ((v[j+(i+1)*nosx_v +1]+v[j+i*nosx_v +1]) - (v[j+(i+1)*nosx_v-1]+v[j+i*nosx_v-1]))/(4*dy);
                        
                        double etax = (eta[j+i*nosx_p+1] - eta[j+i*nosx_p-1])/(2*dx);
                        double etay = (eta[j+(i+1)*nosx_p] - eta[j+(i-1)*nosx_p])/(2*dy);
    
                        
    
                        gamax[j+i*nosx_p] = (etax*(2*ux) + etay*(uy+vx));
                        gamay[j+i*nosx_p] = (etay*(2*vy) + etax*(uy+vx));
                    }               
                }
    
    
        /***************************** */
            erro_u=1;
            while(erro_u>tol)
            {
                for(int i = 1;i<nosy_u-1;i++)
            {
                us[i*nosx_u+1] = v_medio +  0*us[i*nosx_u+2]; // esquerda
                us[(i+1)*nosx_u-2] = 0*us[(i+1)*nosx_u-2-1]; // direita
            }
            /*****************************/
    
            /*ghost cells esquerda e direita*/
            for(int i = 1;i<nosy_u-1;i++)
            {
                us[i*nosx_u] = v_medio + 0*us[i*nosx_u+1]; //esquerda
                us[(i+1)*nosx_u-1] = 0*us[(i+1)*nosx_u-1-2]; //direita
            }
            /*****************************/
            /*ghost cells baixo e cima****/
            for(int i = 1;i<nosx_u-1;i++)
            {
                us[i] = 2*v_ref - us[i + nosx_u]; // superior
                us[(nosx_u)*(nosy_u-1)+i] = -us[(nosx_u)*(nosy_u-2)+i];//inferior
            }
                /*****************************/
                /**********''***meio***********/
                for(int i = 1;i<nosy_u-1;i++)
                {
                    for(int j = 2;j<nosx_u-2;j++)
                    {
                    double etaij = (eta[j+i*nosx_p]+eta[j+i*nosx_p-1])/2;
                    double gamaxij = (gamax[j+i*nosx_p]+gamax[j+i*nosx_p-1]/2);
    
                    double a1 = 1 + 2*dt*etaij/(dx*dx)+2*dt*etaij/(dy*dy); //coeficiente que acompanha uij
                        
                    double difusivo = 
                    (etaij*dt/(dx*dx))*(us[j+i*nosx_u+1] + us[j+i*nosx_u-1]) + 
                    (etaij*dt/(dy*dy))*(us[j+(i+1)*nosx_u] + us[j+(i-1)*nosx_u]); // termo difusivo por diferenças centradas implicito
                    
                    
                    double vij = (v[j+i*nosx_v] + v[j+i*nosx_v-1]+ v[j+(i+1)*nosx_v-1]+ v[j+(i+1)*nosx_v])/4;
    
                    double Pex = u[j+i*nosx_u]*dx/etaij;
                    double Pey = vij*dy/etaij;
    
                    double advec;
                    double advecx;
                    double advecy;
    
                    advecx =  (us[j+i*nosx_u+1] - us[j+i*nosx_u-1])/2.0;
                    advecy =  (us[j+(i+1)*nosx_u] - us[j+(i-1)*nosx_u])/2.0;
    
                    advec = 
                    dt*u[j+i*nosx_u]*advecx/dx + dt*vij*advecy/dy;
    
                    double fonte = u[j+i*nosx_u] - (dt/(rho*dx))*(p[j+i*nosx_p]-p[j+i*nosx_p-1]) + gamaxij*0;
                    
                    us[j+i*nosx_u] = (
                            fonte
                        + advec
                        + difusivo
                        )/a1;
                    }
                }
                /*****************************/
                iteracoes = iteracoes +1;
                if(iteracoes == 10000){
                    break;
                }
                erro_u = calcula_erro(us,residual_u,nosx_u*nosy_u);
                for(int i = 0;i<(nosx_u)*(nosy_u);i++)
                {
                    residual_u[i] = us[i];
                }
            }
            
            printf("passo : %d iteracao de u :%d\n",k,iteracoes);
    
    
    
            erro_v = 1;
            while(erro_v >tol){  
                /********* solver de v* ******/
                /*parede esquerda e direita ghost cells*/
                for(int i = 1;i<nosy_v-1;i++)
                {
                    vs[i*nosx_v] =  -vs[i*nosx_v+1]; //esquerda
                    vs[(i+1)*nosx_v-1] = -vs[((i+1)*nosx_v-1)-1]; //direita
                }
                /******************************/
                /*parede superior e inferior ghost cells*/
                for(int i = 1;i<nosx_v-1;i++)
                {
                    vs[i] = 0; //superior
                    vs[(nosy_v-1)*nosx_v + i] = 0; //inferior
                }
                /****************************/
                /*parede superior e inferior*/
                    for(int i = 1;i<nosx_v-1;i++)
                {
                    vs[nosx_v+i] = 0; //superior
                    vs[(nosy_v-2)*nosx_v + i] = 0; //inferior
                }
                /******************************/
                /************meio*************/
                for(int i = 2;i<nosy_v-2;i++)
                {
                    for(int j = 1;j<nosx_v -1;j++)
                    {
                    double etaij = (eta[j+i*nosx_p]+eta[j+(i-1)*nosx_p])/2;
                    double gamayij = (gamay[j+i*nosx_p]+gamay[j+(i-1)*nosx_p])/2;
    
                    double a1 = 1 + 2*etaij*dt/(dx*dx) + 2*etaij*dt/(dy*dy)   ;//coeficiente que acompanha uij
    
                    double difusivo = 
                    (etaij*dt/(dx*dx))*(vs[j+i*nosx_v+1]+vs[j+i*nosx_v-1]) + 
                    (etaij*dt/(dy*dy))*(vs[j+(i+1)*nosx_v] + vs[j+(i-1)*nosx_v]);// termo difusivo por diferenças centradas implicito
    
                    double uij = (u[j+i*nosx_u] + u[j+(i-1)*nosx_u] + u[j+(i-1)*nosx_u+1] + u[j+i*nosx_u+1])/4;
    
    
                    double Pex = uij*dx/etaij;
                    double Pey = v[j+i*nosx_v]*dy/etaij;
    
                    double advec;
                        double advecx = (vs[j+i*nosx_v+1] - vs[j+i*nosx_v-1])/(2);
                        double advecy = (vs[j+(i+1)*nosx_v] - vs[j+(i-1)*nosx_v])/(2);
                        advec = 
                        dt*uij*advecx/dx + dt*vs[j+i*nosx_v]*advecy/dy;
    
    
    
                    double Tij = (T[j+i*nosx_p]+T[j+(i-1)*nosx_p])/2;
    
                    double fonte = v[j+i*nosx_v] - (dt/(rho*dy))*(p[j+i*nosx_p]-p[j+(i-1)*nosx_p]) - g*betha*dt*(Tij - Tref) + gamayij*0;//termo de pressão e passo anterior
    
                    vs[j+i*nosx_v] = (fonte+advec+difusivo)/a1;
                    }
                }
                /*****************************/
                /*****************************/
                
                iteracoes = iteracoes +1;
                if(iteracoes == 10000){
                    break;
                }
                erro_v = calcula_erro(vs,residual_v,nosx_v*nosy_v);
                for(int i = 0;i<(nosx_v)*(nosy_v);i++)
                {
                    residual_v[i] = vs[i];
                }
            }
    
            printf("passo : %d iteracao de v :%d\n",k,iteracoes);
    
            erro_p = 0;
            for(int i = 1;i<nosy_p-1;i++)
            {
                for(int j =1; j<nosx_p -1;j++)
                {
                    double uij = (us[j+i*nosx_u+1]-us[j+i*nosx_u]); //influencia de u* em p por diferenças finitas
                    double vij = (vs[j+(i+1)*nosx_v]-vs[j+i*nosx_v]); //influencia de v* em p por diferenças finitas
    
                    double px = pl[j+i*nosx_p+1] + pl[j+i*nosx_p-1];
                    double py = pl[j+(i+1)*nosx_p] + pl[j+(i-1)*nosx_p];
    
    
                    double af = rho/dt; // termo que acompanha as derivadas
                    double a1 = 2/(dx*dx)+2/(dy*dy);
                    double fonte = af*(uij/dx+vij/dy);
    
                    residual_p[j+i*nosx_p] =  - fonte  - (pl[j+i*nosx_p]*a1-px/(dx*dx)-py/(dy*dy));
                    p_p[j+i*nosx_p] = residual_p[j+i*nosx_p];
                    erro_p = erro_p + residual_p[j+i*nosx_p]*residual_p[j+i*nosx_p];
                
                }               
            }
    
            erro_p = sqrt(erro_p);
            
            iteracoes = 0;
    
            while(erro_p>tol){
    
                for(int i = 1;i<nosx_p-1;i++)
                {
                    p_p[i] = p_p[i+nosx_p];//superior
                    p_p[nosx_p*(nosy_p-1) + i] =  p_p[nosx_p*(nosy_p-1) + i - nosx_p]; //inferior 
                }
    
                for(int i = 1;i<nosy_p-1;i++)
                {
                    p_p[i*nosx_p] = p_p[i*nosx_p+1]; //esquerda
                    p_p[(i+1)*nosx_p-1]= p_p[((i+1)*nosx_p-1)-1]; //direita
                }
    
                double numerador = 0;
                double numeradorBetha = 0;
                double denominador = 0; 
    
                for(int i = 1;i<nosy_p-1;i++)
                {
                    for(int j =1; j<nosx_p -1;j++)
                    {
                        double px = p_p[j+i*nosx_p+1] + p_p[j+i*nosx_p-1];
                        double py = p_p[j+(i+1)*nosx_p] + p_p[j+(i-1)*nosx_p];
    
    
                        double a1 = 2/(dx*dx)+2/(dy*dy);
    
                        denominador = denominador + (p_p[j+i*nosx_p]*a1-px/(dx*dx)-py/(dy*dy))*p_p[j+i*nosx_p];
                        
                        numerador = numerador + residual_p[j+i*nosx_p]*residual_p[j+i*nosx_p];
                    }               
                }
    
                xalpha = numerador/denominador;
                erro_p = 0;
    
                for(int i = 1;i<nosy_p-1;i++)
                {
                    for(int j =1; j<nosx_p -1;j++)
                    {
                        double px = p_p[j+i*nosx_p+1] + p_p[j+i*nosx_p-1];
                        double py = p_p[j+(i+1)*nosx_p] + p_p[j+(i-1)*nosx_p];
    
    
                        double a1 = 2/(dx*dx)+2/(dy*dy);
                        
                        pl[j+i*nosx_p] = pl[j+i*nosx_p] + xalpha*p_p[j+i*nosx_p];
    
                        residual_p[j+i*nosx_p] = residual_p[j+i*nosx_p] - xalpha*(p_p[j+i*nosx_p]*a1-px/(dx*dx)-py/(dy*dy));
                        
                        erro_p = erro_p + residual_p[j+i*nosx_p]*residual_p[j+i*nosx_p];
    
                        numeradorBetha = numeradorBetha +  residual_p[j+i*nosx_p]*residual_p[j+i*nosx_p];
                    }               
                }
    
                erro_p = sqrt(erro_p);
                xbetha = numeradorBetha/numerador;
    
                for(int i = 1;i<nosy_p-1;i++)
                {
                    for(int j = 1; j<nosx_p -1;j++)
                    {
                        p_p[j+i*nosx_p] = residual_p[j+i*nosx_p] +  xbetha*p_p[j+i*nosx_p];
                    }           
                }
    
                iteracoes = iteracoes +1;
                
                if(iteracoes == 10000)
                    break;
            }
            for(int i = 1;i<nosx_p-1;i++)
            {
                pl[i] = pl[i+nosx_p];//superior
                pl[nosx_p*(nosy_p-1) + i] =  pl[nosx_p*(nosy_p-1) + i - nosx_p]; //inferior 
            }
            /******************************* */
            /*ghost cells esquerda e direita*/
            for(int i = 1;i<nosy_p-1;i++)
            {
                pl[i*nosx_p] = pl[i*nosx_p+1]; //esquerda
                pl[(i+1)*nosx_p-1]= pl[((i+1)*nosx_p-1)-1]; //direita
            }
            
            printf("passo : %d iteracao de P :%d\n",k,iteracoes);
            iteracoes = 0;
            
            /*atualização para a pressão****/
            
            /*******************************/
            /*atualização para a velocidade*/
            /********atualização de u*******/
            for(int i = 1;i<nosy_u-1;i++)
            {
                for(int j = 1; j<nosx_u-1;j++){
                    u[j+i*nosx_u] = us[j+i*nosx_u] - (dt/(rho*dx))*(pl[j+i*nosx_p]-pl[j+i*nosx_p-1]);
                }
            }
            /*******************************/
            /********atualização de v*******/
            for(int i = 1;i<nosy_v-1;i++)
            {
                for(int j = 1; j<nosx_v-1;j++){
                    v[j+i*nosx_v] = vs[j+i*nosx_v] - (dt/(rho*dy))*(pl[j+i*nosx_p]-pl[j+(i-1)*nosx_p]) ;
                }
            }
            /*******************************/
            /*******************************/
            for(int i = 1;i<nosy_p-1;i++)
            {
                for(int j = 1; j<nosx_p-1;j++){
                    p[j+i*nosx_p] =p[j+i*nosx_p]+pl[j+i*nosx_p];
                }
            }
            erro_T = 0;
            for(int i = 1;i<nosy_p-1;i++)
                {
                    for(int j = 1; j<nosx_p -1;j++)
                    {
                        double a1 = (1 + 2.0*dt*alpha/(dx*dx) + 2.0*dt*alpha/(dy*dy));
    
                        double uij = (u[j+i*nosx_u+1]+u[j+i*nosx_u])/2.0;
                        double vij = (v[j+(i+1)*nosx_v]+v[j+i*nosx_v])/2.0;
                        double advec = 
                        uij*dt/(2*dx)*(T[j+i*nosx_p+1]-T[j+i*nosx_p-1])+
                        vij*dt/(2*dy)*(T[j+(i+1)*nosx_p]-T[j+(i-1)*nosx_p]);
    
                        double difusivo = 
                        alpha*dt*
                        ((T[j+i*nosx_p+1]+T[j+i*nosx_p-1])/(dx*dx) + 
                        (T[j+(i+1)*nosx_p]+T[j+(i-1)*nosx_p])/(dy*dy));
    
                        residual_T[j+i*nosx_p] = -(T[j+i*nosx_p]*a1-difusivo+advec - T[j+i*nosx_p]);
    
                        p_t[j+i*nosx_p] = residual_T[j+i*nosx_p];
                        erro_T = erro_T + residual_T[j+i*nosx_p]*residual_T[j+i*nosx_p];
                    }           
                }
                erro_T = sqrt(erro_T);
                iteracoes = 0;
            //***************************solver de T***************/
            while(erro_T>tol && useTermal){
                erro_T = 0;
                for(int i = 1;i<nosx_p-1;i++)
                {
                    p_t[i] = p_t[i+nosx_p];//superior
                    p_t[nosx_p*(nosy_p-1) + i] =  p_t[nosx_p*(nosy_p-1) + i - nosx_p]; //inferior 
                }
                for(int i = 1;i<nosy_p-1;i++)
                {
                    p_t[i*nosx_p] = -p_t[i*nosx_p+1]; //esquerda
                    p_t[(i+1)*nosx_p-1]= -p_t[((i+1)*nosx_p-1)-1]; //direita
                }
                double numerador = 0;
                double numeradorBetha = 0;
                double denominador = 0; 
    
                for(int i = 1;i<nosy_p-1;i++)
                {
                    for(int j = 1; j<nosx_p -1;j++)
                    {
                        double a1 = (1.0 + 2.0*dt*alpha/(dx*dx) + 2.0*dt*alpha/(dy*dy));
    
                        double uij = (u[j+i*nosx_u+1]+u[j+i*nosx_u])/2.0;
                        double vij = (v[j+(i+1)*nosx_v]+v[j+i*nosx_v])/2.0;
                        double advec = 
                        uij*dt/(2*dx)*(p_t[j+i*nosx_p+1]-p_t[j+i*nosx_p-1])+
                        vij*dt/(2*dy)*(p_t[j+(i+1)*nosx_p]-p_t[j+(i-1)*nosx_p]);
    
                        double difusivo = 
                        alpha*dt*
                        ((p_t[j+i*nosx_p+1]+p_t[j+i*nosx_p-1])/(dx*dx) + 
                        (p_t[j+(i+1)*nosx_p]+p_t[j+(i-1)*nosx_p])/(dy*dy));
    
    
    
                        denominador = denominador + (p_t[j+i*nosx_p]*a1-difusivo+advec)*p_t[j+i*nosx_p];
    
    
                        numerador = numerador + residual_T[j+i*nosx_p]*residual_T[j+i*nosx_p];
                    }           
                }
                xalpha = numerador/denominador;
    
            
                for(int i = 1;i<nosy_p-1;i++)
                {
                    for(int j = 1; j<nosx_p -1;j++)
                    {
                        double a1 = (1.0 + 2.0*dt*alpha/(dx*dx) + 2.0*dt*alpha/(dy*dy));
    
                        double uij = (u[j+i*nosx_u+1]+u[j+i*nosx_u])/2.0;
                        double vij = (v[j+(i+1)*nosx_v]+v[j+i*nosx_v])/2.0;
                        double advec = 
                        uij*dt/(2*dx)*(p_t[j+i*nosx_p+1]-p_t[j+i*nosx_p-1])+
                        vij*dt/(2*dy)*(p_t[j+(i+1)*nosx_p]-p_t[j+(i-1)*nosx_p]);
    
                        double difusivo = 
                        alpha*dt*
                        ((p_t[j+i*nosx_p+1]+p_t[j+i*nosx_p-1])/(dx*dx) + 
                        (p_t[j+(i+1)*nosx_p]+p_t[j+(i-1)*nosx_p])/(dy*dy));
    
                        T[j+i*nosx_p] = T[j+i*nosx_p] + xalpha*p_t[j+i*nosx_p];
    
                        residual_T[j+i*nosx_p] = residual_T[j+i*nosx_p] - xalpha*(p_t[j+i*nosx_p]*a1-difusivo+advec);
    
    
                        numeradorBetha = numeradorBetha +  residual_T[j+i*nosx_p]*residual_T[j+i*nosx_p];
                        erro_T = erro_T + residual_T[j+i*nosx_p]*residual_T[j+i*nosx_p];
                    }           
                }
    
                erro_T = sqrt(erro_T);
                xbetha = numeradorBetha/numerador;
                for(int i = 1;i<nosy_p-1;i++)
                {
                    for(int j = 1; j<nosx_p -1;j++)
                    {
                        p_t[j+i*nosx_p] = residual_T[j+i*nosx_p] +  xbetha*p_t[j+i*nosx_p];
                    }           
                }
    
                iteracoes = iteracoes +1;
                if(iteracoes == 1000)
                break;
            }
    
            for(int i = 1;i<nosx_p-1;i++)
            {
                T[i] = T[i+nosx_p];//superior
                T[nosx_p*(nosy_p-1) + i] =  T[nosx_p*(nosy_p-1) + i - nosx_p]; //inferior 
            }
            /******************************* */
            /*ghost cells esquerda e direita*/
            for(int i = 1;i<nosy_p-1;i++)
            {
                T[i*nosx_p] = 2.0*Tf - T[i*nosx_p+1]; //esquerda
                T[(i+1)*nosx_p-1]= 2.0*Tq -T[((i+1)*nosx_p-1)-1]; //direita
            }
            printf("passo : %d iteracao de T :%d\n",k,iteracoes);
        }
    
    
        end = clock();
        cpu_time_used = ((double) (end - start)) / CLOCKS_PER_SEC;
    
        saveToCSV(u,v, p,T,nosy_p,nosx_p,nosx_u,nosx_v);    
        free(u);
        free(us);
        free(v);
        free(vs);
        free(p);
        free(pl);
        free(T);
        free(residual_p);
        free(residual_T);
        free(residual_u);
        free(residual_v);
        free(p_t);
        free(p_p);
        free(p_u);
        free(p_v);
        free(eta);
        free(gamax);
        free(gamay);
        return 0;
    }
\end{lstlisting}
\end{appendices}

\end{document}
