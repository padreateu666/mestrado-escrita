\chapter{Conclusão}

Apresentou-se neste trabalho uma descrição detalhada dos modelos e métodos utilizados para o desenvolvimento das rotinas computacionais utilizadas. Foram abordados diferentes tipos de escoamentos incompressiveis bidimensionais, incluindo newtonianos isotérmicos e não isotérmicos além de escoamentos Não-Newtonianos. Simulou-se diferentes tipos de escoamentos consagrados na literatura, como o escoamento de Poiseulle, de Couette, da cavidade diferencialmente aquecida e da Cavidade com tampa deslizante, que serviram para a validação das rotinas implementadas. Comparações dos resultados obtidos com os de diversos autores foram utilizadas a fim de validar e verificar a coerência dos métodos implementados. 

Entre os escoamentos, variou-se parâmetros adimensionais, de forma a validar diferentes regimes de escoamento e o comportamento físico e numérico da simulação em cada um deles. Foram utilizadas técnicas como a malha deslocada e o passo fracionado, os quais foram detalhados durante o trabalho, para que se conseguisse a precisão adequada para os escoamentos.

Os resultados obtidos com os escoamentos clássicos mostraram um grau de compatibilidade com a literatura consoante com os métodos utilizados nas rotinas implementadas, de tal forma que foi possível verificar a validade da implementação computacional.

Um importante resultado obtido, foi a descrição do comportamento dos diferentes modelos de invariantes, de acordo com o regime de escoamento simulado e com o tipo de fluido não-newtoniano que cada um dos modelos estava representando. De forma que o modelo apresentado por \citet{Bird1987} lida melhor com fluidos newtonianos pseudoplasticos, e o modelo apresentado por \cite{Aristeu2020} lida melhor com fluidos dilatantes. 

Esta condição, apesar de se mostrar verdadeira no escoamento de Poiseuille, deve ser verificada em trabalhos futuros a fim de validar sua veracidade, uma vez que o escoamento simulado é de baixa complexidade e seus resultados podem não transmitir o comportamento dos modelos a niveis industriais.

Tem-se que o trabalho desenvolvido foi de essencial importância para o aprofundamento no entendimento a respeito da natureza e dos desafios dos métodos numéricos e computacionais utilizados na solução de escoamentos de diferentes naturezas.