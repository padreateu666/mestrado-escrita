%====================================================================
% Resultados: Escreva logo após o 
% \chapter{Resultados} o texto de seu trabalho referente 
% aos resultados obtidos no desenvolvimento de seu estudo. 
%====================================================================
\chapter{Modelos Computacionais}

Com base no modelo numérico, foram construídas rotinas computacionais para solucionar os escoamentos. O código foi desenvolvido em linguagem C, utilizando as bibliotecas padrões de entrada e saída de dados. O software desenvolvido está disponibilizado integralmente no apêndice B, nesta secção serão feitos comentários e explicações pontuais a respeito da implementação dos métodos, as condições de contorno e o valor das propriedades físicas utilizadas

As propriedades físicas do escoamento foram variadas a fim de garantir que os valores adimensionais sejam compatíveis com a literatura, para a comparação de resultados. 

\section{Couette Plano}

Para o escoamento de Couette, como já detalhado no modelo computacional e visível da Figura \ref{Couettefisico}. Tem-se as seguintes condições de contorno.

O valor de u na parede superior é igual a velocidade de referência, neste caso adotado como $1\ \left[\frac{m}{s}\right]$, com esta exceção, todos os valores de condições de contorno, respeitando a Tabela \ref{contorno - couette}, são iguais a zero.	
\begin{table}[ht]
    \caption{Condições de contorno para o escoamento de Couette}
    \vspace{0.1cm}
    \centering
    \begin{tabular}{c c c c }
    \hline
    {\bfseries Parede/Propriedade} & $ \bm{u}$ & $ \bm{v}$ & $ \bm{p}$ \\
    \hline
    Esquerda & 2ª espécie & 2ª espécie & 1ª espécie \\

    Direita & 2ª espécie & 2ª espécie & 1ª espécie \\

    Superior & 1ª espécie & 1ª espécie & 2ª espécie \\

    Inferior & 1ª espécie & 1ª espécie & 2ª espécie \\
    \hline
    \end{tabular}
    \label{contorno - couette}\\
    \vspace{0.4cm}
    {\footnotesize Fonte: Do próprio autor}
\end{table}


As propriedades $\eta$, $\nu$ e m foram variadas de forma que se obtivesse diferentes números de Reynolds ao longo do escoamento, tanto no caso newtoniano quanto no caso não-newtoniano. Os valores utilizados bem como os resultados correspondentes a eles são detalhados no capítulo de resultados.

Os valores adimensionais utilizados para os escoamentos de Couette simulados, bem como as propriedades dos métodos numéricos como a tolerância e a quantidade máxima de iterações de Gauss-Seidel são dispostos na tabela \ref{propriedades - couette}, e serão retomados na secção de resultados.

\begin{table}[ht]
    \caption{Condições de contorno para o escoamento de Couette}
    \vspace{0.2cm}
    \centering
    \begin{tabular}{ l c }
    \hline
    {\bfseries Propriedade} & {\bfseries Valor} \\
    \hline
    Velocidade fronteira superior ($v_{ref}$) & 1 $\frac{m}{s}$ \\
    Massa específica ($\rho$) & 1 $\frac{kg}{m^3}$ \\
    Tolerância & $10^{-9}$ \\
    H & 1 [m] \\
    L & 1 [m] \\
    Máximo de iterações & $10^4$ \\
    \hline
    \end{tabular}
    \label{propriedades - couette} \\
    \vspace{0.2cm}
    {\footnotesize Fonte: Do próprio autor}
\end{table}

O domínio utilizado foi de 41x41, sendo que, para garantir um $CFL$ baixo, se utilizou 9600 passos de tempo, o que garantiu a convergência para todos os escoamentos simulados. O tempo computacional final foi de $100 (s)$.

O domínio utilizado é representado na Figura \ref{Couettenumerico}:

\begin{figure}[ht]
    \centering
    \includegraphics[width=0.7\textwidth]{imagens/Couette-computacional.png}
    \caption{Modelo computacional do escoamento de Couette.}
    \label{Couettenumerico}
    {\footnotesize Fonte: Do próprio autor} 
\end{figure}

\section{Poiseuille Plano}

As condições do escoamento de Poiseuille plano são mostrados na tabela \ref{contorno - Poiseuille}, o escoamento de Poiseulle, assim como o de Couette, foi utilizado para testar tantos fluidos newtonianos quanto os não-newtonianos.

\begin{table}[H]
    \caption{Condições de contorno para o escoamento de Poiseuille}
    \vspace{0.2cm}
    \centering
    \begin{tabular}{ c c c c }
    \hline
    {\bfseries Parede/Propriedade} & $ \bm{u}$ & $ \bm{v}$ & $ \bm{p}$ \\
    \hline
    Esquerda & 1ª espécie & 1ª espécie & 2ª espécie \\

    Direita & 2ª espécie & 2ª espécie & 1ª espécie \\

    Superior & 1ª espécie & 1ª espécie & 2ª espécie \\

    Inferior & 1ª espécie & 1ª espécie & 2ª espécie \\
    \hline
    \end{tabular}
    \label{contorno - Poiseuille}\\
    \vspace{0.2cm}
    {\footnotesize Fonte: Do próprio autor}
\end{table}

Pare este escoamento, as condições de contorno são iguais a zero, com exceção da componente $u$ do vetor velocidade à esquerda, que tem um perfil uniforme de velocidade, variável entre os escoamentos.

As mesmas propriedades, que foram variadas no escoamento de Couette, foram variadas para diferentes números de Reynolds neste mesmo escoamento, porém, além destas também se variou a velocidade de entrada a esquerda do escoamento. Como mostrado na secção de resultados.

As dimensões deste escoamento, ilustradas na Figura \ref{Poiseuillenumerico}, são maiores que os demais, pois o perfil de velocidade precisa de espaço e tempo para se desenvolver completamente. Neste caso a escolha do domínio físico pode influenciar no transiente e no resultado, para diferentes números de Reynolds. A escolha de malha utilizada foi de 100x100, para garantir que para altos números de $n$ e de Reynolds, o escoamento possa convergir.

\begin{figure}[ht]
    \centering
    \includegraphics[width=0.5\textwidth]{imagens/Poiseulle-numerico.png}
    \caption{Modelo computacional do escoamento de Poiseulle.}
    \label{Poiseuillenumerico}
    {\footnotesize Fonte: Do próprio autor} 
\end{figure}

As propriedades físicas que não foram variadas para o escoamento de Poiseuille são as mesmas que para o escoamento de Couette, com exceção da dimensão L, de forma que podem ser referenciadas pela tabela \ref{propriedades - Poiseuille}.

\begin{table}[ht]
    \caption{Propriedades físicas para o escoamento de Poiseuille}
    \vspace{0.2cm}
    \centering
    \begin{tabular}{ll}
    \toprule
    {\bfseries Propriedade} & {\bfseries Valor} \\
    \midrule
    Velocidade fronteira superior ($v_{ref}$) & \SI{1}{\meter\per\second} \\
    Massa específica ($\rho$) & \SI{1}{\kilogram\per\cubic\meter} \\
    Tolerância & \SI{1e-9}{} \\
    H & \SI{1}{\meter} \\
    L & \SI{30}{\meter} \\
    Máximo de iterações & $10^4$ \\
    Diferença de pressão ($\Delta p$) & \SI{0.6}{\pascal} \\
    \bottomrule
    \end{tabular}
    \label{propriedades - Poiseuille} \\
    \vspace{0.2cm}
    {\footnotesize Fonte: Do próprio autor}
\end{table}

\section{Cavidade com Tampa Deslizante}

Sendo o escoamento da cavidade com tampa deslizante um escoamento mais complexo que os demais, tem-se que suas condições de contorno, mostradas na Tabela \ref{contorno - lid-driven}, trazem complicações numéricas.

\begin{table}[ht]
    \caption{Condições de contorno para o escoamento de cavidade com tampa deslizante}
    \vspace{0.2cm}
    \centering
    \begin{tabular}{ c c c c }
    \hline
    {\bfseries Parede/Propriedade} & $ \bm{u}$ & $ \bm{v}$ & $ \bm{p}$ \\
    \hline
    Esquerda & 1ª espécie & 1ª espécie & 2ª espécie \\

    Direita & 1ª espécie & 1ª espécie & 2ª espécie \\

    Superior & 1ª espécie & 1ª espécie & 2ª espécie \\

    Inferior & 1ª espécie & 1ª espécie & 2ª espécie \\
    \hline
    \end{tabular}
    \label{contorno - lid-driven}\\
    \vspace{0.2cm}
    {\footnotesize Fonte: Do próprio autor}
\end{table}

Para este escoamento, por ter condições de segunda espécie em todas as direções de $p$, a equação de Poisson resulta em um sistema indeterminado e singular, com determinante igual a zero. Tal condição faz com que a equação da correção da pressão não satisfaça os critérios para a convergência do método de Gauss-Seidel. Esse é um dos exemplos de escoamentos que mostram a instabilidade da equação de correção da pressão e porque, dentre as equações resolvidas para resolver escoamentos, esta é uma das mais difíceis em termos numéricos. 

Para os valores de condição de contorno, é mostrado na Figura \ref{lid-drivennumerico} que todos os valores, são iguais a zero, com exceção da fronteira superior que é igual a velocidade de referência. A malha utilizada variou de acordo com o número de Reynolds, de forma que um maior número de Reynolds exigiu uma malha mais refinada. Pode-se assumir que uma malha de 131x131 resolve todo o intervalo de número de Reynolds proposto.

\begin{figure}[ht]
    \centering
    \includegraphics[width=0.6\textwidth]{imagens/LId-driven-Numerico.png}
    \caption{Modelo computacional da cavidade com tampa deslizante.}
    \label{lid-drivennumerico}
    {\footnotesize Fonte: Do próprio autor} 
\end{figure}

Para este escoamento apenas a viscosidade foi variada a fim de se variar o número de Reynolds, visto que este problema não foi utilizado para validar modelos não-newtonianos. As propriedades constantes são as mesmas que as dos escoamentos de Couette e Poiseuille.

\section{Cavidade Diferencialmente Aquecida}

Para este escoamento, se faz necessário a definição das condições de contorno de mais uma propriedade, a temperatura. As condições de contorno são descritas pela tabela \ref{contorno - thermal}:

\begin{table}[ht]
    \caption{Condições de contorno para cavidade diferencialmente aquecida}
    \vspace{0.2cm}
    \centering
    \begin{tabular}{ c c c c c }
    \hline
    \bfseries {Parede/Propriedade} & $ \bm{u}$ & $ \bm{v}$ & $ \bm{p}$ & $ \bm{T}$ \\
    \hline
    Esquerda & 1ª espécie & 1ª espécie & 2ª espécie & 1ª espécie \\

    Direita & 1ª espécie & 1ª espécie & 2ª espécie & 1ª espécie \\

    Superior & 1ª espécie & 1ª espécie & 2ª espécie & 2ª espécie \\

    Inferior & 1ª espécie & 1ª espécie & 2ª espécie & 2ª espécie \\
    \hline
    \end{tabular}
    \label{contorno - thermal}\\
    \vspace{0.2cm}
    {\footnotesize Fonte: Do próprio autor}
\end{table}

Os valores utilizados para as condições de contorno são zero em todos em todas as direções para todas as propriedades com exceção da temperatura. A temperatura das paredes esquerda e direita são definidas como $80 ^\circ\mathrm{C}$ e $50 ^\circ\mathrm{C}$  respectivamente e permanecidas constantes durante todo o escoamento.

As propriedades físicas utilizadas são descritas pela Tabela \ref{propriedades - thermal}.

\begin{table}[ht]
    \caption{Propriedades físicas utilizadas do escoamento de cavidade diferencialmente aquecida}
    \vspace{0.2cm}
    \centering
    \begin{tabular}{l l}
    \toprule
    {\bfseries Propriedade} & {\bfseries Valor} \\
    \midrule
    Velocidade fronteira superior ($v_{ref}$) & \SI{1}{\meter\per\second} \\
    Massa específica ($\rho$) & \SI{1}{\kilogram\per\cubic\meter} \\
    Tolerância & \SI{1e-9}{} \\
    H & \SI{1}{\meter} \\
    L & \SI{1}{\meter} \\
    Máximo de iterações & $10^4$ \\
    $\alpha$ & \SI{0.01}{\meter\squared\per\second} \\
    $g$ & \SI{9.81}{\meter\per\second\squared} \\
    $\nu$ & \SI{0.0071}{\meter\squared\per\second} \\
    $Pr = \frac{\nu}{\alpha}$ & 0.71 \\
    \bottomrule
    \end{tabular}
    \label{propriedades - thermal} \\
    \vspace{0.2cm}
    {\footnotesize Fonte: Do próprio autor}
\end{table}

O coeficiente de expansão térmica é variável e calculado a partir da seguinte equação, Eq. \ref{equação para betha}:

\begin{equation}
\beta = \frac{Ra \cdot \nu  \cdot \alpha}{g \cdot (T_{quente} - T_{frio}) \cdot H^3}
    \label{equação para betha}
\end{equation}

De forma que se possa impor o valor de coeficiente de Rayligh para diferentes condições de escoamento. Possibilitando comparar estes valores com a literatura. 

A Figura \ref{Thermal-numerico} exemplifica o domínio físico utilizado para as equações, é possível observar a semelhança com o escoamento anterior, mudando apenas a velocidade da fronteira superior e a questão térmica do escoamento:

\begin{figure}[ht]
    \centering
    \includegraphics[width=0.65\textwidth]{imagens/thermal-numerico.png}
    \caption{Modelo computacional da cavidade térmica.}
    \label{Thermal-numerico}
    {\footnotesize Fonte: Do próprio autor} 
\end{figure}

A malha computacional utilizada para este escoamento foi de 100x100, a fim de se garantir que todo o intervalo de números de Rayleigh fosse atingido. 

\section{Algoritmos de Solução de Sistemas Lineares}

A implementação do método de Gauss-seidel é feita da como mostra a Figura \ref{Algoritmo-Seidel}

\begin{figure}[ht]
    \centering
    \includegraphics[height=0.5\textwidth]{imagens/loop-Gauss-Seidel.png}
    \caption{Lógica do algoritmo de Gauss-Seidel utilizado no código.}
    \label{Algoritmo-Seidel}
    {\footnotesize Fonte: Do próprio autor} 
\end{figure}

Inicialmente, o método começa com a atualização das condições de contorno, garantindo que os valores das células fantasmas sejam atualizadas de forma correta. Em seguida, se aplica a equação \ref{sistema gauss-seidel}, onde cada incógnita é atualizada de forma sucessiva com base nos valores mais recentes das demais incógnitas. Após cada iteração, calcula-se o erro, comparando a solução atual com a obtida no passo anterior. O processo continua até que um dos critérios de parada do número máximo de equações ou o erro menor que o especificado seja atingido. Por ser um método iterativo relativamente simples, o Gauss-Seidel requer menos etapas de implementação em comparação com os gradientes conjugados.

O método dos gradientes conjugados, por ser mais complexo, requer mais etapas para sua implementação. seu algorítimo é representado pela Figura \ref{Algoritmo-Conjulgados}.

\begin{figure}[ht]
    \centering
    \includegraphics[height=1.0\textwidth]{imagens/loop-gradientes.png}
    \caption{Diagrama de blocos do método dos gradientes conjugados.}
    \label{Algoritmo-Conjulgados}
    {\footnotesize Fonte: Do próprio autor} 
\end{figure}


Inicialmente se inicializa os vetores de resíduo $r$ e o vetor conjugado $p$ com o estado inicial do sistema. Na rotina desenvolvida se utilizou como chute inicial para essa inicialização o passo de tempo anterior para acelerar a convergência do método. Após isto se calcula um passo $\alpha$ e se desloca os vetores $x$ e $r$ na direção $p$ com um passo $\alpha$. Caso o critério de tolerância não tenha sido atingido, se calcula uma correção $\beta_k$ de forma que se possa iniciar o passo de tempo posterior

Devido a teoria de gradientes conjugados garantir a convergência do método em no máximo $n$ iterações para um sistema de $n$ variáveis, não se utilizou o critério de número máximo de iterações para a parada, uma vez que caso a convergência não tenha sido atingida antes desse valor o método falhou em resolver o sistema. Neste caso o código é interrompido.

