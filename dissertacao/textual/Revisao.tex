\chapter{Revisão Bibliográfica}

\section{Escoamentos em Tubulações}

Para compreender a dinâmica dos escoamentos bifásicos em tubulações, é útil dmapear os regimes de escoamento de acordo com as propriedades do escoamento e do fluido. Uma classificação comum desses regimes são: bolhas dispersas (\textit{Bubbly flow}), bolhas golfadas (\textit{Slug flow}), padrão caôtico (\textit{Churn Flow}), anular (\textit{Annular flow}) e escoamento em névoa (\textit{Mist flow}).


\begin{figure}[ht]
    \centering
    \includegraphics[width=0.7\textwidth]{imagens/regimes-de-escoamento.png.png}
    \caption{Diferentes regimes de escoamentos A) bolhas dispersas B) bolhas golfadas C) padrão caôtico D) anular E) escoamento em névoa}
    \label{regimes-de-escoamento}
    {\footnotesize Fonte: \citet{Annechien2024} adaptado} 
\end{figure}

VOF é citado por PUCKETT


\section{Métodos de Simulação para Filmes Finos}


\section{Método Multiníveis}