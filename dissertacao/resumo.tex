%====================================================================
% Resumo em português: Escreva no ambiente abaixo o resumo colocando 
% no final, via comando \palavraschaveportugues, as palavras-chave
% em português. Não coloque ponto final no final das palavras-chave
% pois o comando já insere esse ponto.
%  
%  Exemplo:
%  
%  \begin{resumo-portugues}
%     O presente trabalho foi desenvolvido com a finalidade de 
%     realizar um estudo sobre diversas metodologias de análise 
%     de dados.
%
%     \palavraschaveportugues{Análise de dados, inferência}
%  \end{resumo-portugues}
%====================================================================

\begin{resumo-portugues}

Um dos principais desafios no contexto de aplicação dos métodos de Dinâmica dos Fluidos Computacional ($DFC$), é a compreensão dos métodos e modelos utilizados nos diferentes tipos de escoamentos simulados. Diante deste contexto, no presente trabalho apresenta-se a metodologia utilizada para a construção de um código computacional voltado à solução de escoamentos newtonianos com e sem a transferência de energia térmica, além de escoamentos não-newtonianos. Inicialmente, apresentou-se um detalhamento a respeito do modelo físico dos escoamentos utilizados para validar as rotinas implementadas, após ter detalhado as equações para modelar os problemas abordados no presente trabalho, trazendo um breve contexto histórico dos mesmos, bem como as especificidades tratadas na literatura a respeito destes. Após esta etapa foram detalhados os métodos numéricos utilizados para discretizar as equações, bem como a metodologia de malha e os métodos para solução de sistemas lineares. Por fim, são apresentados os resultados obtidos a partir da simulação dos problemas elencados, bem como a comparação destes com a literatura e com modelos analíticos. Os mesmos se mostraram coerentes com a literatura, em específico, nos escoamentos não-newtonianos, onde observou-se a influência na escolha do invariante na qualidade dos resultados para os diferentes regimes de escoamento. O código corroborou para o entendimento dos desafios da implementação dos modelos e das limitações numéricas dos métodos utilizados. O trabalho foi desenvolvido no Laboratório de Mecânica dos Fluidos da Faculdade de Engenharia Mecânica da Universidade Federal de Uberlândia ($MFLab$-$FEMEC$-$UFU$).

\palavraschaveportugues{ Dinâmica dos Fluidos Computacional, Fluidos Não-Newtonianos, Escoamentos Térmicos, Volumes finitos}
	
\end{resumo-portugues}

